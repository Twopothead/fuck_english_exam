\subsection{Text 1}
Wild Bill Donovan would have loved the Internet. The American spymaster who built the Office of Strategic Services in the World War II and later laid the roots for the CIA was fascinated with information. Donovan believed in using whatever tools came to hand in the “great game” of espionage -- spying as a “profession.” These days the Net, which has already re-made such everyday pastimes as buying books and sending mail, is reshaping Donovan’s vocation as well.

The latest revolution isn’t simply a matter of gentlemen reading other gentlemen’s e-mail. That kind of electronic spying has been going on for decades. In the past three or four years, the World Wide Web has given birth to a whole industry of point-and-click spying. The spooks call it “open source intelligence,” and as the Net grows, it is becoming increasingly influential. In 1995 the CIA held a contest to see who could compile the most data about Burundi. The winner, by a large margin, was a tiny Virginia company called Open-Source Solutions, whose clear advantage was its mastery of the electronic world.

Among the firms making the biggest splash in the new world is Straitford, Inc., a private intelligence-analysis firm based in Austin, Texas. Straitford makes money by selling the results of spying (covering nations from Chile to Russia) to corporations like energy-services firm McDermott International. Many of its predictions are available online at www.straitford.com.

Straitford president George Friedman says he sees the online world as a kind of mutually reinforcing tool for both information collection and distribution, a spymaster’s dream. Last week his firm was busy vacuuming up data bits from the far corners of the world and predicting a crisis in Ukraine. “As soon as that report runs, we’ll suddenly get 500 new internet sign-ups from Ukraine,” says Friedman, a former political science professor. “And we’ll hear back from some of them.” Open-source spying does have its risks, of course, since it can be difficult to tell good information from bad. That’s where Straitford earns its keep.

Friedman relies on a lean staff in Austin. Several of his staff members have military-intelligence backgrounds. He sees the firm’s outsider status as the key to its success. Straitford’s briefs don’t sound like the usual Washington back-and-forthing, whereby agencies avoid dramatic declarations on the chance they might be wrong. Straitford, says Friedman, takes pride in its independent voice.

\begin{questions} \sethlcolor{cyan}
\question 41.	To make your humor work, you should \ltk{}.\\
\fourch{  take advantage of different kinds of audience
}{ make fun of the disorganized people
}{ address different problems to different people
}{ show sympathy for your listeners
}

\question 42.	The joke about doctors implies that, in the eyes of nurses, they are \ltk{}.\\
\fourch{  impolite to new arrivals
}{ very conscious of their godlike role
}{ entitled to some privileges
}{ very busy even during lunch hours
}

\question 43.	It can be inferred from the text that public services \ltk{}.\\
\fourch{  have benefited many people
}{ are the focus of public attention
}{ are an inappropriate subject for humor
}{ have often been the laughing stock
}

\question 44.	To achieve the desired result, humorous stories should be delivered \ltk{}.\\
\fourch{  in well-worded language
}{ as awkwardly as possible
}{ in exaggerated statements
}{ as casually as possible
}

\question 45.	The best title for the text may be \ltk{}.\\
\fourch{  Use Humor Effectively
}{ Various Kinds of Humor
}{ Add Humor to Speech
}{ Different Humor Strategies
}\end{questions}

\subsection{Text 2}
Since the dawn of human ingenuity, people have devised ever more cunning tools to cope with work that is dangerous, boring, burdensome, or just plain nasty. That compulsion has resulted in robotics -- the science of conferring various human capabilities on machines. And if scientists have yet to create the mechanical version of science fiction, they have begun to come close.

As a result, the modern world is increasingly populated by intelligent gizmos whose presence we barely notice but whose universal existence has removed much human labor. Our factories hum to the rhythm of robot assembly arms. Our banking is done at automated teller terminals that thank us with mechanical politeness for the transaction. Our subway trains are controlled by tireless robot-drivers. And thanks to the continual miniaturization of electronics and micro-mechanics, there are already robot systems that can perform some kinds of brain and bone surgery with submillimeter accuracy -- far greater precision than highly skilled physicians can achieve with their hands alone.

But if robots are to reach the next stage of laborsaving utility, they will have to operate with less human supervision and be able to make at least a few decisions for themselves -- goals that pose a real challenge. “While we know how to tell a robot to handle a specific error,” says Dave Lavery, manager of a robotics program at NASA, “we can’t yet give a robot enough ‘common sense’ to reliably interact with a dynamic world.”

Indeed the quest for true artificial intelligence has produced very mixed results. Despite a spell of initial optimism in the 1960s and 1970s when it appeared that transistor circuits and microprocessors might be able to copy the action of the human brain by the year 2010, researchers lately have begun to extend that forecast by decades if not centuries.

What they found, in attempting to model thought, is that the human brain’s roughly one hundred billion nerve cells are much more talented -- and human perception far more complicated -- than previously imagined. They have built robots that can recognize the error of a machine panel by a fraction of a millimeter in a controlled factory environment. But the human mind can glimpse a rapidly changing scene and immediately disregard the 98 percent that is irrelevant, instantaneously focusing on the monkey at the side of a winding forest road or the single suspicious face in a big crowd. The most advanced computer systems on Earth can’t approach that kind of ability, and neuroscientists still don’t know quite how we do it.
\begin{questions} \sethlcolor{cyan}
\question 46.	Human ingenuity was initially demonstrated in \ltk{}.\\
\fourch{  the use of machines to produce science fiction
}{ the wide use of machines in manufacturing industry
}{ the invention of tools for difficult and dangerous work
}{ the elite’s cunning tackling of dangerous and boring work
}

\question 47.	The word “gizmos” (line 1, paragraph 2) most probably means \ltk{}.\\
\fourch{  programs
}{ experts
}{ devices
}{ creatures
}

\question 48.	According to the text, what is beyond man’s ability now is to design a robot that can \ltk{}.\\
\fourch{  fulfill delicate tasks like performing brain surgery
}{ interact with human beings verbally
}{ have a little common sense
}{ respond independently to a changing world
}

\question 49.	Besides reducing human labor, robots can also \ltk{}.\\
\fourch{  make a few decisions for themselves
}{ deal with some errors with human intervention
}{ improve factory environments
}{ cultivate human creativity
}

\question 50.	The author uses the example of a monkey to argue that robots are \ltk{}.\\
\fourch{  expected to copy human brain in internal structure
}{ able to perceive abnormalities immediately
}{ far less able than human brain in focusing on relevant information
}{ best used in a controlled environment
}\end{questions}

\subsection{Text 3}
Could the bad old days of economic decline be about to return? Since OPEC agreed to supply-cuts in March, the price of crude oil has jumped to almost $26 a barrel, up from less than $10 last December. This near-tripling of oil prices calls up scary memories of the 1973 oil shock, when prices quadrupled, and 1979-80, when they also almost tripled. Both previous shocks resulted in double-digit inflation and global economic decline. So where are the headlines warning of gloom and doom this time?

The oil price was given another push up this week when Iraq suspended oil exports. Strengthening economic growth, at the same time as winter grips the northern hemisphere, could push the price higher still in the short term.

Yet there are good reasons to expect the economic consequences now to be less severe than in the 1970s. In most countries the cost of crude oil now accounts for a smaller share of the price of petrol than it did in the 1970s. In Europe, taxes account for up to four-fifths of the retail price, so even quite big changes in the price of crude have a more muted effect on pump prices than in the past.

Rich economies are also less dependent on oil than they were, and so less sensitive to swings in the oil price. Energy conservation, a shift to other fuels and a decline in the importance of heavy, energy-intensive industries have reduced oil consumption. Software, consultancy and mobile telephones use far less oil than steel or car production. For each dollar of GDP (in constant prices) rich economies now use nearly 50% less oil than in 1973. The OECD estimates in its latest Economic Outlook that, if oil prices averaged $22 a barrel for a full year, compared with $13 in 1998, this would increase the oil import bill in rich economies by only 0.25-0.5% of GDP. That is less than one-quarter of the income loss in 1974 or 1980. On the other hand, oil-importing emerging economies -- to which heavy industry has shifted -- have become more energy-intensive, and so could be more seriously squeezed.

One more reason not to lose sleep over the rise in oil prices is that, unlike the rises in the 1970s, it has not occurred against the background of general commodity-price inflation and global excess demand. A sizable portion of the world is only just emerging from economic decline. The Economist’s commodity price index is broadly unchanging from a year ago. In 1973 commodity prices jumped by 70%, and in 1979 by almost 30%.
\begin{questions} \sethlcolor{cyan}
    \question 51.	The main reason for the latest rise of oil price is \ltk{}.\\
\fourch{  global inflation
}{ reduction in supply
}{ fast growth in economy
}{ Iraq’s suspension of exports
}

\question 52.	It can be inferred from the text that the retail price of petrol will go up dramatically if \ltk{}.\\
\fourch{  price of crude rises
}{ commodity prices rise
}{ consumption rises
}{ oil taxes rise
}

\question 53.	The estimates in Economic Outlook show that in rich countries \ltk{}.\\
\fourch{  heavy industry becomes more energy-intensive
}{ income loss mainly results from fluctuating crude oil prices
}{ manufacturing industry has been seriously squeezed
}{ oil price changes have no significant impact on GDP
}

\question 54.	We can draw a conclusion from the text that \ltk{}.\\
\fourch{  oil-price shocks are less shocking now
}{ inflation seems irrelevant to oil-price shocks
}{ energy conservation can keep down the oil prices
}{ the price rise of crude leads to the shrinking of heavy industry
}

\question 55.	From the text we can see that the writer seems \ltk{}.\\
\fourch{  optimistic
}{ sensitive
}{ gloomy
}{ scared
}\end{questions}

\subsection{Text 4}
The Supreme Court’s decisions on physician-assisted suicide carry important implications for how medicine seeks to relieve dying patients of pain and suffering.
Although it ruled that there is no constitutional right to physician-assisted suicide, the Court in effect supported the medical principle of “double effect,” a centuries-old moral principle holding that an action having two effects -- a good one that is intended and a harmful one that is foreseen -- is permissible if the actor intends only the good effect.
Doctors have used that principle in recent years to justify using high doses of morphine to control terminally ill patients’ pain, even though increasing dosages will eventually kill the patient.
Nancy Dubler, director of Montefiore Medical Center, contends that the principle will shield doctors who “until now have very, very strongly insisted that they could not give patients sufficient mediation to control their pain if that might hasten death.”
George Annas, chair of the health law department at Boston University, maintains that, as long as a doctor prescribes a drug for a legitimate medical purpose, the doctor has done nothing illegal even if the patient uses the drug to hasten death. “It’s like surgery,” he says. “We don’t call those deaths homicides because the doctors didn’t intend to kill their patients, although they risked their death. If you’re a physician, you can risk your patient’s suicide as long as you don’t intend their suicide.”
On another level, many in the medical community acknowledge that the assisted-suicide debate has been fueled in part by the despair of patients for whom modern medicine has prolonged the physical agony of dying.
Just three weeks before the Court’s ruling on physician-assisted suicide, the National Academy of Science (NAS) released a two-volume report, Approaching Death: Improving Care at the End of Life. It identifies the undertreatment of pain and the aggressive use of “ineffectual and forced medical procedures that may prolong and even dishonor the period of dying” as the twin problems of end-of-life care.
The profession is taking steps to require young doctors to train in hospices, to test knowledge of aggressive pain management therapies, to develop a Medicare billing code for hospital-based care, and to develop new standards for assessing and treating pain at the end of life.
Annas says lawyers can play a key role in insisting that these well-meaning medical initiatives translate into better care. “Large numbers of physicians seem unconcerned with the pain their patients are needlessly and predictably suffering,” to the extent that it constitutes “systematic patient abuse.” He says medical licensing boards “must make it clear… that painful deaths are presumptively ones that are incompetently managed and should result in license suspension.”
\begin{questions} \sethlcolor{cyan}
    \question  56.	From the first three paragraphs, we learn that \ltk{}.\\
\fourch{  doctors used to increase drug dosages to control their patients’ pain
}{ it is still illegal for doctors to help the dying end their lives
}{ the Supreme Court strongly opposes physician-assisted suicide
}{ patients have no constitutional right to commit suicide
}

\question 57.	Which of the following statements is true according to the text?
\fourch{  Doctors will be held guilty if they risk their patients’ death.
}{ Modern medicine has assisted terminally ill patients in painless recovery.
}{ The Court ruled that high-dosage pain-relieving medication can be prescribed.
}{ A doctor’s medication is no longer justified by his intentions.
}

\question 58.	According to the NAS’s report, one of the problems in end-of-life care is \ltk{}.\\
\fourch{  prolonged medical procedures
}{ inadequate treatment of pain
}{ systematic drug abuse
}{ insufficient hospital care
}

\question 59.	Which of the following best defines the word “aggressive” (line 3, paragraph 7)?
\fourch{  Bold
}{ Harmful
}{ Careless
}{ Desperate
}

\question 60.	George Annas would probably agree that doctors should be punished if they \ltk{}.\\
\fourch{  manage their patients incompetently
}{ give patients more medicine than needed
}{ reduce drug dosages for their patients
}{ prolong the needless suffering of the patients
}\end{questions}