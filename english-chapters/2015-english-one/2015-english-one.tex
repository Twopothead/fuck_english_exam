  
\section{2015}
\subsection{Text 1}
    King Juan Carlos of Spain once insisted “kings don’t abdicate, they die in their sleep.” But embarrassing scandals and the popularity of the republican left in the recent Euro-elections have forced him to eat his words and stand down. So, does the Spanish crisis suggest that monarchy is seeing its last days? Does that mean the writing is on the wall for all European royals, with their magnificent uniforms and majestic lifestyles?
    The Spanish case provides arguments both for and against monarchy. When public opinion is particularly polarised, as it was following the end of the Franco regime, monarchs can rise above “mere”politics and “embody” a spirit of national unity.
     It is this apparent transcendence of politics that explains monarchs’ continuing popularity as heads of state. And so, the Middle East excepted, Europe is the most monarch-infested region in the world, with 10 kingdoms (not counting Vatican City and Andorra). But unlike their absolutist counterparts in the Gulf and Asia, most royal families have survived because they allow voters to avoid the difficult search for a non-controversial but respected public figure.
    Even so, kings and queens undoubtedly have a downside. Symbolic of national unity as they claim to be, their very history—and sometimes the way they behave today——embodies outdated and indefensible privileges and inequalities. At a time when Thomas Piketty and other economists are warning of rising inequality and the increasing power of inherited wealth, it is bizarre that wealthy aristocratic families should still be the symbolic heart of modern democratic states.
    The most successful monarchies strive to abandon or hide their old aristocratic ways. Princes and princesses have day-jobs and ride bicycles, not horses (or helicopters). Even so, these are wealthy families who party with the international 1\%, and media intrusiveness makes it increasingly difficult to maintain the right image.
    While Europe’s monarchies will no doubt be smart enough to survive for some time to come, it is the British royals who have most to fear from the Spanish example.
     It is only the Queen who has preserved the monarchy’s reputation with her rather ordinary (if well-heeled) granny style. The danger will come with Charles, who has both an expensive taste of lifestyle and a pretty hierarchical view of the world. He has failed to understand that monarchies have largely survived because they provide a service——as non-controversial and non-political heads of state. Charles ought to know that as English history shows, it is kings, not republicans, who are the monarchy’s worst enemies.
\begin{questions} \sethlcolor{cyan}\question 21. According to the first two Paragraphs, King Juan Carlosof Spain\ltk{}.
\\ \fourch{ used to enjoy high public support
}{ was unpopular among European royals
}{ eased his relationship with his rivals
}{ ended his reign in embarrassment
}\question 22. Monarchs are kept as heads of state in Europe mostly\ltk{}.
\\ \fourch{ owing to their undoubted and respectable status
}{ to achieve a balance between tradition and reality
}{ to give voters more public figures to look up to
}{ due to their everlasting political embodiment
}\question 23.  Which of the following is shown to be odd, according to Paragraph 4?
\\ \fourch{ Aristocrats’ excessive reliance on inherited wealth.
}{ The role of the nobility in modern democracies.
}{ The simple lifestyle of the aristocratic families.
}{ The nobility’s adherence to their privileges.
}\question 24. The British royals “have most to fear” because Charles\ltk{}.
\\ \fourch{ takes a rough line on political issues
}{ fails to change his lifestyle as advised
}{ takes republicans as his potential allies
}{ fails to adapt himself to his future role
}\question 25. Which of the following is the best title of the text?
\\ \fourch{ Carlos, Glory and Disgrace Combined
}{ Charles, Anxious to Succeed to the Throne
}{ Carlos, a Lesson for All European Monarchs
}{ Charles, Slow to React to the Coming Threats
}\end{questions}
\subsection{Text 2}
   Just how much does the Constitution protect your digital data? The Supreme Court will now consider whether police can search the contents of a mobile phone without a warrant if the phone is on or around a person during an arrest.
   California has asked the justices to refrain from a sweeping ruling, particularly one that upsets the old assumption that authorities may search through the possessions of suspects at the time of their arrest. It is hard, the state argues, for judges to assess the implications of new and rapidly changing technologies.
   The court would be recklessly modest if it followed California’s advice. Enough of the implications are discernable, even obvious, so that the justices can and should provide updated guidelines to police,lawyers and defendants.
   They should start by discarding California’s lame argument that exploring the contents of a smartphone — a vast storehouse of digital information — is similar to, say, going through a suspect’s purse. The court has ruled that police don’t violate the Fourth Amendment when they go through the wallet or pocketbook of an arrestee without a warrant. But exploring one’s smartphone is more like entering his or her home. A smartphone may contain an arrestee’s reading history, financial history, medical history and comprehensive records of recent correspondence. The development of “cloud computing,” meanwhile, has made that exploration so much the easier.
   Americans should take steps to protect their digital privacy. But keeping sensitive information on these devices is increasingly a requirement of normal life. Citizens still have a right to expect private documents to remain private and protected by the Constitution’s prohibition on unreasonable searches.
   As so often is the case, stating that principle doesn’t ease the challenge of line-drawing. In many cases, it would not be overly burdensome for authorities to obtain a warrant to search through phone contents. They could still invalidate Fourth Amendment protections when facing severe,urgent circumstances, and they could take reasonable measures to ensure that phone data are not erased or altered while waiting for a warrant. The court, though, may want to allow room for police to cite situations where they are entitled to more freedom.
    But the justices should not swallow California’s argument whole. New, disruptive technology sometimes demands novel applications of the Constitution’s protections. Orin Kerr, a law professor, compares the explosion and accessibility of digital information in the 21st century with the establishment of automobile use as a virtual necessity of life in the 20th: The justices had to specify novel rules for the new personal domain of the passenger car then; they must sort out how the Fourth Amendment applies to digital information now.	
\begin{questions} \sethlcolor{cyan}\question 26. The Supreme Court will work out whether, during an arrest, it is legitimate to\ltk{}.
\\ \fourch{ prevent suspects from deleting their phone contents
}{ search for suspects’ mobile phones without a warrant
}{ check suspects’ phone contents without being authorized
}{ prohibit suspects from using their mobile phones
}\question 27. The author’s attitude toward California’s argument is one of\ltk{}.
\\ \fourch{ disapproval
}{ indifference
}{ tolerance
}{ cautiousness
}\question 28. The author believes that exploring one’s phone contents is comparable to\ltk{}.
\\ \fourch{ getting into one’s residence
}{ handling one’s historical records
}{ scanning one’s correspondences
}{ going through one’s wallet
}\question 29. In Paragraph 5 and 6, the author shows his concern that\ltk{}.
\\ \fourch{ principles are hard to be clearly expressed
}{ the court is giving police less room for action
}{ citizens’ privacy is not effectively protected
}{ phones are used to store sensitive information
}\question 30. Orin Kerr’s comparison is quoted to indicate that\ltk{}.
\\ \fourch{ the Constitution should be implemented flexibly
}{ new technology requires reinterpretation of the Constitution
}{ California’s argument violates principles of the Constitution
}{ principles of the Constitution should never be altered
}
\end{questions}
\subsection{Text 3}
    The journal Science is adding an extra round of statistical checks to its peer-review process, editor-in-chief Marcia McNutt announced today. The policy follows similar efforts from other journals, after widespread concern that basic mistakes in data analysis are contributing to the irreproducibility of many published research findings.
    “Readers must have confidence in the conclusions published in our journal,” writes McNutt in an editorial. Working with the American Statistical Association, the journal has appointed seven experts to a statistic board of reviewing editors (SBoRE). Manuscript will beflagged upfor additional scrutiny by the journal’s internal editors, or by its existing Board of Reviewing Editors or by outside peer reviewers. The SBoRE panel will then find external statisticians to review these manuscripts.
    Asked whether any particular papers had impelled the change, McNutt said: “The creation of the ‘statistics board’ was motivated by concerns broadly with the application of statistics and data analysis in scientific research and is part of Science’s overall drive to increase reproducibility in the research we publish.”
   Giovanni Parmigiani, a biostatistician at the Harvard School of Public Health, a member of the SBoRE group, says he expects the board to “play primarily an advisory role.” He agreed to join because he “found the foresight behind the establishment of the SBoRE to be novel, unique and likely to have a lasting impact. This impact will not only be through the publications in Science itself, but hopefully through a larger group of publishing places that may want to model their approach after Science.”
   John Ioannidis, a physician who studies research methodology, says that the policy is “a most welcome step forward” and “long overdue.” “Most journals are weak in statistical review, and this damages the quality of what they publish. I think that, for the majority of scientific papers nowadays, statistical review is more essential than expert review,” he says. But he noted that biomedical journals such asAnnals of Internal Medicine, the Journal of the American Medical Association and The Lancet pay strong attention to statistical review.
   Professional scientists are expected to know how to analyze data, but statistical errors are alarmingly common in published research, according to David Vaux, a cell biologist. Researchers should improve their standards, he wrote in 2012, but journals should also take a tougher line, “engaging reviewers who are statistically literate and editors who can verify the process.” Vaux says thatScience’s idea to pass some papers to statisticians “has some merit, but a weakness is that it relies on the board of reviewing editors to identify ‘the papers that need scrutiny’ in the first place”.
\begin{questions} \sethlcolor{cyan}\question 31. It can be learned from Paragraph 1 that\ltk{}.
\\ \fourch{ Science intends to simplify its peer-review process
}{ journals are strengthening their statistical checks
}{ few journals are blamed for mistakes in data analysis
}{ lack of data analysis is common in research projects
}\question 32. The phrase “flagged up” (Para. 2) is the closest in meaning to\ltk{}.
\\ \fourch{ found    
}{ marked
}{ revised
}{ stored
}\question   33. Giovanni Parmigiani believes that the establishment of the SBoRE may\ltk{}.
\\ \fourch{ pose a threat to all its peers
}{ meet with strong opposition
}{ increase Science’s circulation
}{ set an example for other journals
}\question 34. David Vaux holds that what Science is doing now\ltk{}.
\\ \fourch{ adds to researchers’ workload
}{ diminishes the role of reviewers
}{ has room for further improvement
}{ is to fail in the foreseeable future
}\question 35. Which of the following is the best title of the text?
\\ \fourch{ Science Joins Push to Screen Statistics in Papers
}{ Professional Statisticians Deserve More Respect
}{ Data Analysis Finds Its Way onto Editors’ Desks
}{ Statisticians Are Coming Back with Science
}\end{questions}
\subsection{Text 4}
   Two years ago, Rupert Murdoch’s daughter, Elisabeth, spoke of the “unsettling dearth of integrity across so many of our institutions.” Integrity had collapsed, she argued, because of a collective acceptance that the only “sorting mechanism” in society should be profit and the market. But “it’s us, human beings, we the people who create the society we want, not profit.”
   Driving her point home, she continued: “It’s increasingly apparent that the absence of purpose, of a moral language within government, media or business could become one of the most dangerous goals for capitalism and freedom.” This same absence of moral purpose was wounding companies such as News International, she thought, making it more likely that it would lose its way as it had with widespread illegal telephone hacking .
   As the hacking trial concludes——finding guilty one ex-editor of the News of the World, Andy Coulson, for conspiring to hack phones, and finding his predecessor, Rebekah Brooks, innocent of the same charge —the wider issue of dearth of integrity still stands, Journalists are known to have hacked the phones of up to 5,500 people. This is hacking on an industrial scale, as was acknowledged by Glenn Mulcaire, the man hired by the News of the World in 2001 to be the point person for phone hacking. Others await trial. This long story still unfolds.
   In many respects, the dearth of moral purpose frames not only the fact of such widespread phone hacking but the terms on which the trial took place. One of the astonishing revelations was how little Rebekah Brooks knew of what went on in her newsroom, how little she thought to ask and the fact that she never inquired how the stories arrived. The core of her successful defence was that she knew nothing.
   In today’s world, it has become normal that well-paid executives should not be accountable for what happens in the organizations that they run. Perhaps we should not be so surprised. For a generation, the collective doctrine has been that the sorting mechanism of society should be profit. The words that have mattered are efficiency, flexibility, shareholder value, business–friendly, wealth generation, sales, impact and, in newspapers, circulation. Words degraded to the margin have been justice, fairness, tolerance, proportionality and accountability.
   The purpose of editing the News of the World was not to promote reader understanding, to be fair in what was written or to betray any common humanity. It was to ruin lives in the quest for circulation and impact. Ms Brooks may or may not have had suspicions about how her journalists got their stories, but she asked no questions, gave no instructions—nor received traceable, recorded answers.
\begin{questions} \sethlcolor{cyan}\question 36. According to the first two paragraphs, Elisabeth was upset by\ltk{}.
\\ \fourch{ the consequences of the current sorting mechanism
}{ companies’ financial loss due to immoral practices
}{ governmental ineffectiveness on moral issues
}{ the wide misuse of integrity among institutions
 }\question 37. It can be inferred from Paragraph 3 that\ltk{}.
\\ \fourch{ Glem Mulcaire may deny phone hacking as a crime
}{ more journalists may be found guilty of phone hacking
}{ Andy Coulson should be held innocent of the charge
}{ phone hacking will be accepted on certain occasions
}\question 38. The author believes the Rebekah Books’s defence\ltk{}.
\\ \fourch{ revealed a cunning personality
}{ centered on trivial issues
}{ was hardly convincing
}{ was part of a conspiracy
}\question 39. The author holds that the current collective doctrine shows\ltk{}.
\\ \fourch{ generally distorted values
}{ unfair wealth distribution
}{ a marginalized lifestyle
}{ a rigid moral code
}\question 40. Which of the following is suggested in the last paragraph?
\\ \fourch{ The quality of writing is of primary importance.
}{ Common humanity is central to news reporting.
}{ Moral awareness matters in editing a newspaper.
}{ Journalists need stricter industrial regulations.
}\end{questions}    
