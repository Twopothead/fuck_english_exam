
\section{1999}
\subsection{Text 1}
It’s a rough world out there. Step outside and you could break a leg slipping on your doormat. Light up the stove and you could burn down the house. Luckily, if the doormat or stove failed to warn of coming disaster, a successful lawsuit might compensate you for your troubles. Or so the thinking has gone since the early 1980s, when juries began holding more companies liable for their customers’ misfortunes.

Feeling threatened, companies responded by writing ever-longer warning labels, trying to anticipate every possible accident. Today, stepladders carry labels several inches long that warn, among other things, that you might -- surprise! -- fall off. The label on a child’s Batman cape cautions that the toy “does not enable user to fly.”

While warnings are often appropriate and necessary -- the dangers of drug interactions, for example -- and many are required by state or federal regulations, it isn’t clear that they actually protect the manufacturers and sellers from liability if a customer is injured. About 50 percent of the companies lose when injured customers take them to court.

Now the tide appears to be turning. As personal injury claims continue as before, some courts are beginning to side with defendants, especially in cases where a warning label probably wouldn’t have changed anything. In May, Julie Nimmons, president of Schutt Sports in Illinois, successfully fought a lawsuit involving a football player who was paralyzed in a game while wearing a Schutt helmet. “We’re really sorry he has become paralyzed, but helmets aren’t designed to prevent those kinds of injuries,” says Nimmons. The jury agreed that the nature of the game, not the helmet, was the reason for the athlete’s injury. At the same time, the American Law Institute -- a group of judges, lawyers, and academics whose recommendations carry substantial weight -- issued new guidelines for tort law stating that companies need not warn customers of obvious dangers or bombard them with a lengthy list of possible ones. “Important information can get buried in a sea of trivialities,” says a law professor at Cornell law School who helped draft the new guidelines. If the moderate end of the legal community has its way, the information on products might actually be provided for the benefit of customers and not as protection against legal liability.
\begin{questions} \sethlcolor{cyan} \question 52.	What were things like in 1980s when accidents happened? \\
\fourch{ Customers might be relieved of their disasters through lawsuits.
}{ Injured customers could expect protection from the legal system.
}{ Companies would avoid being sued by providing new warnings.
}{ Juries tended to find fault with the compensations companies promised.
}  \question 52.	Manufacturers as mentioned in the passage tend to \ltk{}.\\
\fourch{ satisfy customers by writing long warnings on products
}{ become honest in describing the inadequacies of their products
}{ make the best use of labels to avoid legal liability
}{ feel obliged to view customers’ safety as their first concern
}  \question 53.	The case of Schutt helmet demonstrated that \ltk{}.\\
\fourch{ some injury claims were no longer supported by law
}{ helmets were not designed to prevent injuries
}{ product labels would eventually be discarded
}{ some sports games might lose popularity with athletes
}  \question 54.	The author’s attitude towards the issue seems to be \ltk{}.\\
\fourch{ biased
}{ indifferent
}{ puzzling
}{ objective
}\end{questions}    \subsection{Text 2}
In the first year or so of Web business, most of the action has revolved around efforts to tap the consumer market. More recently, as the Web proved to be more than a fashion, companies have started to buy and sell products and services with one another. Such business-to-business sales make sense because business people typically know what product they’re looking for.

Nonetheless, many companies still hesitate to use the Web because of doubts about its reliability. “Businesses need to feel they can trust the pathway between them and the supplier,” says senior analyst Blane Erwin of Forrester Research. Some companies are limiting the risk by conducting online transactions only with established business partners who are given access to the company’s private intranet.

Another major shift in the model for Internet commerce concerns the technology available for marketing. Until recently, Internet marketing activities have focused on strategies to “pull” customers into sites. In the past year, however, software companies have developed tools that allow companies to “push” information directly out to consumers, transmitting marketing messages directly to targeted customers. Most notably, the Pointcast Network uses a screen saver to deliver a continually updated stream of news and advertisements to subscribers’ computer monitors. Subscribers can customize the information they want to receive and proceed directly to a company’s Web site. Companies such as Virtual Vineyards are already starting to use similar technologies to push messages to customers about special sales, product offerings, or other events. But push technology has earned the contempt of many Web users. Online culture thinks highly of the notion that the information flowing onto the screen comes there by specific request. Once commercial promotion begins to fill the screen uninvited, the distinction between the Web and television fades. That’s a prospect that horrifies Net purists.

But it is hardly inevitable that companies on the Web will need to resort to push strategies to make money. The examples of Virtual Vineyards, Amazon.com, and other pioneers show that a Web site selling the right kind of products with the right mix of interactivity, hospitality, and security will attract online customers. And the cost of computing power continues to free fall, which is a good sign for any enterprise setting up shop in silicon. People looking back 5 or 10 years from now may well wonder why so few companies took the online plunge.
\begin{questions} \sethlcolor{cyan} \question 55.	We learn from the beginning of the passage that Web business \ltk{}.\\
\fourch{ has been striving to expand its market
}{ intended to follow a fanciful fashion
}{ tried but in vain to control the market
}{ has been booming for one year or so
} \question 56.	Speaking of the online technology available for marketing, the author implies that \ltk{}.\\
\fourch{ the technology is popular with many Web users
}{ businesses have faith in the reliability of online transactions
}{ there is a radical change in strategy
}{ it is accessible limitedly to established partners
} \question 57.	In the view of Net purists, \ltk{}.\\
\fourch{ there should be no marketing messages in online culture
}{ money making should be given priority to on the Web
}{ the Web should be able to function as the television set
}{ there should be no online commercial information without requests
} \question 58.	We learn from the last paragraph that \ltk{}.\\
\fourch{ pushing information on the Web is essential to Internet commerce
}{ interactivity, hospitality and security are important to online customers
}{ leading companies began to take the online plunge decades ago
}{ setting up shops in silicon is independent of the cost of computing power
}
\end{questions}    \subsection{Text 3}
An invisible border divides those arguing for computers in the classroom on the behalf of students’ career prospects and those arguing for computers in the classroom for broader reasons of radical educational reform. Very few writers on the subject have explored this distinction -- indeed, contradiction -- which goes to the heart of what is wrong with the campaign to put computers in the classroom.

An education that aims at getting a student a certain kind of job is a technical education, justified for reasons radically different from why education is universally required by law. It is not simply to raise everyone’s job prospects that all children are legally required to attend school into their teens. Rather, we have a certain conception of the American citizen, a character who is incomplete if he cannot competently assess how his livelihood and happiness are affected by things outside of himself. But this was not always the case; before it was legally required for all children to attend school until a certain age, it was widely accepted that some were just not equipped by nature to pursue this kind of education. With optimism characteristic of all industrialized countries, we came to accept that everyone is fit to be educated. Computer-education advocates forsake this optimistic notion for a pessimism that betrays their otherwise cheery outlook. Banking on the confusion between educational and vocational reasons for bringing computers into schools, computer-education advocates often emphasize the job prospects of graduates over their educational achievement.

There are some good arguments for a technical education given the right kind of student. Many European schools introduce the concept of professional training early on in order to make sure children are properly equipped for the professions they want to join. It is, however, presumptuous to insist that there will only be so many jobs for so many scientists, so many businessmen, so many accountants. Besides, this is unlikely to produce the needed number of every kind of professional in a country as large as ours and where the economy is spread over so many states and involves so many international corporations.

But, for a small group of students, professional training might be the way to go since well-developed skills, all other factors being equal, can be the difference between having a job and not. Of course, the basics of using any computer these days are very simple. It does not take a lifelong acquaintance to pick up various software programs. If one wanted to become a computer engineer, that is, of course, an entirely different story. Basic computer skills take -- at the very longest -- a couple of months to learn. In any case, basic computer skills are only complementary to the host of real skills that are necessary to becoming any kind of professional. It should be observed, of course, that no school, vocational or not, is helped by a confusion over its purpose.
\begin{questions} \sethlcolor{cyan}  \question 59.	The author thinks the present rush to put computers in the classroom is \ltk{}.\\
\fourch{ far-reaching
}{ dubiously oriented
}{ self-contradictory
}{ radically reformatory
} \question 60.	The belief that education is indispensable to all children \ltk{}.\\
\fourch{ is indicative of a pessimism in disguise
}{ came into being along with the arrival of computers
}{ is deeply rooted in the minds of computer-ed advocates
}{ originated from the optimistic attitude of industrialized countries
} \question 61.	It could be inferred from the passage that in the author’s country the European model of professional training is \ltk{}.\\
\fourch{ dependent upon the starting age of candidates
}{ worth trying in various social sections
}{ of little practical value
}{ attractive to every kind of professional
} \question 62.	According to the author, basic computer skills should be \ltk{}.\\
\fourch{ included as an auxiliary course in school
}{ highlighted in acquisition of professional qualifications
}{ mastered through a life-long course
}{ equally emphasized by any school, vocational or otherwise
}\end{questions}
\subsection{Text 4}
When a Scottish research team startled the world by revealing 3 months ago that it had cloned an adult sheep, President Clinton moved swiftly. Declaring that he was opposed to using this unusual animal husbandry technique to clone humans, he ordered that federal funds not be used for such an experiment -- although no one had proposed to do so -- and asked an independent panel of experts chaired by Princeton President Harold Shapiro to report back to the White House in 90 days with recommendations for a national policy on human cloning. That group -- the National Bioethics Advisory Commission (NBAC) -- has been working feverishly to put its wisdom on paper, and at a meeting on 17 May, members agreed on a near-final draft of their recommendations.
NBAC will ask that Clinton’s 90-day ban on federal funds for human cloning be extended indefinitely, and possibly that it be made law. But NBAC members are planning to word the recommendation narrowly to avoid new restrictions on research that involves the cloning of human DNA or cells -- routine in molecular biology. The panel has not yet reached agreement on a crucial question, however, whether to recommend legislation that would make it a crime for private funding to be used for human cloning.
In a draft preface to the recommendations, discussed at the 17 May meeting, Shapiro suggested that the panel had found a broad consensus that it would be “morally unacceptable to attempt to create a human child by adult nuclear cloning.” Shapiro explained during the meeting that the moral doubt stems mainly from fears about the risk to the health of the child. The panel then informally accepted several general conclusions, although some details have not been settled.
NBAC plans to call for a continued ban on federal government funding for any attempt to clone body cell nuclei to create a child. Because current federal law already forbids the use of federal funds to create embryos (the earliest stage of human offspring before birth) for research or to knowingly endanger an embryo’s life, NBAC will remain silent on embryo research. NBAC members also indicated that they will appeal to privately funded researchers and clinics not to try to clone humans by body cell nuclear transfer. But they were divided on whether to go further by calling for a federal law that would impose a complete ban on human cloning. Shapiro and most members favored an appeal for such legislation, but in a phone interview, he said this issue was still “up in the air.”
\begin{questions} \sethlcolor{cyan} \question 63.	We can learn from the first paragraph that \ltk{}.\\
\fourch{ federal funds have been used in a project to clone humans
}{ the White House responded strongly to the news of cloning
}{ NBAC was authorized to control the misuse of cloning technique
}{ the White House has got the panel’s recommendations on cloning
} \question 64.	The panel agreed on all of the following except that \ltk{}.\\
\fourch{ the ban on federal funds for human cloning should be made a law
}{ the cloning of human DNA is not to be put under more control
}{ it is criminal to use private funding for human cloning
}{ it would be against ethical values to clone a human being
} \question 65.	NBAC will leave the issue of embryo research undiscussed because \ltk{}.\\
\fourch{ embryo research is just a current development of cloning
}{ the health of the child is not the main concern of embryo research
}{ an embryo’s life will not be endangered in embryo research
}{ the issue is explicitly stated and settled in the law
} \question 66.	It can be inferred from the last paragraph that \ltk{}.\\
\fourch{ some NBAC members hesitate to ban human cloning completely
}{ a law banning human cloning is to be passed in no time
}{ privately funded researchers will respond positively to NBAC’s appeal
}{ the issue of human cloning will soon be settled
}
\end{questions}   

\subsection{Text 5}
Science, in practice, depends far less on the experiments it prepares than on the preparedness of the minds of the men who watch the experiments. Sir Isaac Newton supposedly discovered gravity through the fall of an apple. Apples had been falling in many places for centuries and thousands of people had seen them fall. But Newton for years had been curious about the cause of the orbital motion of the moon and planets. What kept them in place? Why didn’t they fall out of the sky? The fact that the apple fell down toward the earth and not up into the tree answered the question he had been asking himself about those larger fruits of the heavens, the moon and the planets.

How many men would have considered the possibility of an apple falling up into the tree? Newton did because he was not trying to predict anything. He was just wondering. His mind was ready for the unpredictable. Unpredictability is part of the essential nature of research. If you don’t have unpredictable things, you don’t have research. Scientists tend to forget this when writing their cut and dried reports for the technical journals, but history is filled with examples of it.

In talking to some scientists, particularly younger ones, you might gather the impression that they find the “scientific method” a substitute for imaginative thought. I’ve attended research conferences where a scientist has been asked what he thinks about the advisability of continuing a certain experiment. The scientist has frowned, looked at the graphs, and said “the data are still inconclusive.” “We know that,” the men from the budget office have said, “but what do you think? Is it worthwhile going on? What do you think we might expect?” The scientist has been shocked at having even been asked to speculate.

What this amounts to, of course, is that the scientist has become the victim of his own writings. He has put forward unquestioned claims so consistently that he not only believes them himself, but has convinced industrial and business management that they are true. If experiments are planned and carried out according to plan as faithfully as the reports in the science journals indicate, then it is perfectly logical for management to expect research to produce results measurable in dollars and cents. It is entirely reasonable for auditors to believe that scientists who know exactly where they are going and how they will get there should not be distracted by the necessity of keeping one eye on the cash register while the other eye is on the microscope. Nor, if regularity and conformity to a standard pattern are as desirable to the scientist as the writing of his papers would appear to reflect, is management to be blamed for discriminating against the “odd balls” among researchers in favor of more conventional thinkers who “work well with the team.”
\begin{questions} \sethlcolor{cyan}   \question 67.	The author wants to prove with the example of Isaac Newton that \ltk{}.\\
\fourch{ inquiring minds are more important than scientific experiments
}{ science advances when fruitful researches are conducted
}{ scientists seldom forget the essential nature of research
}{ unpredictability weighs less than prediction in scientific research
}    \question 68.	The author asserts that scientists \ltk{}.\\
\fourch{ shouldn’t replace “scientific method” with imaginative thought
}{ shouldn’t neglect to speculate on unpredictable things
}{ should write more concise reports for technical journals
}{ should be confident about their research findings
}    \question 69.	It seems that some young scientists \ltk{}.\\
\fourch{ have a keen interest in prediction
}{ often speculate on the future
}{ think highly of creative thinking
}{ stick to “scientific method”
}    \question 70.	The author implies that the results of scientific research \ltk{}.\\
\fourch{ may not be as profitable as they are expected
}{ can be measured in dollars and cents
}{ rely on conformity to a standard pattern
}{ are mostly underestimated by management
}\end{questions}