
 \section{2011}
 \subsection{Text 1}
 The decision of the New York Philharmonic to hire Alan
 Gilbert as its next music director has been the talk of the classical-music world ever since the sudden announcement of his appointment in 2009. For the most part, the response has been favorable, to say the least. “Hooray! At last!” wrote Anthony Tommasini, a sober-sided classical-music critic.
 One of the reasons why the appointment came as such a surprise, however, is that Gilbert is comparatively little known. Even Tommasini, who had advocated Gilbert’s appointment in theTimes, calls him “an unpretentious musician with no air of the formidable conductor about him.” As a description of the next music director of an orchestra that has hitherto been led by musicians like Gustav Mahler and Pierre Boulez, that seems likely to have struck at least some Times readers as faint praise.
 For my part, I have no idea whether Gilbert is a great conductor or even a good one. To be sure, he performs an impressive variety of interesting compositions, but it is not necessary for me to visit Avery Fisher Hall, or anywhere else, to hear interesting orchestral music. All I have to do is to go to my CD shelf, or boot up my computer and download still more recorded music from iTunes.
 Devoted concertgoers who reply that recordings are no substitute for live performance are missing the point. For the time, attention, and money of the art-loving public, classical instrumentalists must compete not only with opera houses, dance troupes, theater companies, and museums, but also with the recorded performances of the great classical musicians of the 20thcentury. These recordings are cheap, available everywhere, and very often much higher in artistic quality than today’s live performances; moreover, they can be “consumed” at a time and place of the listener’s choosing. The widespread availability of such recordings has thus brought about a crisis in the institution of the traditional classical concert.
 One possible response is for classical performers to program attractive new music that is not yet available on record. Gilbert’s own interest in new music has been widely noted: Alex Ross, a classical-music critic, has described him as a man who is capable of turning the Philharmonic into “a markedly different, more vibrant organization.” But what will be the nature of that difference? Merely expanding the orchestra’s repertoire will not be enough. If Gilbert and the Philharmonic are to succeed, they must first change the relationship between America’s oldest orchestra and the new audience it hopes toattract.
 \begin{questions} \sethlcolor{cyan}\question 21. We learn from Paragraph 1 that Gilbert’s appointment has
 \\ \fourch{ incurred criticism.
 }{ raised suspicion.
 }{ received acclaim.
 }{ aroused curiosity.
 }\question  22. Tommasini regards Gilbert as an artist who is
 \\ \fourch{ influential.
 }{ modest.
 }{ respectable.
 }{ talented.
 }\question 23.  The author believes that the devoted concertgoers
 \\ \fourch{ ignore the expenses of live performances.
 }{ reject most kinds of recorded performances.
 }{ exaggerate the variety of live performances.
 }{ overestimate the value of live performances.
 }\question 24. According to the text, which of the following is true of recordings?
 \\ \fourch{ They are often inferior to live concerts in quality.
 }{ They are easily accessible to the general public.
 }{ They help improve the quality of music.
 }{ They have only covered masterpieces.
 }\question  25. Regarding Gilbert’s role in revitalizing the Philharmonic, the author feels
 \\ \fourch{ doubtful.
 }{ enthusiastic.
 }{ confident.
 }{ puzzled.
 }\end{questions}      \subsection{Text 2}
 When Liam McGee departed as president of Bank of America in August, his explanation was surprisingly straight up. Rather than cloaking his exit in the usual vague excuses, he came right out and said he was leaving “to pursue my goal of running a company.” Broadcasting his ambition was “very much my decision,” McGee says. Within two weeks, he was talking for the first time with the board of Hartford Financial Services Group, which named him CEO and chairman on September 29.
 McGee says leaving without a position lined up gave him time to reflect on what kind of company he wanted to run. It also sent a clear message to the outside world about his aspirations. And McGee isn’t alone. In recent weeks the No. 2 executives at Avon and American Express quit with the explanation that they were looking for a CEO post. As boards scrutinize succession plans in response to shareholder pressure, executives who don’t get the nod also may wish to move on. A turbulent business environment also has senior managers cautious of letting vague pronouncements cloud their reputations.
 As the first signs of recovery begin to take hold, deputy chiefs may be more willing to make the jump without a net. In the third quarter, CEO turnover was down 23\% from a year ago as nervous boards stuck with the leaders they had, according to Liberum Research. As the economy picks up, opportunities will abound for aspiring leaders.
 The decision to quit a senior position to look for a better one is unconventional. For years executives and headhunters have adhered to the rule that the most attractive CEO candidates are the ones who must be poached. Says Korn/Ferry senior partner Dennis Carey: “I can’t think of a single search I’ve done where a board has not instructed me to look at sitting CEOs first.”
 Those who jumped without a job haven’t always landed in top positions quickly. Ellen Marram quit as chief of Tropicana a decade ago, saying she wanted to be a CEO. It was a year before she became head of a tiny Internet-based commodities exchange. Robert Willumstad left Citigroup in 2005 with ambitions to be a CEO. He finally took that post at a major financial institution three years later.
 Many recruiters say the old disgrace is fading for top
 performers. The financial crisis has made it more acceptable to be between jobs or to leave a bad one. “The traditional rule was it’s safer to stay where you are, but that’s been fundamentally inverted,” says one headhunter. “The people who’ve been hurt the worst are those who’ve stayed too long.”
 \begin{questions} \sethlcolor{cyan}\question 26. When McGee announced his departure, his manner can best be described as being
 \\ \fourch{ arrogant.
 }{ frank.
 }{ self-centered.
 }{ impulsive.
 }\question 27. According to Paragraph 2, senior executives’ quitting may be spurred by
 \\ \fourch{ their expectation of better financial status.
 }{ their need to reflect on their private life.
 }{ their strained relations with the boards.
 }{ their pursuit of new career goals.
 }\question 28. The word “poached” (Line 3, Paragraph 4) most probably means
 \\ \fourch{ approved of.
 }{ attended to.
 }{ hunted for.
 }{ guarded against.
 }\question 29. It can be inferred from the last paragraph that
 \\ \fourch{ top performers used to cling to their posts.
 }{ loyalty of top performers is getting out-dated.
 }{ top performers care more about reputations.
 }{ it’s safer to stick to the traditional rules.
 }\question 30. Which of the following is the best title for the text?
 \\ \fourch{ CEOs: Where to Go?
 }{ CEOs: All the Way Up?
 }{ Top Managers Jump without a Net
 }{ The Only Way Out for Top Performers
 }\end{questions}      \subsection{Text 3}
 The rough guide to marketing success used to be that you got what you paid for. No longer. While traditional “paid” media—such as television commercials and print advertisements – still play a major role, companies today can exploit many alternative forms of media. Consumers passionate about a product may create “earned” media by willingly promoting it to friends, and a company may leverage “owned” media by sending e-mail alerts about products and sales to customers registered with its Web site. The way consumers now approach the process of making purchase decisions means that marketing’s impact stems from a broad range of factors beyond conventional paid media.
 Paid and owned media are controlled by marketers promoting their own products. For earned media, such marketers act as the initiator for users’ responses. But in some cases, one marketer’s owned media become another marketer’s paid media – for instance, when an e-commerce retailer sells ad space on its Web site. We define such sold media as owned media whose traffic is so strong that other organizations place their content or e-commerce engines within that environment. This trend, which we believe is still in its infancy, effectively began with retailers and travel providers such as airlines and hotels and will no doubt go further. Johnson \& Johnson, for example, has created BabyCenter, a stand-alone media property that promotes complementary and even competitive products. Besides generating income, the presence of other marketers makes the site seem objective, gives companies opportunities to learn valuable information about the appeal of other companies’ marketing, and may help expand user traffic for all companies concerned.
 The same dramatic technological changes that have provided marketers with more (and more diverse) communications choices have also increased the risk that passionate consumers will voice their opinions in quicker, more visible, and much more damaging ways.Such hijacked media are the opposite of earned media: an asset or campaign becomes hostage to consumers, other stakeholders, or activists who make negative allegations about a brand or product. Members of social networks, for instance, are learning that they can hijack media to apply pressure on the businesses that originally created them.
 If that happens, passionate consumers would try to persuade others to boycott products, putting the reputation of the target company at risk. In such a case, the company’s response may not be sufficiently quick or thoughtful, and the learning curve has been steep. Toyota Motor, for example, alleviated some of the damage from its recall crisis earlier this year with a relatively quick and well-orchestrated social-media response campaign, which included efforts to engage with consumers directly on sites such as Twitter and the social-news site Digg.
 \begin{questions} \sethlcolor{cyan}\question 31. Consumers may create “earned” media when they are
 \\ \fourch{ obsessed with online shopping at certain Web sites.
 }{ inspired by product-promoting e-mails sent to them.
 }{ eager to help their friends promote quality products.
 }{ enthusiastic about recommending their favorite products.
 }\question 32. According to Paragraph 2, sold media feature
 \\ \fourch{ a safe business environment.
 }{ random competition.
 }{ strong user traffic.
 }{ flexibility in organization.
 }\question   33. The author indicates in Paragraph 3 that earned media
 \\ \fourch{ invite constant conflicts with passionate consumers.
 }{ can be used to produce negative effects in marketing.
 }{ may be responsible for fiercer competition.
 }{ deserve all the negative comments about them.
 }\question 34. Toyota Motor’s experience is cited as an example of
 \\ \fourch{ responding effectively to hijacked media.
 }{ persuading customers into boycotting products.
 }{ cooperating with supportive consumers.
 }{ taking advantage of hijacked media.
 }\question 35. Which of the following is the text mainly about?
 \\ \fourch{ Alternatives to conventional paid media.
 }{ Conflict between hijacked and earned media.
 }{ Dominance of hijacked media.
 }{ Popularity of owned media.
 }\end{questions}      \subsection{Text 4}
 It’s no surprise that Jennifer Senior’s insightful, provocative magazine cover story, “I Love My Children, I Hate My Life,” is arousing much chatter – nothing gets people talking like the suggestion that child rearing is anything less than a completely fulfilling, life-enriching experience. Rather than concluding that children make parents either happy or miserable, Senior suggests we need to redefine happiness: instead of thinking of it as something that can be measured by moment-to-moment joy, we should consider being happy as a past-tense condition. Even though the day-to-day experience of raising kids can be soul-crushingly hard, Senior writes that “the very things that in the moment dampen our moods can later be sources of intense gratification and delight.”
 The magazine cover showing an attractive mother holding a cute baby is hardly the only Madonna-and-child image on newsstands this week. There are also stories about newly adoptive – and newly single – mom Sandra Bullock, as well as the usual “Jennifer Aniston is pregnant” news. Practically every week features at least one celebrity mom, or mom-to-be, smiling on the newsstands.
 In a society that so persistently celebrates procreation, is it any wonder that admitting you regret having children is equivalent to admitting you support kitten-killing? It doesn’t seem quite fair, then, to compare the regrets of parents to the regrets of the childless. Unhappy parents rarely are provoked to wonder if they shouldn’t have had kids, but unhappy childless folks are bothered with the message that children are the single most important thing in the world:obviously their misery must be a direct result of the gaping baby-size holes in their lives.
 Of course, the image of parenthood that celebrity magazines like Us Weekly and People present is hugely unrealistic, especially when the parents are single mothers like Bullock. According to several studies concluding that parents are less happy than childless couples, single parents are the least happy of all. No shock there, considering how much work it is to raise a kid without a partner to lean on; yet to hear Sandra and Britney tell it, raising a kid on their “own” (read: with round-the-clock help) is a piece of cake.
 It’s hard to imagine that many people are dumb enough to want children just because Reese and Angelina make it look so glamorous: most adults understand that a baby is not a haircut. But it’s interesting to wonder if the images we see every week of stress-free, happiness-enhancing parenthood aren’t in some small, subconscious way contributing to our own dissatisfactions with the actual experience, in the same way that a small part of us hoped getting “the Rachel” might make us look just a little bit like Jennifer Aniston.
 \begin{questions} \sethlcolor{cyan}\question 36. Jennifer Senior suggests in her article that raising a childcan bring
 \\ \fourch{ temporary delight.
 }{ enjoyment in progress.
 }{ happiness in retrospect.
 }{ lasting reward.
 }\question 37.We learn from Paragraph 2 that
 \\ \fourch{ celebrity moms are a permanent source for gossip.
 }{ single mothers with babies deserve greater attention.
 }{ news about pregnant celebrities is entertaining.
 }{ having children is highly valued by the public.
 }\question 38. It is suggested in Paragraph 3 that聽childless folks
 \\ \fourch{ are constantly exposed to criticism.
 }{ are largely ignored by the media.
 }{ fail to fulfill their social responsibilities.
 }{ are less likely to be satisfied with their life.
 }\question 39.According to Paragraph 4, the message conveyed by celebrity magazines is
 \\ \fourch{ soothing.
 }{ ambiguous.
 }{ compensatory.
 }{ misleading.
 }\question 40.Which of the following can be inferred from the last paragraph?
 \\ \fourch{ Having children contributes little to the glamour of celebrity moms.
 }{ Celebrity moms have influenced our attitude towards child rearing.
 }{ Having children intensifies our dissatisfaction with life.
 }{ We sometimes neglect the happiness from child rearing.
 }\end{questions}    
 