\section{2003年全真试题}
\subsection{Text 1 Syies Like Us}
Wild Bill Donovan \ul{would have loved} the Internet. The American spymaster who built the Office of Strategic Services in the World War II and later laid the roots for the CIA was fascinated with information. 
Donovan believed in using whatever tools came to hand in the “great game” of 
\ann{espionage:间谍活动}\ghl{espionage} -- spying as a “profession.” These days 
\ding{193}\sethlcolor{yellow}\hl{the Net}, which has already re-made such everyday 
\ann{pasttimes:消遣活动}\ghl{pastimes} as buying books and sending mail, is 
\ding{192}\sethlcolor{yellow}\hl{reshaping Donovan}’s 
\ann{vocation:职业}vocation as well.

The latest revolution isn’t simply a matter of gentlemen reading other gentlemen’s e-mail. That kind of electronic spying has been going on for decades. In the past three or four years, the World Wide Web has given birth to a whole industry of point-and-click spying. The spooks call it “open source intelligence,” and as the Net grows, it is becoming increasingly influential. In 1995 the CIA held a contest to see who could compile the most data about Burundi. The winner, by a large margin, was a tiny Virginia company called Open-Source Solutions, whose 
\ding{194}\sethlcolor{yellow}\hl{clear advantage was its mastery of the electronic world}.

Among the firms \sethlcolor{yellow}\hl{making the biggest} 
\ann{splash:溅泼,溅泼声,溅湿.make a splash:惹人注目,引起轰动}\ghl{splash} 
in the new world is Straitford, Inc., a private 
\ann{intelligence-analysis firm:情报分析公司}intelligence-analysis firm based in Austin, Texas. Straitford makes money by selling the results of spying (covering nations from Chile to Russia) to corporations like energy-services firm McDermott International. Many of its predictions are available online at www.straitford.com.

Straitford president George Friedman says he sees the online world as a kind of 
\ann{mutually:相互地,双方地}mutually reinforcing tool for both information collection and distribution, a spymaster’s dream. Last week his firm was busy vacuuming up data bits from the far corners of the world and predicting a crisis in Ukraine. “As soon as that report runs, we’ll suddenly get 500 new internet sign-ups from Ukraine,” says Friedman, a former political science professor. “And we’ll hear back from some of them.” Open-source spying does have its risks, of course, since it can be difficult to tell good information from bad. That’s where Straitford earns its keep.

Friedman relies on a \ann{lean:节俭的,精简的}lean staff in Austin. Several of his staff members have military-intelligence backgrounds. He sees the firm’s outsider status as the key to its success. 
\ding{196}\sethlcolor{yellow}\hl{Straitford}’\hl{s briefs don}’\hl{t sound like the usual Washington} \annmark{美国政府} \ann{back-and-forthing:来回,往返。文中喻指言辞闪烁}back-and-forthing, whereby agencies avoid dramatic declarations on the chance they might be wrong. 
\ding{196}\sethlcolor{yellow}\hl{Straitford, says Friedman, takes pride in its independent voice}.

\begin{questions} \sethlcolor{cyan}
\question 41.	\hl{The emergence of the Net} has \ltk{}.\\
\fourch{  received support from fans like Donovan
}{\textcolor{red}{remolded the intelligence services} 
\mmark{remold改变。句义:改变了情报收集工作.remold=reshape.}
}{restored many common pastimes \mmark{restore:恢复}
}{ revived spying as a profession \mmark{revive:复兴}
}

\question 42.	Donovan’s story is mentioned in the text to  \ltk{}.\\
\fourch{ \textcolor{blue}{introduce the topic of online spying}
}{ show how he fought for the U.S.
}{ give an episode of the information war
}{ honor his unique services to the CIA
}

\question 43.	The phrase “making the biggest splash” (line 1, paragraph 3) most probably means \ltk{}.\\
\fourch{ causing the biggest trouble
}{ exerting the greatest effort
}{ \textcolor{blue}{achieving the greatest success}
}{ enjoying the widest popularity
}
\question 44.	It can be learned from paragraph 4 that \ltk{}.\\
\fourch{ Straitford’s prediction about Ukraine has proved true
}{ Straitford guarantees the truthfulness of its information
}{ Straitford’s business is characterized by unpredictability
}{\textcolor{blue}{is able to provide fairly reliable information}
}
\begin{solution}
倾向选“具有弹性,留有余地“的选项:some,many,fairly(相当的),rather;
慎选那些太绝对的选项:completely,absolutely,guarantee,certainly
\end{solution}

\question 45.	\hl{Straitford} is \hl{most proud of its} \ltk{}.\\
\fourch{ official status
}{ \textcolor{blue}{nonconformist image} \mmark{noncomformist image:不随大流的形象}
}{ efficient staff
}{ military background
}\end{questions}    

\subsection{Text 2 Animal Rights:Reaching the Public}
To 
\ann{paraphrase: n.意译,释译;v.改述,解释 这里指曾经说过}\ghl{paraphrase} 18th-century statesman Edmund Burke, “all that is needed for the 
\ann{triumph:胜利,成功}\ghl{triumph} of a misguided cause is that good people do nothing.” \ding{192}\sethlcolor{yellow}\hl{One such cause} now seeks to end biomedical research because of the theory that animals have rights 
\ann{rule out:阻止}ruling out their use in research. \ding{192}\sethlcolor{yellow}\hl{Scientists need to respond forcefully to animal rights advocates}, whose arguments are confusing the public and thereby threatening advances in health knowledge and care. Leaders of the animal rights movement target biomedical research because it depends on public funding, and 
\ding{194}\sethlcolor{yellow}\hl{few people understand the process of health care research}. Hearing 
\ann{allegation:说法,指控}\ghl{allegations} of \ding{193}\sethlcolor{yellow}\hl{cruelty to animals} in research settings, many are 
\ann{perplexed:不明白}\ghl{perplexed} that anyone would \ding{193}\sethlcolor{yellow}\hl{deliberately harm an animal}.

For example, a grandmotherly woman 
\ann{staff:为配备职员,在..工作}staffing an animal rights 
\ann{booth:摊位;(有特殊用途的)小房间}\ghl{booth} at a recent street fair was distributing a brochure that encouraged readers not to use anything that comes from or is tested in animals—no meat, no fur, no medicines. Asked if she opposed immunizations, she wanted to know if vaccines come from animal research. When assured that they do, she replied, “Then I would have to say yes.” Asked what will happen when 
\ann{epidemic:传染病}\ghl{epidemics} return, she said, “Don’t worry, scientists will find some way of using computers.” 
\ding{194}\sethlcolor{yellow}\hl{Such well-meaning people just don}’\hl{t understand}.

\ding{195}\sethlcolor{yellow}\hl{Scientists must communicate their message to the public} in a 
\ann{compassionate:表示怜悯的,有同情心的}\ghl{compassionate}, understandable way -- in human terms, not in the language of 
\ann{molecular:分子}\ghl{molecular} biology. We need to make clear the connection between animal research and a grandmother’s hip replacement, a father’s 
\ann{bypass:旁路,旁道;(给心脏接旁桶管的)分流术,搭桥术}bypass operation, a baby’s 
\ann{vaccination:接种疫苗}vaccinations, and even a pet’s shots. To those who are unaware that animal research was needed to produce these treatments, as well as new treatments and vaccines, animal research seems
 \ann{...at best...at worst:说得好听点,说得难听点}\ding{193}\sethlcolor{yellow}\hl{wasteful at best and crue at worst}.

 \ding{195}\sethlcolor{yellow}\hl{Much can be done. Scientists couldMuch can be done. Scientists could} “adopt” middle school classes and present their own research. They should be quick to respond to letters to the editor, lest animal rights misinformation go unchallenged and acquire a
\ann{deceptive:骗人的}\ghl{deceptive} appearance of truth. Research institutions could be opened to tours, to show that laboratory animals receive humane care. Finally, because the ultimate 
\ann{stakeholder:有发言权的人}\ghl{stakeholders} are patients, the health research community should actively recruit to its cause not only 
\ding{196}\sethlcolor{yellow}\hl{well-known personalities} such as \ding{196}\sethlcolor{yellow}\hl{Stephen Cooper, who has made courageous statements about the value of animal research}, but all who receive medical treatment. 
\ding{192}\sethlcolor{yellow}\hl{If good people do nothing there is a real possibility that an} 
\ann{uninformed citizenry:不明真相的民众}\hl{uninformed citizenry will} \ann{extinguish the precious embers of...:熄灭...的宝贵火种}\sethlcolor{yellow}\hl{extinguish the precious} \ann{ember:火种}\ghl{embers} \sethlcolor{yellow}\hl{of medical progress}.
\begin{questions} \sethlcolor{cyan}
    \question 46.	The author begins his article \hl{with Edmund Burke}’\hl{s words to} \ltk{}.\\
\fourch{ \textcolor{blue}{call on scientists to take some actions}
}{ criticize the misguided cause of animal rights
}{ warn of the doom of biomedical research
}{ show the triumph of the animal rights movement
}
\begin{solution}
    all that is needed for the triumph of a misguided cause is that good people do nothing.
    被误导事业的得逞源自好人的不作为
\end{solution}
\question 47.	\hl{Misled people tend to think} that \hl{using an animal in research} is \ltk{}.\\
\fourch{ cruel but natural
}{ \textcolor{red}{inhuman and unacceptable} 
}{ inevitable but vicious \mmark{inevitable:不可避免;vicious:不道德的}
}{ pointless and wasteful
}
\begin{solution}
    wasteful at best and crue at worst:说得好听点是浪费,说得不好听是残忍
\end{solution}
\question 48.	\hl{The example of the grandmotherly woman} is used to show \hl{the public}’s \ltk{}.\\
\fourch{ discontent with animal research \mmark{discontent:不满足,不满意} 
}{ \textcolor{red}{ignorance about medical science}  \mmark{ignorance => do not understand}
}{ indifference to epidemics
}{ anxiety about animal rights
}
\begin{solution}
    干扰项往往“对应事例细节信息,却无法涵盖事例整体信息,且无法与上下午衔接,甚至偏离上下文论述对象”;正确选项需既能体现事例完整信息,又能和上下文顺滑衔接,成为一个整体
\end{solution}
\question 49.	\hl{The author believes} that, in face of the challenge from \hl{animal rights advocates, scientists should} \ltk{}.\\
\fourch{ \textcolor{blue}{communicate more with the public}
}{ employ hi-tech means in research \mmark{“先进医疗手段与动物研究息息相关,(因此要捍卫动物研究)”改为“科学家应采取高科技手段(以捍卫动物研究)”}
}{ feel no shame for their cause
}{ strive to develop new cures \mmark{strive to:努力}
}\question 50.	From the text we learn that \hl{Stephen Cooper is} \ltk{}.\\
\fourch{ a well-known humanist \mmark{personality:【因常出现在报纸电视上而知名的】名人 "well-known personalities"偷换成“a well-known humanist”}
}{ a medical practitioner
}{ an enthusiast in animal rights
}{ \textcolor{blue}{a supporter of animal research}
}\end{questions}    

\subsection{Text 3 铁路公司合并可能造成垄断}
In recent years, railroads have been combining with each other, merging into super systems, causing 
\ann{heighten:加强,提高,增加}\ghl{heightened} concerns about monopoly. As recently as 1995, the top four railroads accounted for under 70 percent of the total ton-miles moved by rails. Next year, after a series of mergers is completed, just four railroads will control well over 90 percent of all the 
\ann{frieight:货运}freight moved by major rail carriers.

\ding{192}\sethlcolor{yellow}\hl{Supporters of the new super systems argue that these mergers} will allow for 
\ann{substantial:大量的,可观的}substantial cost reductions and better 
\ann{coordinate:(使)协调}coordinated service. 
\ding{192}\sethlcolor{yellow}\hl{Any threat of monopoly, they argue, is removed by fierce competition from trucks}. But many \ding{193}\sethlcolor{yellow}\hl{shippers complain} that for heavy 
\ann{bulk:(货物购买、运输等)大批的,大量的}\ghl{bulk} commodities traveling long distances, such as coal, chemicals, and grain, trucking is too costly and the railroads therefore 
\ann{have them by the throat:扼住其咽喉,牵制控制}have them by the throat.

The vast 
\ann{consolidation:合并,联合}consolidation within the rail industry means that most shippers are served by only one rail company. Railroads typically charge such \ann{captive:【仅用于名词前】人身自由受限制的,受控制的,受垄断的}“captive” shippers 20 to 30 percent more than they do when another railroad is competing for the business. 
\ding{194}\sethlcolor{yellow}\hl{Shippers who feel they are being overcharged} have the right to appeal to the federal government’s Surface Transportation Board for rate relief, \ding{194}\sethlcolor{yellow}\hl{but the process is expensive, time consuming, and will work only in truly extreme cases}.

Railroads \ann{justify:提供合理的理由,辩解,证明正当}justify rate discrimination against captive shippers on the grounds that in the long run it reduces everyone’s cost. 
If railroads charged all customers the same average rate, they argue, shippers who have the option of switching to trucks or other forms of transportation would do so, 
leaving remaining customers to shoulder the cost of keeping up the line. It’s a theory to which many economists 
\ann{subscribe to sth:同意赞成}subscribe, but in practice it often leaves railroads in the position of \ding{195}\sethlcolor{yellow}\hl{determining which companies will }
\ann{flourish:繁荣}\hl{flourish and which will fail}. “Do we really want railroads to be the 
\ann{arbiter:仲裁人;权威人士}\ghl{arbiters} of \ding{195}\sethlcolor{yellow}\hl{who wins and who loses in the marketplace}?” asks Martin Bercovici, a Washington lawyer who frequently represents shipper.

Many \ding{193}\hl{captive shippers also worry} they will soon be hit with a round of huge rate increases. The railroad industry as a whole, 
\ding{196}\hl{despite its brightening fortunes, still does not earn enough to cover the cost of the capital it must invest to keep up with its} 
\ann{surge:(需求价格利用等)飞涨}\ghl{surging} 
\sethlcolor{yellow}\hl{traffic. Yet railroads continue to borrow billions to acquire one another, with Wall Street cheering them on}. Consider the \$10.2 billion bid by Norfolk Southern and CSX to acquire Conrail this year. Conrail’s net railway operating income in 1996 was just \$427 million, less than half of the carrying costs of the transaction. Who’s going to pay for the rest of the bill? Many \ding{193}\sethlcolor{yellow}\hl{captive shippers fear} that they will, as Norfolk Southern and CSX increase their grip on the market.

\begin{questions} \sethlcolor{cyan}
\question 52.	According to \hl{those who support mergers, railway monopoly is unlikely} because \ltk{}.\\
\fourch{ cost reduction is based on competition
}{ services call for cross-trade coordination
}{ \textcolor{blue}{outside competitors will continue to exist}
}{ shippers will have the railway by the throat
}
\begin{solution}
    根据显性观点词确定大方向:argue,believe,hold,think,support
\end{solution}  

\question 52.	What is many \hl{captive shippers}’ \hl{attitude} towards \hl{the consolidation in the rail industry}?\\
\fourch{ Indifferent.
}{ Supportive.
}{ Indignant. \mmark{indignant:愤怒。 大方向正确,但选项将内心不满、担忧夸大为公开的愤怒}
}{ \textcolor{red}{Apprehensive}. \mmark{apprehensive:忧虑} 
}  
\question 53.	It can be \hl{inferred} from \hl{paragraph 3} that \ltk{}.\\
\fourch{ shippers will be charged less without a rival railroad
}{ there will soon be only one railroad company nationwide
}{ \textcolor{blue}{overcharged shippers are unlikely to appeal for rate relief}
}{ a government board ensures fair play in railway business
}  
\question 54.	The word “\hl{arbiters}” (line 7, paragraph 4) most probably refers to those \ltk{}.\\
\fourch{ who work as coordinators
}{ \textcolor{red}{who function as judges} 
}{ who supervise transactions
}{ who determine the price
} 

\question 55.	According to the text, the cost increase in the rail industry is mainly caused by \ltk{}.\\
\fourch{ \textcolor{blue}{the continuing acquisition(收购)} 
}{ the growing traffic
}{ the cheering Wall Street
}{ the shrinking market
}
\end{questions}

\subsection{Text 4 The Best Health Care Only So Far}
It is said that in England death is 
\ann{pressing:难以退却的,不容忽视的}\ghl{pressing}, in Canada 
\ann{inevitable:不可避免的}\ghl{inevitable} and in California \ding{192}\sethlcolor{yellow}\hl{optional}. 
\ann{small wonder:不足为奇}Small wonder. Americans’ life expectancy has nearly doubled over the past century. \ann{Failing hips:髋骨出毛病}Failing hips can be replaced, clinical depression controlled, \ann{cataracts:白内障}\ghl{cataracts} removed in a 30-minutes surgical procedure. 
\ding{192}\sethlcolor{yellow}\hl{Such advances offer the aging population a quality of life that was unimaginable when I entered medicine 50 years ago}. 
\ding{192}\ding{196}\sethlcolor{yellow}\hl{But not even a great health-care system can cure death} -- \hl{and our failure to confront that reality now threatens this greatness of ours}.

\ding{196}\sethlcolor{yellow}\hl{Death is normal}; we are genetically programmed to \ann{disintegrate:瓦解}disintegrate and \ann{perish:死亡}\ghl{perish}, even under ideal conditions. We all understand that at some level, yet as medical consumers we treat death as a problem to be solved. 
\ann{句意:由于受第三方付款人的保护免交医疗护理费用}Shielded by third-party payers from the cost of our care, we demand everything that can possibly be done for us, 
\ding{193}\sethlcolor{yellow}\hl{even if it}’\hl{s useless}. The most obvious \ding{193}\hl{example is late-stage cancer care}. Physicians -- frustrated by their inability to cure the disease and fearing loss of hope in the patient -- too often offer aggressive treatment far beyond what is scientifically justified.

In 1950, the U.S. spent \$12.7 billion on health care. In 2002, the cost will be \$1,540 billion. Anyone can see this trend is unsustainable. Yet few seem willing to try to reverse it. Some scholars conclude that a government with finite resources should simply stop paying for medical care that sustains life beyond a certain age -- say 83 or so. Former Colorado governor Richard Lamm has been quoted as saying that 
\ann{the old and infirm:年老体弱者}the old and infirm “have a duty to die and get out of the way,” so that younger, healthier people can realize their potential.
\ding{194}\sethlcolor{yellow}\hl{I would not go that far}. Energetic people now routinely work through their 60s and beyond, and remain 
\ann{dazzlingly:惊人地,眩目地}\ghl{dazzlingly} productive. At 78, Viacom chairman Sumner Redstone jokingly claims to be  53. Supreme Court Justice Sandra Day O’Connor is in her 70s, and former surgeon general C. Everett Koop chairs an Internet start-up in his 80s. These leaders are living proof that prevention works and that we can manage \ding{196}\sethlcolor{yellow}\hl{the health problems that come naturally with age}. As a mere 68-year-old, I wish to age as productively as they have.

\ding{194}\sethlcolor{yellow}\hl{Yet there are limits to what a society can spend in this pursuit}. Ask a physician, I know the most costly and dramatic measures may be ineffective and painful. I also know that people in 
\ding{195}\sethlcolor{yellow}\hl{Japan and Sweden, countries} that \hl{spend far less on medical care}, \hl{have achieved longer, healthier lives} than we have. As a nation, we may be 
\ding{195}\sethlcolor{yellow}\hl{overfunding the quest for unlikely cures while underfunding research on}
\ann{overfunding:投入多.quest:寻求} 
\ann{humble:简单而实用的,平常的}\ghl{humbler} therapies that could improve people’s lives.
\begin{questions} \sethlcolor{cyan} \question 56.	What is \hl{implied} in \hl{the first sentence}? \\
\fourch{ Americans are better prepared for death than other people.
}{ Americans enjoy a higher life quality \ul{than ever before}. \mmark{错将50 years ago 延伸为 than ever before}
}{ \textcolor{red}{Americans are over-confident of their medical technology}.
}{ Americans take a vain pride in their long life expectancy. \mmark{错将盲目骄傲的对象“such advances(医疗技术)“偷换成”life expectancy“.}
} \question 57.	The author uses the example of \hl{cancer patients} to show that \ltk{}.\\
\fourch{ \textcolor{blue}{medical resources are often wasted} \mmark{useless=wasted}
}{ doctors are helpless against fatal diseases
}{ some treatments are too aggressive \mmark{B和C均犯了”将事实细节信息等同于写作目的的错误“}
}{ medical costs are becoming unaffordable
} \question 58.	The author’s attitude toward \hl{Richard Lamm}’s remark is one of \ltk{}.\\
\fourch{ strong disapproval
}{ \textcolor{blue}{reserved consent} \mmark{consent:赞同,同意准许}
}{ slight contempt \mmark{contempt轻视,轻蔑}
}{ enthusiastic support
} \question 59.	In contrast to \hl{the U.S., Japan and Sweden are funding their medical care} \ltk{}.\\
\fourch{ more flexibly
}{ more extravagantly \mmark{extravagant:奢侈}
}{ more cautiously \mmark{一叶障目,只看到更少,没看到其后的效果。不足以概况”投资少“但”成效卓著“的特征}
}{ \textcolor{blue}{more reasonably}
} \question 60.	The text intends to \hl{express the idea} that \ltk{}.\\
\fourch{ medicine will further prolong people’s lives
}{ life beyond a certain limit is not worth living
}{ \textcolor{red}{death should be accepted as a fact of life}
}{ excessive demands increase the cost of health care \mmark{excessive demands :过分的要求}
}\end{questions}


