\section{2012}
\subsection{Text 1}
Come on—Everybody’s doing it. That whispered message, half invitation and half forcing, is what most of us think of when we hear the words peer pressure. It usually leads to no good—drinking, drugs and casual sex. But in her new book Join the Club, Tina Rosenberg contends that peer pressure can also be a positive force through what she calls the social cure, in which organizations and officials use the power of group dynamics to help individuals improve their lives and possibly the world.
Rosenberg, the recipient of a Pulitzer Prize, offers a host of examples of the social cure in action: In South Carolina, a state-sponsored antismoking program called Rage Against the Haze sets out to make cigarettes uncool. In South Africa, an HIV-prevention initiative known as LoveLife recruits young people to promote safe sex among their peers.
The idea seems promising, and Rosenberg is a perceptive observer. Her critique of the lameness of many pubic-health campaigns is spot-on: they fail to mobilize peer pressure for healthy habits, and they demonstrate a seriously flawed understanding of psychology. “Dare to be different, please don’t smoke!”pleads one billboard campaign aimed at reducing smoking among teenagers-teenagers, who desire nothing more than fitting in. Rosenberg argues convincingly that public-health advocates ought to take a page from advertisers, so skilled at applying peer pressure. 
But on the general effectiveness of the social cure, Rosenberg is less persuasive. Join the Club is filled with too much irrelevant detail and not enough exploration of the social and biological factors that make peer pressure so powerful. The most glaring flaw of the social cure as it’s presented here is that it doesn’t work very well for very long. Rage Against the Haze failed once state funding was cut. Evidence that the LoveLife program produces lasting changes is limited and mixed.
There’s no doubt that our peer groups exert enormous influence on our behavior. An emerging body of research shows that positive health habits—as well as negative ones—spread through networks of friends via social communication. This is a subtle form of peer pressure: we  unconsciously imitate the behavior we see every day.
Far less certain, however, is how successfully experts and bureaucrats can select our peer  groups and steer their activities in virtuous directions. It’s like the teacher who breaks up the  troublemakers in the back row by pairing them with better-behaved classmates. The tactic never  really works. And that’s the problem with a social cure engineered from the outside: in the real  world, as in school, we insist on choosing our own friends.

\begin{questions} \sethlcolor{cyan}\question 21. According to the first paragraph, peer pressure often emerges as       .
   \\ \fourch{ a supplement to the social cure             }{ a stimulus to group dynamics
   }{ an obstacle to social progress             	}{ a cause of undesirable behaviors
}\question 22. Rosenberg holds that public-health advocates should        .
   \\ \fourch{ recruit professional advertisers             }{ learn from advertisers’ experience
   }{ stay away from commercial advertisers     }{ recognize the limitations of advertisements
}\question 23.  In the author’ s view, Rosenberg’ s book fails to         .
   \\ \fourch{ adequately probe social and biological factors
   }{ effectively evade the flaws of the social cure
   }{ illustrate the functions of state funding
   }{ produce a long-lasting social effect
}\question 24. Paragraph 5 shows that our imitation of behaviors         .
   \\ \fourch{ is harmful to our networks of friends    }{ will mislead behavioral studies
   }{ occurs without our realizing it         }{ can produce negative health habits
}\question 25. The author suggests in the last paragraph that the effect of peer pressure is         .
   \\ \fourch{ harmful                   }{ desirable   
   }{ profound                  }{ questionable
}\end{questions}      \subsection{Text 2}
    A deal is a deal—except, apparently, when Entergy is involved. The company, a major energy supplier in New England, provoked justified outrage in Vermont last week when it announced it was reneging on a longstanding commitment to abide by the state’s strict nuclear regulations.
Instead, the company has done precisely what it had long promised it would not: challenge the constitutionality of Vermont’s rules in the federal court, as part of a desperate effort to keep its Vermont Yankee nuclear power plant running. It’s a stunning move.
The conflict has been surfacing since 2002, when the corporation bought Vermont’s only nuclear power plant, an aging reactor in Vernon. As a condition of receiving state approval for the sale, the company agreed to seek permission from state regulators to operate past 2012. In 2006, the state went a step further, requiring that any extension of the plant’s license be subject to Vermont legislature’s approval. Then, too, the company went along.
Either Entergy never really intended to live by those commitments, or it simply didn’t foresee what would happen next. A string of accidents, including the partial collapse of a cooling tower in 2007 and the discovery of an underground pipe system leakage, raised serious questions about both Vermont Yankee’s safety and Entergy’s management—especially after the company made misleading  statements about the pipe. Enraged by Entergy’s behavior, the Vermont Senate voted 26 to 4 last year against allowing an extension.
    Now the company is suddenly claiming that the 2002 agreement is invalid because of the 2006 legislation, and that only the federal government has regulatory power over nuclear issues. The legal issues in the case are obscure: whereas the Supreme Court has ruled that states do have some regulatory authority over nuclear power, legal scholars say the Vermont case will offer a precedent-setting test of how far those powers extend. Certainly, there are valid concerns about the patchwork regulations that could result if every state sets its own rules. But had Entergy kept its word, that debate would be beside the point.
    The company seems to have concluded that its reputation in Vermont is already so damaged that it has nothing left to lose by going to war with the state. But there should be consequences. Permission to run a nuclear plant is a public trust. Entergy runs 11 other reactors in the United States, including Pilgrim Nuclear station in Plymouth. Pledging to run Pilgrim safely, the company has applied for federal permission to keep it open for another 20 years. But as the Nuclear Regulatory Commission (NRC) reviews the company’s application, it should keep in mind what promises from Entergy are worth.

\begin{questions} \sethlcolor{cyan}\question 26. The phrase “reneging on” (Line 3, Paragraph 1) is closest in meaning to        .   
   \\ \fourch{ condemning        }{ reaffirming     }{ dishonoring    }{ securing
}\question 27. By entering into the 2002 agreement, Entergy intended to        .
   \\ \fourch{ obtain protection from Vermont regulators.
   }{ seek favor from the federal legislature.
   }{ acquire an extension of its business license.
   }{ get permission to purchase a power plant.
}\question 28. According to Paragraph 4, Entergy seems to have problems with its        .
   \\ \fourch{ managerial practices                      }{ technical innovativeness
   }{ financial goals                           }{ business vision
}\question 29. In the author’s view, the Vermont case will test        .
   \\ \fourch{ Entergy’s capacity to fulfill all its promises
   }{ the nature of states’ patchwork regulations
   }{ the federal authority over nuclear issues
   }{ the limits of states’ power over nuclear issues
}\question 30. It can be inferred from the last paragraph that         .
   \\ \fourch{ Entergy’s business elsewhere might be affected.
   }{ the authority of the NRC will be defied.
   }{ Entergy will withdraw its Plymouth application.
   }{ Vermont’s reputation might be damaged.

}\end{questions}      \subsection{Text 3}
ln the idealized version of how science is done, facts about the world are waiting to be observed and collected by objective researchers who use the scientific method to carry out their work. But in the everyday practice of science, discovery frequently follows an ambiguous and complicated route. We aim to be objective, but we cannot escape the context of our unique life experience. Prior knowledge and interests influence what we experience, what we think our experiences mean, and the subsequent actions we take. Opportunities for misinterpretation, error, and self-deception abound.
Consequently, discovery claims should be thought of as protoscience. Similar to newly staked mining claims, they are full of potential. But it takes collective scrutiny and acceptance to transform a discovery claim into a mature discovery. This is the credibility process, through which the individual researcher’s me, here, now becomes the community’s anyone, anywhere, anytime. Objective knowledge is the goal, not the starting point.
Once a discovery claim becomes public, the discoverer receives intellectual credit. But, unlike with mining claims, the community takes control of what happens next. Within the complex social structure of the scientific community, researchers make discoveries; editors and reviewers act as gatekeepers by controlling the publication process; other scientists use the newfinding to suit their own purposes; and finally, the public (including other scientists) receives the new discovery and possibly accompanying technology. As a discovery claim works its way through the community, the interaction and confrontation between shared and competing beliefs about the scienceand the technology involved transforms an individual’s discovery claim into the community’s credible discovery.
Two paradoxes exist throughout this credibility process. First, scientific work tends to focus on some aspect of prevailing knowledge that is viewed as incomplete or incorrect. Little reward accompanies duplication and confirmation of what is already known and believed. The goal is new-search, not re-search. Not surprisingly, newly published discovery claims and credible discoveries that appear to be important and convincing will always be open to challenge and potential modification or refutation by future researchers. Second, novelty itself frequently provokes disbelief. Nobel Laureate and physiologist Albert Szent-Gyorgyi once described discovery as “seeing what everybody has seen and thinking what nobody has thought.” But thinking what nobody else has thought and telling others what they have missed may not change their views. Sometimes years are required for truly novel discovery claims to be accepted and appreciated.
ln the end, credibility “happens” to a discovery claim—a process that corresponds to what philosopher Annette Baier has described as the commons of the mind. “We reason together, challenge, revise, and complete each other’s reasoning and each other’s conceptions of reason.”

\begin{questions} \sethlcolor{cyan}\question 31. According to the first paragraph, the process of discovery is characterized by its        .
   \\ \fourch{ uncertainty and complexity    }{ misconception and deceptiveness
   }{ logicality and objectivity      }{ systematicness and regularity
}\question 32. It can be inferred from Paragraph 2 that the credibility process requires         .
   \\ \fourch{ strict inspection                           }{ shared efforts
   }{ individual wisdom                         }{ persistent innovation
}\question   33. Paragraph 3 shows that a discovery claim becomes credible after it         .
   \\ \fourch{ has attracted the attention of the general public
   }{ has been examined by the scientific community
   }{ has received recognition from editors and reviewers
   }{ has been frequently quoted by peer scientists
}\question 34. Albert Szent-Gyorgyi would most likely agree that         .
   \\ \fourch{ scientific claims will survive challenges     	}{ discoveries today inspire future research
   }{ efforts to make discoveries are justified     	}{ scientific work calls for a critical mind
   }\question 35.Which of the following would be the best title of the text?
   \\ \fourch{ Novelty as an Engine of Scientific Development.
   }{ Collective Scrutiny in Scientific Discovery.
   }{ Evolution of Credibility in Doing Science.
   }{ Challenge to Credibility at the Gate to Science.

}\end{questions}      \subsection{Text 4}
If the trade unionist Jimmy Hoffa were alive today, he would probably represent civil servants.When Hoffa’s Teamsters were in their prime in 1960, only one in ten American government workers belonged to a union; now 36% do. In 2009 the number of unionists in America’s public sector passed that of their fellow members in the private sector. In Britain, more than half of public-sector workers but only about 15\% of private-sector ones are unionized.
There are three reasons for the public-sector unions’ thriving. First, they can shut things down without suffering much in the way of consequences. Second, they are mostly bright and well-educated. A quarter of America’s public-sector workers have a university degree. Third, they now dominate left-of-centre politics. Some of their ties go back a long way. Britain’s Labor Party, as its name implies, has long been associated with trade unionism. Its current leader, Ed Miliband, owes his position to votes from public-sector unions.
At the state level their influence can be even more fearsome. Mark Baldassare of the Public Policy Institute of California points out that much of the state’s budget is patrolled by unions. The teachers’unions keep an eye on schools, the CCPOA on prisons and a variety of labor groups on health care.
ln many rich countries average wages in the state sector are higher than in the private one. But the real gains come in benefits and work practices. Politicians have repeatedly “backloaded” public-sector pay deals, keeping the pay increases modest but adding to holidays and especially pensions that are already generous.
Reform has been vigorously opposed, perhaps most egregiously in education, where charter schools, academies and merit pay all faced drawn-out battles. Even though there is plenty of evidence that the quality of the teachers is the most important variable, teachers’ unions have fought against getting rid of bad ones and promoting good ones.
As the cost to everyone else has become clearer, politicians have begun to clamp down. In Wisconsin the unions have rallied thousands of supporters against Scott Walker, the hardline Republican governor. But many within the public sector suffer under the current system, too.
John Donahue at Harvard’s Kennedy School points out that the norms of culture in Western civil services suit those who want to stay put but is bad for high achievers. The only American public-sector workers who earn well above \$ 250,000 a year are university sports coaches and the president of the United States. Bankers’fat pay packets have attracted much criticism, but a public-sector system that does not reward high achievers may be a much bigger problem for America.

\begin{questions} \sethlcolor{cyan}\question 36.It can be learned from the first paragraph that        .
   \\ \fourch{ Teamsters still have a large body of members
   }{ Jimmy Hoffa used to work as a civil servant
   }{ unions have enlarged their public-sector membership
   }{ the government has improved its relationship with unionists
   }\question 37.Which of the following is true of Paragraph 2?
   \\ \fourch{ Public-sector unions are prudent in taking actions.
   }{ Education is required for public-sector union membership.
   }{ Labor Party has long been fighting against public-sector unions.
   }{ Public-sector unions seldom get in trouble for their actions.
   }\question 38.It can be learned from Paragraph 4 that the income in the state sector is        .
   \\ \fourch{ illegally secured                   }{ indirectly augmented
   }{ excessively increased               }{ fairly adjusted
   }\question 39.The example of the unions in Wisconsin shows that unions         .
   \\ \fourch{ often run against the current political system
   }{ can change people’s political attitudes
   }{ may be a barrier to public-sector reforms
   }{ are dominant in the government
   }\question 40.John Donahue’s attitude towards the public-sector system is one of        .
   \\ \fourch{ disapproval      }{ appreciation    }{ tolerance      }{ indifference
}\end{questions}    