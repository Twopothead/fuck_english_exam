
\section{2018}
\subsection{Text 1}
Among the annoying challenges facing the middle class is one that will probably go unmentioned in the next presidential campaign: What happens when the robots come for their jobs?
Don't dismiss that possibility entirely. About half of U.S. jobs are at high risk of being automated, according to a University of Oxford study, with the middle class disproportionately squeezed. Lower-income jobs like gardening or day care don't appeal to robots. But many middle-class occupations-trucking, financial advice, software engineering — have aroused their interest, or soon will. The rich own the robots, so they will be fine.
This isn't to be alarmist. Optimists point out that technological upheaval has benefited workers in the past. The Industrial Revolution didn't go so well for Luddites whose jobs were displaced by mechanized looms, but it eventually raised living standards and created more jobs than it destroyed. Likewise, automation should eventually boost productivity, stimulate demand by driving down prices, and free workers from hard, boring work. But in the medium term, middle-class workers may need a lot of help adjusting.
The first step, as Erik Brynjolfsson and Andrew McAfee argue in The Second Machine Age, should be rethinking education and job training. Curriculums —from grammar school to college- should evolve to focus less on memorizing facts and more on creativity and complex communication. Vocational schools should do a better job of fostering problem-solving skills and helping students work alongside robots. Online education can supplement the traditional kind. It could make extra training and instruction affordable. Professionals trying to acquire new skills will be able to do so without going into debt.
The challenge of coping with automation underlines the need for the U.S. to revive its fading business dynamism: Starting new companies must be made easier. In previous eras of drastic technological change, entrepreneurs smoothed the transition by dreaming up ways to combine labor and machines. The best uses of 3D printers and virtual reality haven't been invented yet. The U.S. needs the new companies that will invent them.
Finally, because automation threatens to widen the gap between capital income and labor income, taxes and the safety net will have to be rethought. Taxes on low-wage labor need to be cut, and wage subsidies such as the earned income tax credit should be expanded: This would boost incomes, encourage work, reward companies for job creation, and reduce inequality.
Technology will improve society in ways big and small over the next few years, yet this will be little comfort to those who find their lives and careers upended by automation. 
Destroying the machines that are coming for our jobs would be nuts. But policies to help workers adapt will be indispensable.
\begin{questions} \sethlcolor{cyan}\question 21.Who will be most threatened by automation?
\\ \fourch{ Leading politicians.
}{Low-wage laborers.
}{Robot owners.
}{Middle-class workers.
}\question 22 .Which of the following best represent the author’s view?
\\ \fourch{ Worries about automation are in fact groundless.
}{Optimists' opinions on new tech find little support.
}{Issues arising from automation need to be tackled
}{Negative consequences of new tech can be avoided
}\question 23. Education in the age of automation should put more emphasis on
\\ \fourch{ creative potential.
}{job-hunting skills.
}{individual needs.
}{cooperative spirit.
}\question 24.The author suggests that tax policies be aimed at
\\ \fourch{ encouraging the development of automation.
}{increasing the return on capital investment.
}{easing the hostility between rich and poor.
}{preventing the income gap from widening.
}\question 25.In this text, the author presents a problem with
\\ \fourch{ opposing views on it.
}{possible solutions to it.
}{its alarming impacts.
}{its major variations.
}\end{questions}      \subsection{Text 2}
A new survey by Harvard University finds more than two-thirds of young Americans disapprove of President Trump’s use of Twitter. The implication is that Millennials prefer news from the White House to be filtered through other source, Not a president’s social media platform.
Most Americans rely on social media to check daily headlines. Yet as distrust has risen toward all media, people may be starting to beef up their media literacy skills. Such a trend is badly needed. During the 2016 presidential campaign, nearly a quarter of web content shared by Twitter users in the politically critical state of Michigan was fake news, according to the University of Oxford. And a survey conducted for BuzzFeed News found 44 percent of Facebook users rarely or never trust news from the media giant.
Young people who are digital natives are indeed becoming more skillful at separating fact from fiction in cyberspace. A Knight Foundation focus-group survey of young people between ages 14and24 found they use “distributed trust” to verify stories. They cross-check sources and prefer news from different perspectives—especially those that are open about any bias. “Many young people assume a great deal of personal responsibility for educating themselves and actively seeking out opposing viewpoints,” the survey concluded.
Such active research can have another effect. A 2014 survey conducted in Australia, Britain, and the United States by the University of Wisconsin-Madison found that young people’s reliance on social media led to greater political engagement.
Social media allows users to experience news events more intimately and immediately while also permitting them to re-share news as a projection of their values and interests. This forces users to be more conscious of their role in passing along information. A survey by Barna research group found the top reason given by Americans for the fake news phenomenon is “reader error,” more so than made-up stories or factual mistakes in reporting. About a third say the problem of fake news lies in “misinterpretation or exaggeration of actual news” via social media. In other words, the choice to share news on social media may be the heart of the issue. “This indicates there is a real personal responsibility in counteracting this problem,” says Roxanne Stone, editor in chief at Barna Group.
So when young people are critical of an over-tweeting president, they reveal a mental discipline in thinking skills – and in their choices on when to share on social media.
\begin{questions} \sethlcolor{cyan}\question 26. According to the Paragraphs 1 and 2, many young Americans cast doubts on
\\ \fourch{ the justification of the news-filtering practice.
}{ people’s preference for social media platforms.
}{ the administrations ability to handle information.
}{ social media was a reliable source of news.
}\question 27. The phrase “beer up”(Line 2, Para. 2) is closest in meaning to
\\ \fourch{ sharpen
}{ define
}{ boast
}{ share
}\question 28. According to the knight foundation survey, young people
\\ \fourch{ tend to voice their opinions in cyberspace.
}{ verify news by referring to diverse resources.
}{ have s strong sense of responsibility.
}{ like to exchange views on “distributed trust”
}\question 29. The Barna survey found that a main cause for the fake news problem is
\\ \fourch{ readers outdated values.
}{ journalists’ biased reporting
}{ readers’ misinterpretation
}{ journalists’ made-up stories.
}\question 30. Which of the following would be the best title for the text?
\\ \fourch{ A Rise in Critical Skills for Sharing News Online
}{ A Counteraction Against the Over-tweeting Trend
}{ The Accumulation of Mutual Trust on Social Media.
}{ The Platforms for Projection of Personal Interests.
}\end{questions}      \subsection{Text 3}
Any fair-minded assessment of the dangers of the deal between Britain's National Health Service (NHS) and DeepMind must start by acknowledging that both sides mean well. DeepMind is one of the leading artificial intelligence (AI) companies in the world. The potential of this work applied to healthcare is very great, but it could also lead to further concentration of power in the tech giants. It Is against that background that the information commissioner, Elizabeth Denham, has issued her damning verdict against the Royal Free hospital trust under the NHS, which handed over to DeepMind the records of 1.6 million patients In 2015 on the basis of a vague agreement which took far too little account of the patients' rights and their expectations of privacy.
DeepMind has almost apologized. The NHS trust has mended its ways. Further arrangements- and there may be many-between the NHS and DeepMind will be carefully scrutinised to ensure that all necessary permissions have been asked of patients and all unnecessary data has been cleaned. There are lessons about informed patient consent to learn. But privacy is not the only angle in this case and not even the most important. Ms Denham chose to concentrate the blame on the NHS trust, since under existing law it “controlled” the data and DeepMind merely “processed" it. But this distinction misses the point that it is processing and aggregation, not the mere possession of bits, that gives the data value.
The great question is who should benefit from the analysis of all the data that our lives now generate. Privacy law builds on the concept of damage to an individual from identifiable knowledge about them. That misses the way the surveillance economy works. The data of an individual there gains its value only when it is compared with the data of countless millions more.
The use of privacy law to curb the tech giants in this instance feels slightly maladapted. This practice does not address the real worry. It is not enough to say that the algorithms DeepMind develops will benefit patients and save lives. What matters is that they will belong to a private monopoly which developed them using public resources. If software promises to save lives on the scale that dugs now can, big data may be expected to behave as a big pharm has done. We are still at the beginning of this revolution and small choices now may turn out to have gigantic consequences later. A long struggle will be needed to avoid a future of digital feudalism. Ms Denham's report is a welcome start.
\begin{questions} \sethlcolor{cyan}\question 31.Wha is true of the agreement between the NHS and DeepMind ?
\\ \fourch{ It caused conflicts among tech giants.
}{ It failed to pay due attention to patient’s rights.
}{ It fell short of the latter's expectations
}{ It put both sides into a dangerous situation.
}\question 32. The NHS trust responded to Denham's verdict with
\\ \fourch{ empty promises.
}{ tough resistance.
}{ necessary adjustments.
}{ sincere apologies.
}\question  33.The author argues in Paragraph 2 that
\\ \fourch{ privacy protection must be secured at all costs.
}{ leaking patients' data is worse than selling it.
}{ making profits from patients' data is illegal.
}{ the value of data comes from the processing of it
}\question 34.According to the last paragraph, the real worry arising from this deal is
\\ \fourch{ the vicious rivalry among big pharmas.
}{ the ineffective enforcement of privacy law.
}{ the uncontrolled use of new software.
}{ the monopoly of big data by tech giants.
}\question 35.The author's attitude toward the application of AI to healthcare is
\\ \fourch{ ambiguous.
}{ cautious.
}{ appreciative.
}{ contemptuous.
}\end{questions}      \subsection{Text 4}
The U.S. Postal Service (USPS) continues to bleed red ink. It reported a net loss of \$5.6 billion for fiscal 2016, the 10th straight year its expenses have exceeded revenue. Meanwhile, it has more than \$120 billion in unfunded liabilities, mostly for employee health and retirement costs. There are many bankruptcies. Fundamentally, the USPS is in a historic squeeze between technological change that has permanently decreased demand for its bread-and-butter product, first-class mail, and a regulatory structure that denies management the flexibility to adjust its operations to the new reality
And interest groups ranging from postal unions to greeting-card makers exert self-interested pressure on the USPS’s ultimate overseer-Congress-insisting that whatever else happens to the Postal Service, aspects of the status quo they depend on get protected. This is why repeated attempts at reform legislation have failed in recent years, leaving the Postal Service unable to pay its bills except by deferring vital modernization.
Now comes word that everyone involved---Democrats, Republicans, the Postal Service, the unions and the system's heaviest users—has finally agreed on a plan to fix the system. Legislation is moving through the House that would save USPS an estimated \$28.6 billion over five years, which could help pay for new vehicles, among other survival measures. Most of the money would come from a penny-per-letter permanent rate increase and from shifting postal retirees into Medicare. The latter step would largely offset the financial burden of annually pre-funding retiree health care, thus addressing a long-standing complaint by the USPS and its union.
If it clears the House, this measure would still have to get through the Senate – where someone is bound to point out that it amounts to the bare, bare minimum necessary to keep the Postal Service afloat, not comprehensive reform. There’s no change to collective bargaining at the USPS, a major omission considering that personnel accounts for 80 percent of the agency’s costs. Also missing is any discussion of eliminating Saturday letter delivery. That common-sense change enjoys wide public support and would save the USPS \$2 billion per year. But postal special-interest groups seem to have killed it, at least in the House. The emerging consensus around the bill is a sign that legislators are getting frightened about a politically embarrassing short-term collapse at the USPS. It is not, however, a sign that they’re getting serious about transforming the postal system for the 21st century.
\begin{questions} \sethlcolor{cyan}\question 36.The financial problem with the USPS is caused partly by
\\ \fourch{. its unbalanced budget.
}{ .its rigid management.
}{ .the cost for technical upgrading.
}{. the withdrawal of bank support.
 }\question 37. According to Paragraph 2, the USPS fails to modernize itself due to
\\ \fourch{. the interference from interest groups.
}{ .the inadequate funding from Congress.
}{ .the shrinking demand for postal service.
}{ .the incompetence of postal unions.
}\question 38.The long-standing complaint by the USPS and its unions can be addressed by
\\ \fourch{ .removing its burden of retiree health care.
}{ .making more investment in new vehicles.
}{ .adopting a new rate-increase mechanism.
}{. attracting more first-class mail users.
}\question 39.In the last paragraph, the author seems to view legislators with
\\ \fourch{ respect.
}{ tolerance.
}{ discontent.
}{ gratitude.
}\question 40.Which of the following would be the best title for the text?
\\ \fourch{ The USPS Starts to Miss Its Good Old Days
}{ The Postal Service: Keep Away from My Cheese
}{ The USPS: Chronic Illness Requires a Quick Cure
}{ The Postal Service Needs More than a Band-Aid
}\end{questions}
