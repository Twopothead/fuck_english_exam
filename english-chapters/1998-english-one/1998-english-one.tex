
\section{1998}
\subsection{Text 1}
Few creations of big technology capture the imagination like giant dams. Perhaps it is humankind’s long suffering at the mercy of flood and drought that makes the idea of forcing the waters to do our bidding so fascinating. But to be fascinated is also, sometimes, to be blind. Several giant dam projects threaten to do more harm than good.

The lesson from dams is that big is not always beautiful. It doesn’t help that building a big, powerful dam has become a symbol of achievement for nations and people striving to assert themselves. Egypt’s leadership in the Arab world was cemented by the Aswan High Dam. Turkey’s bid for First World status includes the giant Ataturk Dam.

But big dams tend not to work as intended. The Aswan Dam, for example, stopped the Nile flooding but deprived Egypt of the fertile silt that floods left -- all in return for a giant reservoir of disease which is now so full of silt that it barely generates electricity.

And yet, the myth of controlling the waters persists. This week, in the heart of civilized Europe, Slovaks and Hungarians stopped just short of sending in the troops in their contention over a dam on the Danube. The huge complex will probably have all the usual problems of big dams. But Slovakia is bidding for independence from the Czechs, and now needs a dam to prove itself.

Meanwhile, in India, the World Bank has given the go-ahead to the even more wrong-headed Narmada Dam. And the bank has done this even though its advisors say the dam will cause hardship for the powerless and environmental destruction. The benefits are for the powerful, but they are far from guaranteed.

Proper, scientific study of the impacts of dams and of the cost and benefits of controlling water can help to resolve these conflicts. Hydroelectric power and flood control and irrigation are possible without building monster dams. But when you are dealing with myths, it is hard to be either proper, or scientific. It is time that the world learned the lessons of Aswan. You don’t need a dam to be saved.
\begin{questions} \sethlcolor{cyan}   \question 52.	The third sentence of paragraph 1 implies that \ltk{}.\\
\fourch{ people would be happy if they shut their eyes to reality
}{ the blind could be happier than the sighted
}{ over-excited people tend to neglect vital things
}{ fascination makes people lose their eyesight
}  \question 52.	In paragraph 5, “the powerless” probably refers to \ltk{}.\\
\fourch{ areas short of electricity
}{ dams without power stations
}{ poor countries around India
}{ common people in the Narmada Dam area
}  \question 53.	What is the myth concerning giant dams?\\
\fourch{ They bring in more fertile soil.
}{ They help defend the country.
}{ They strengthen international ties.
}{ They have universal control of the waters.
}  \question 54.	What the author tries to suggest may best be interpreted as \ltk{}.\\
\fourch{ “It’s no use crying over spilt milk”
}{ “More haste, less speed”
}{ “Look before you leap”
}{ “He who laughs last laughs best”
}\end{questions}

\subsection{Text 2}
Well, no gain without pain, they say. But what about pain without gain? Everywhere you go in America, you hear tales of corporate revival. What is harder to establish is whether the productivity revolution that businessmen assume they are presiding over is for real.
The official statistics are mildly discouraging. They show that, if you lump manufacturing and services together, productivity has grown on average by 1.2% since 1987. That is somewhat faster than the average during the previous decade. And since 1991, productivity has increased by about 2% a year, which is more than twice the 1978-1987 average. The trouble is that part of the recent acceleration is due to the usual rebound that occurs at this point in a business cycle, and so is not conclusive evidence of a revival in the underlying trend. There is, as Robert Rubin, the treasury secretary, says, a “disjunction” between the mass of business anecdote that points to a leap in productivity and the picture reflected by the statistics.
Some of this can be easily explained. New ways of organizing the workplace -- all that re-engineering and downsizing -- are only one contribution to the overall productivity of an economy, which is driven by many other factors such as joint investment in equipment and machinery, new technology, and investment in education and training. Moreover, most of the changes that companies make are intended to keep them profitable, and this need not always mean increasing productivity: switching to new markets or improving quality can matter just as much.
Two other explanations are more speculative. First, some of the business restructuring of recent years may have been ineptly done. Second, even if it was well done, it may have spread much less widely than people suppose.
Leonard Schlesinger, a Harvard academic and former chief executive of Au Bong Pain, a rapidly growing chain of bakery cafes, says that much “re-engineering” has been crude. In many cases, he believes, the loss of revenue has been greater than the reductions in cost. His colleague, Michael Beer, says that far too many companies have applied re-engineering in a mechanistic fashion, chopping out costs without giving sufficient thought to long term profitability. BBDO’s Al Rosenshine is blunter. He dismisses a lot of the work of re-engineering consultants as mere rubbish -- “the worst sort of ambulance cashing.”
\begin{questions} \sethlcolor{cyan} \question 55.	According to the author, the American economic situation is \ltk{}.\\
\fourch{ not as good as it seems
}{ at its turning point
}{ much better than it seems
}{ near to complete recovery
} \question 56.	The official statistics on productivity growth \ltk{}.\\
\fourch{ exclude the usual rebound in a business cycle
}{ fall short of businessmen’s anticipation
}{ meet the expectation of business people
}{ fail to reflect the true state of economy
} \question 57.	The author raises the question “what about pain without gain?” because \ltk{}.\\
\fourch{ he questions the truth of “no gain without pain”
}{ he does not think the productivity revolution works
}{ he wonders if the official statistics are misleading
}{ he has conclusive evidence for the revival of businesses
} \question 58.	Which of the following statements is NOT mentioned in the passage?\\
\fourch{ Radical reforms are essential for the increase of productivity.
}{ New ways of organizing workplaces may help to increase productivity.
}{ The reduction of costs is not a sure way to gain long term profitability.
}{ The consultants are a bunch of good-for-nothings.
}\end{questions}

\subsection{Text 3}
Science has long had an uneasy relationship with other aspects of culture. Think of Gallileo’s 17th century trial for his rebelling belief before the Catholic Church or poet William Blake’s harsh remarks against the mechanistic worldview of Isaac Newton. The schism between science and the humanities has, if anything, deepened in this century.
Until recently, the scientific community was so powerful that it could afford to ignore its critics -- but no longer. As funding for science has declined, scientists have attacked “antiscience” in several books, notably Higher Superstition, by Paul R. Gross, a biologist at the University of Virginia, and Norman Levitt, a mathematician at Rutgers University; and The Demon-Haunted World, by Carl Sagan of Cornell University.
Defenders of science have also voiced their concerns at meetings such as “The Flight from Science and Reason,” held in New York City in 1995, and “Science in the Age of (Mis) information,” which assembled last June near Buffalo.
Antiscience clearly means different things to different people. Gross and Levitt find fault primarily with sociologists, philosophers and other academics who have questioned science’s objectivity. Sagan is more concerned with those who believe in ghosts, creationism and other phenomena that contradict the scientific worldview.
A survey of news stories in 1996 reveals that the antiscience tag has been attached to many other groups as well, from authorities who advocated the elimination of the last remaining stocks of smallpox virus to Republicans who advocated decreased funding for basic research.
Few would dispute that the term applies to the Unabomber, whose manifesto, published in 1995, scorns science and longs for return to a pre-technological utopia. But surely that does not mean environmentalists concerned about uncontrolled industrial growth are antiscience, as an essay in US News \& World Report last May seeWorld Report last May seemed to suggest.
The environmentalists, inevitably, respond to such critics. The true enemies of science, argues Paul Ehrlich of Stanford University, a pioneer of environmental studies, are those who question the evidence supporting global warming, the depletion of the ozone layer and other consequences of industrial growth.
Indeed, some observers fear that the antiscience epithet is in danger of becoming meaningless. “The term ‘antiscience’ can lump together too many, quite different things,” notes Harvard University philosopher Gerald Holton in his 1993 work Science and Anti-Science. “They have in common only one thing that they tend to annoy or threaten those who regard themselves as more enlightened.”

\begin{questions} \sethlcolor{cyan} \question 59.	The word “schism” (Line 4, Paragraph 1) in the context probably means \ltk{}.\\
\fourch{ confrontation
}{ dissatisfaction
}{ separation
}{ contempt
} \question 60.	Paragraphs 2 and 3 are written to \ltk{}.\\
\fourch{ discuss the cause of the decline of science’s power
}{ show the author’s sympathy with scientists
}{ explain the way in which science develops
}{ exemplify the division of science and the humanities
} \question 61.	Which of the following is true according to the passage?\\
\fourch{ Environmentalists were blamed for antiscience in an essay.
}{ Politicians are not subject to the labeling of antiscience.
}{ The “more enlightened” tend to tag others as antiscience.
}{ Tagging environmentalists as “antiscience” is justifiable.
} \question 62.	The author’s attitude toward the issue of “science vs. antiscience” is \ltk{}.\\
\fourch{ impartial
}{ subjective
}{ biased
}{ puzzling
}\end{questions}

\subsection{Text 4}
Emerging from the 1980 census is the picture of a nation developing more and more regional competition, as population growth in the Northeast and Midwest reaches a near standstill.
This development -- and its strong implications for US politics and economy in years ahead -- has enthroned the South as America’s most densely populated region for the first time in the history of the nation’s head counting.
Altogether, the US population rose in the 1970s by 23.2 million people -- numerically the third largest growth ever recorded in a single decade. Even so, that gain adds up to only 11.4 percent, lowest in American annual records except for the Depression years.
Americans have been migrating south and west in larger number since World War II, and the pattern still prevails.
Three sun-belt states -- Florida, Texas and California -- together had nearly 10 million more people in 1980 than a decade earlier. Among large cities, San Diego moved from 14th to 8th and San Antonio from 15th to 10th -- with Cleveland and Washington. D. C. dropping out of the top 10.
Not all that shift can be attributed to the movement out of the snow belt, census officials say, Nonstop waves of immigrants played a role, too -- and so did bigger crops of babies as yesterday’s “baby boom” generation reached its child bearing years.
Moreover, demographers see the continuing shift south and west as joined by a related but newer phenomenon: More and more, Americans apparently are looking not just for places with more jobs but with fewer people, too. Some instances—
■Regionally, the Rocky Mountain states reported the most rapid growth rate -- 37.1 percent since 1970 in a vast area with only 5 percent of the US population.
■Among states, Nevada and Arizona grew fastest of all: 63.5 and 53.1 percent respectively. Except for Florida and Texas, the top 10 in rate of growth is composed of Western states with 7.5 million people -- about 9 per square mile.
The flight from overcrowdedness affects the migration from snow belt to more bearable climates.
Nowhere do 1980 census statistics dramatize more the American search for spacious living than in the Far West. There, California added 3.7 million to its population in the 1970s, more than any other state.
In that decade, however, large numbers also migrated from California, mostly to other parts of the West. Often they chose -- and still are choosing -- somewhat colder climates such as Oregon, Idaho and Alaska in order to escape smog, crime and other plagues of urbanization in the Golden State.
As a result, California’s growth rate dropped during the 1970s, to 18.5 percent -- little more than two thirds the 1960s’ growth figure and considerably below that of other Western states.
\begin{questions} \sethlcolor{cyan} \question 63.	Discerned from the perplexing picture of population growth the 1980 census provided, America in 1970s \ltk{}.\\
\fourch{ enjoyed the lowest net growth of population in history
}{ witnessed a southwestern shift of population
}{ underwent an unparalleled period of population growth
}{ brought to a standstill its pattern of migration since World War II
} \question 64.	The census distinguished itself from previous studies on population movement in that \ltk{}.\\
\fourch{ it stresses the climatic influence on population distribution
}{ it highlights the contribution of continuous waves of immigrants
}{ it reveals the Americans’ new pursuit of spacious living
}{ it elaborates the delayed effects of yesterday’s “baby boom”
} \question 65.	We can see from the available statistics that \ltk{}.\\
\fourch{ California was once the most thinly populated area in the whole US
}{ the top 10 states in growth rate of population were all located in the West
}{ cities with better climates benefited unanimously from migration
}{ Arizona ranked second of all states in its growth rate of population
} \question 66.	The word “demographers” (Line 1, Paragraph 8) most probably means \ltk{}.\\
\fourch{ people in favor of the trend of democracy
}{ advocates of migration between states
}{ scientists engaged in the study of population
}{ conservatives clinging to old patterns of life
}
\end{questions}    \subsection{Text 5}
Scattered around the globe are more than 100 small regions of isolated volcanic activity known to geologists as hot spots. Unlike most of the world’s volcanoes, they are not always found at the boundaries of the great drifting plates that make up the earth’s surface; on the contrary, many of them lie deep in the interior of a plate. Most of the hot spots move only slowly, and in some cases the movement of the plates past them has left trails of dead volcanoes. The hot spots and their volcanic trails are milestones that mark the passage of the plates.
That the plates are moving is now beyond dispute. Africa and South America, for example, are moving away from each other as new material is injected into the sea floor between them. The complementary coastlines and certain geological features that seem to span the ocean are reminders of where the two continents were once joined. The relative motion of the plates carrying these continents has been constructed in detail, but the motion of one plate with respect to another cannot readily be translated into motion with respect to the earth’s interior. It is not possible to determine whether both continents are moving in opposite directions or whether one continent is stationary and the other is drifting away from it. Hot spots, anchored in the deeper layers of the earth, provide the measuring instruments needed to resolve the question. From an analysis of the hot-spot population it appears that the African plate is stationary and that it has not moved during the past 30 million years.
The significance of hot spots is not confined to their role as a frame of reference. It now appears that they also have an important influence on the geophysical processes that propel the plates across the globe. When a continental plate come to rest over a hot spot, the material rising from deeper layer creates a broad dome. As the dome grows, it develops seed fissures (cracks); in at least a few cases the continent may break entirely along some of these fissures, so that the hot spot initiates the formation of a new ocean. Thus just as earlier theories have explained the mobility of the continents, so hot spots may explain their mutability (inconstancy).
\begin{questions} \sethlcolor{cyan}   \question 67.	The author believes that \ltk{}.\\
\fourch{ the motion of the plates corresponds to that of the earth’s interior
}{ the geological theory about drifting plates has been proved to be true
}{ the hot spots and the plates move slowly in opposite directions
}{ the movement of hot spots proves the continents are moving apart
}    \question 68.	That Africa and South America were once joined can be deduced from the fact that \ltk{}.\\
\fourch{ the two continents are still moving in opposite directions
}{ they have been found to share certain geological features
}{ the African plates has been stable for 30 million years
}{ over 100 hot spots are scattered all around the globe
}    \question 69.	The hot spot theory may prove useful in explaining \ltk{}.\\
\fourch{ the structure of the African plates
}{ the revival of dead volcanoes
}{ the mobility of the continents
}{ the formation of new oceans
}    \question 70.	The passage is mainly about \ltk{}.\\
\fourch{ the features of volcanic activities
}{ the importance of the theory about drifting plates
}{ the significance of hot spots in geophysical studies
}{ the process of the formation of volcanoes
}\end{questions} 
