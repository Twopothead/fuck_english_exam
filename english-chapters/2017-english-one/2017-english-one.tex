
\section{2017}
\subsection{Text 1} 
First two hours , now three hours—this is how far in advance authorities are recommending people show up to catch a domestic flight , at least at some major U.S. airports with increasingly massive security lines. 
Americans are willing to tolerate time-consuming security procedures in return for increased safety. The crash of Egypt Air Flight 804,which terrorists may have downed over the Mediterranean Sea ,provides another tragic reminder of why. But demanding too much of air travelers or providing too little security in return undermines public support for the process. And it should: Wasted time is a drag on Americans’ economic and private lives, not to mention infuriating. 
Last year, the Transportation Security Administration (TSA) found in a secret check that undercover investigators were able to sneak weapons---both fake and real—past airport security nearly every time they tried .Enhanced security measures since then, combined with a rise in airline travel due to the improving Chicago’s O’Hare International .It is not yet clear how much more effective airline security has become—but the lines are obvious. 
Part of the issue is that the government did not anticipate the steep increase in airline travel , so the TSA is now rushing to get new screeners on the line. Part of the issue is that airports have only so much room for screening lanes. Another factor may be that more people are trying to overpack their carry-on bags to avoid checked-baggage fees, though the airlines strongly dispute this. 
There is one step the TSA could take that would not require remodeling airports or rushing to hire: Enroll more people in the PreCheck program. PreCheck is supposed to be a win-win for travelers and the TSA. Passengers who pass a background check are eligible to use expedited screening lanes. This allows the TSA wants to enroll 25 million people in PreCheck. 
It has not gotten anywhere close to that, and one big reason is sticker shock. Passengers must pay \$85 every five years to process their background checks. Since the beginning, this price tag has been PreCheck’s fatal flaw. Upcoming reforms might bring the price to a more reasonable level. But Congress should look into doing so directly, by helping to finance PreCheck enrollment or to cut costs in other ways. 
The TSA cannot continue diverting resources into underused PreCheck lanes while most of the traveling public suffers in unnecessary lines. It is long past time to make the program work. 
\begin{questions} \sethlcolor{cyan}\question 21. According to Paragraph 1, Parkrun has\ltk{}. 
\\ \fourch{ gained great popularity 
}{ created many jobs 
}{strengthened community ties 
}{ become an official festival 
}\question 22. The author believes that London’s Olympic “legacy” has failed to \ltk{}. 
\\ \fourch{ boost population growth 
}{ promote sport participation 
}{improve the city’s image 
}{ increase sport hours in schools 
}\question 23.  Parkrun is different form Olympic games in that it \ltk{}. 
\\ \fourch{ aims at discovering talents 
}{ focuses on mass competition 
}{ does not emphasize elitism 
}{ does not attract first-timers 
}\question 24. With regard to mass sports, the author holds that governments should\ltk{}. 
\\ \fourch{ organize “grassroots” sports events 
}{ supervise local sports associations 
}{ increase funds for sports clubs 
}{ invest in pubic sports facilities 
}\question 25. The author’s attitude to what UK governments have to done for sports is \ltk{}. 
\\ \fourch{tolerant 
}{ critical 
}{uncertain 
}{sympathetic 
}\end{questions}      \subsection{Text 2} 
“The ancient Hawaiians were astronomers,” wrote Queen Liliuokalani, Hawaii’s last reigning monarch, in 1897. Star watchers were among the most esteemed members of Hawaiian society. Sadly, all is not well with astronomy in Hawaii today. Protests have erupted over construction of the Thirty Meter Telescope(TMT), a giant observatory that promises to revolutionize humanity’s view of the cosmos. 
At issue is the TMT’s planned location on Mauna Kea, a dormant volcano worshiped by some Hawaiians as the piko , that connects the Hawaiian Islands to the heavens. But Mauna Kea is also home to some of the world’s most powerful telescopes. Rested in the Pacific Ocean, Mauna Kea’s peak rises above the bulk of our planet’s dense atmosphere, where conditions allow telescopes to obtain images of unsurpassed clarity. 
Opposition to telescopes on Mauna Kea is nothing new. A small but vocal group of Hawaiians and environments have long viewed their presence as disrespect for sacred land and a painful reminder of the occupation of what was once a sovereign nation. 
Some blame for the current controversy belongs to astronomers. In their eagerness to build bigger telescopes, they forgot that science is the only way of understanding the world. They did not always prioritize the protection of Mauna Kea’s fragile ecosystems or its holiness to the island’s inhabitants. Hawaiian culture is not a relic of the past; it is a living culture undergoing a renaissance today. 
Yet science has a cultural history, too, with roots going back to the dawn of civilization. The same curiosity to find what lies beyond the horizon that first brought early Polynesians to Hawaii’s shores inspires astronomers today to explore the heavens. Calls to disassemble all telescopes on Mauna Kea or to ban future development there ignore the reality that astronomy and Hawaiian culture both seek to answer big questions about who we are, where we come from and where we are going. Perhaps that is why we explore the starry skies, as if answering a primal calling to know ourselves and our true ancestral homes. 
The astronomy community is making compromises to change its use of Mauna Kea. The TMT site was chosen to minimize the telescope’s visibility around the island and to avoid archaeological and environmental impact. To limit the number of telescopes on Mauna Kea, old ones will be removed at the end of their lifetimes and their sites returned to a natural state. There is no reason why everyone cannot be welcomed on Mauna Kea to embrace their cultural heritage and to study the stars. 
\begin{questions} \sethlcolor{cyan}\question 26. Queen Liliuokalani’s remark in Paragraph 1 indicates 
\\ \fourch{ its conservative view on the historical role of astronomy. 
}{ the importance of astronomy in ancient Hawaiian society. 
}{ the regrettable decline of astronomy in ancient times. 
}{ her appreciation of star watchers’ feats in her time. 
}\question 27. Mauna Kea is deemed as an ideal astronomical site due to 
\\ \fourch{ its geographical features 
}{ its protective surroundings. 
}{ its religious implications. 
}{ its existing infrastructure. 
}\question 28. The construction of the TMT is opposed by some locals partly because 
\\ \fourch{ it may risk ruining their intellectual life. 
}{ it reminds them of a humiliating history. 
}{ their culture will lose a chance of revival. 
}{ they fear losing control of Mauna Kea. 
}\question 29. It can be inferred from Paragraph 5 that progress in today’s astronomy 
\\ \fourch{ is fulfilling the dreams of ancient Hawaiians. 
}{ helps spread Hawaiian culture across the world. 
}{ may uncover the origin of Hawaiian culture. 
}{ will eventually soften Hawaiians’ hostility. 
}\question 30. The author’s attitude toward choosing Mauna Kea as the TMT site is one of 
\\ \fourch{ severe criticism. 
}{ passive acceptance. 
}{ slight hesitancy. 
}{ full approval. 
}\end{questions}      \subsection{Text 3} 
Robert F. Kennedy once said that a country’s GDP measures “everything except that which makes life worthwhile.” With Britain voting to leave the European Union, and GDP already predicted to slow as a result, it is now a timely moment to assess what he was referring to. 
The question of GDP and its usefulness has annoyed policymakers for over half a century. Many argue that it is a flawed concept. It measures things that do not matter and misses things that do. By most recent measures, the UK’s GDP has been the envy of the Western world, with record low unemployment and high growth figures. If everything was going so well, then why did over 17 million people vote for Brexit, despite the warnings about what it could do to their country’s economic prospects? 
A recent annual study of countries and their ability to convert growth into well-being sheds some light on that question. Across the 163 countries measured, the UK is one of the poorest performers in ensuring that economic growth is translated into meaningful improvements for its citizens. Rather than just focusing on GDP, over 40 different sets of criteria from health, education and civil society engagement have been measured to get a more rounded assessment of how countries are performing. 
While all of these countries face their own challenges , there are a number of consistent themes . Yes , there has been a budding economic recovery since the 2008 global crash , but in key indicators in areas such as health and education , major economies have continued to decline . Yet this isn’t the case with all countries . Some relatively poor European countries have seen huge improvements across measures including civil society , income equality and the environment. 
This is a lesson that rich countries can learn : When GDP is no longer regarded as the sole measure of a country’s success, the world looks very different . 
So, what Kennedy was referring to was that while GDP has been the most common method for measuring the economic activity of nations , as a measure , it is no longer enough . It does not include important factors such as environmental quality or education outcomes – all things that contribute to a person’s sense of well-being. 
The sharp hit to growth predicted around the world and in the UK could lead to a decline in the everyday services we depend on for our well-being and for growth . But policymakers who refocus efforts on improving well-being rather than simply worrying about GDP figures could avoid the forecasted doom and may even see progress . 
\begin{questions} \sethlcolor{cyan}\question 31.Robert F. Kennedy is cited because he 
\\ \fourch{praised the UK for its GDP. 
}{identified GDP with happiness . 
}{misinterpreted the role of GDP . 
}{had a low opinion of GDP . 
}\question 32.It can be inferred from Paragraph 2 that 
\\ \fourch{the UK is reluctant to remold its economic pattern . 
}{GDP as the measure of success is widely defied in the UK . 
}{the UK will contribute less to the world economy . 
}{policymakers in the UK are paying less attention to GDP . 
}\question  33.Which of the following is true about the recent annual study ? 
\\ \fourch{It is sponsored by 163 countries . 
}{It excludes GDP as an indicator. 
}{Its criteria are questionable . 
}{Its results are enlightening . 
}\question 34.In the last two paragraphs , the author suggests that 
\\ \fourch{the UK is preparing for an economic boom . 
}{high GDP foreshadows an economic decline . 
}{it is essential to consider factors beyond GDP . 
}{it requires caution to handle economic issues . 
}\question 35.Which of the following is the best title for the text ? 
\\ \fourch{High GDP But Inadequate Well-being , a UK Lesson 
}{GDP Figures , a Window on Global Economic Health 
}{Rebort F. Kennedy , a Terminator of GDP 
}{Brexit, the UK’s Gateway to Well-being 
}\end{questions}      \subsection{Text 4} 
In a rare unanimous ruling, the US Supreme Court has overturned the corruption conviction of a former Virginia governor, Robert McDonnell. But it did so while holding its nose at the ethics of his conduct, which included accepting gifts such as a Rolex watch and a Ferrari automobile from a company seeking access to government. 
The high court’s decision said the judge in Mr. McDonnell’s trial failed to tell a jury that it must look only at his “official acts,” or the former governor’s decisions on “specific” and “unsettled” issues related to his duties. 
Merely helping a gift-giver gain access to other officials, unless done with clear intent to pressure those officials, is not corruption, the justices found. 
The court did suggest that accepting favors in return for opening doors is “distasteful” and “nasty.” But under anti-bribery laws, proof must be made of concrete benefits, such as approval of a contract or regulation. Simply arranging a meeting, making a phone call, or hosting an event is not an “official act”. 
The court’s ruling is legally sound in defining a kind of favoritism that is not criminal. Elected leaders must be allowed to help supporters deal with bureaucratic problems without fear of prosecution for bribery.” The basic compact underlying representative government,” wrote Chief Justice John Roberts for the court,” assumes that public officials will hear from their constituents and act on their concerns.” 
But the ruling reinforces the need for citizens and their elected representatives, not the courts, to ensure equality of access to government. Officials must not be allowed to play favorites in providing information or in arranging meetings simply because an individual or group provides a campaign donation or a personal gift. This type of integrity requires well-enforced laws in government transparency, such as records of official meetings, rules on lobbying, and information about each elected leader’s source of wealth. 
Favoritism in official access can fan public perceptions of corruption. But it is not always corruption. Rather officials must avoid double standards, or different types of access for average people and the wealthy. If connections can be bought, a basic premise of democratic society—that all are equal in treatment by government—is undermined. Good governance rests on an understanding of the inherent worth of each individual. 
The court’s ruling is a step forward in the struggle against both corruption and official favoritism. 
\begin{questions} \sethlcolor{cyan}\question 36. The undermined sentence (Para.1) most probably shows that the court 
\\ \fourch{ avoided defining the extent of McDonnell’s duties. 
}{ made no compromise in convicting McDonnell. 
}{ was contemptuous of McDonnell’s conduct. 
}{ refused to comment on McDonnell’s ethics. 
 }\question 37. According to Paragraph 4, an official act is deemed corruptive only if it involves 
\\ \fourch{ leaking secrets intentionally. 
}{ sizable gains in the form of gifts. 
}{ concrete returns for gift-givers. 
}{ breaking contracts officially. 
}\question 38. The court’s ruling is based on the assumption that public officials are 
\\ \fourch{ justified in addressing the needs of their constituents. 
}{ qualified to deal independently with bureaucratic issues. 
}{ allowed to focus on the concerns of their supporters. 
}{ exempt from conviction on the charge of favoritism. 
}\question 39. Well-enforced laws in government transparency are needed to 
\\ \fourch{ awaken the conscience of officials. 
}{ guarantee fair play in official access. 
}{ allow for certain kinds of lobbying. 
}{ inspire hopes in average people. 
}\question 40. The author’s attitude toward the court’s ruling is 
\\ \fourch{ sarcastic. 
}{ tolerant. 
}{ skeptical. 
}{ supportive 
}\end{questions}    