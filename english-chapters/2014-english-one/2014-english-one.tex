   
\section{2014}
\subsection{Text 1} 
In order to “change lives for the better” and reduce “dependency,” George Osbome, Chancellor of the Exchequer, introduced the “upfront work search” scheme. Only if the jobless arrive at the job centre with a CV register for online job search, and start looking for work will they be eligible for benefit-and then they should report weekly rather than fortnightly. What could be more reasonable?
More apparent reasonableness followed. There will now be a seven-day wait for the jobseeker’s allowance. “Those first few days should be spent looking for work, not looking to sign on.” he claimed. “We’re doing these things because we know they help people say off benefits and help those on benefits get into work faster” Help? Really? On first hearing, this was the socially concerned chancellor, trying to change lives for the better, complete with “reforms” to an obviously indulgent system that demands too little effort from the newly unemployed to find work, and subsides laziness. What motivated him, we were to understand, was his zeal for “fundamental fairness”-protecting the taxpayer, controlling spending and ensuring that only the most deserving claimants received their benefits.
Losing a job is hurting: you don’t skip down to the jobcentre with a song in your heart, delighted at the prospect of doubling your income from the generous state. It is financially terrifying psychologically embarrassing and you know that support is minimal and extraordinarily hard to get. You are now not wanted; you support is minimal and extraordinarily hard to get. You are now not wanted; you are now excluded from the work environment that offers purpose and structure in your life. Worse, the crucial income to feed yourself and your family and pay the bills has disappeared. Ask anyone newly unemployed what they want and the answer is always: a job. 
But in Osborneland, your first instinct is to fall into dependency- permanent dependency if you can get it-supported by a state only too ready to indulge your falsehood. It is as though 20 years of ever- tougher reforms of the job search and benefit administration system never happened. The principle of British welfare is no longer that you can insure yourself against the risk of unemployment and receive unconditional payments if the disaster happens. Even the very phrase ‘jobseeker’s allowance’-invented in 1996- is about redefining the unemployed as a “jobseeker” who had no mandatory right to a benefit he or she has earned through making national insurance contributions. Instead, the claimant receives a time-limited “allowance,” conditional on actively seeking a job; no entitlement and no insurance, at £71.70 a week, one of the least generous in the EU.

\begin{questions} \sethlcolor{cyan}\question 21. George Osborne’s scheme was intended to \ltk{}.
\\ \fourch{ motivate the unemployed to report voluntarily
}{ provide the unemployed with easier access to benefits
}{ encourage jobseekers’ active engagement in job seeking
}{ guarantee jobseekers’ legitimate right to benefits
}\question 22. The phrase “to sign on” (Line 3, Para. 2) most probably means \ltk{}.
\\ \fourch{ to register for an allowance from the government
}{ to accept the government’s restrictions on the allowance
}{ to check on the availability of jobs at the jobcentre
}{ to attend a governmental job-training program
}\question 23.  What prompted the chancellor to develop his scheme?
\\ \fourch{ A desire to secure a better life for all.
}{ An eagerness to protect the unemployed.
}{ An urge to be generous to the claimants.
}{ A passion to ensure fairness for taxpayers.
}\question 24. According to Paragraph 3, being unemployed makes one feel \ltk{}.
\\ \fourch{ insulted
}{ uneasy
}{ enraged
}{ guilty
}\question 25. To which of the following would the author most probably agree?
\\ \fourch{ Unemployment benefits should not be made conditional.
}{ The British welfare system indulges jobseekers’ laziness.
}{ The jobseekers’ allowance has met their actual needs.
}{ Osborne’s reforms will reduce the risk of unemployment.

}\end{questions}      \subsection{Text 2}
  All around the world, lawyers generate more hostility than the members of any other profession---with the possible exception of journalism. But there are few places where clients have more grounds for complaint than America.
   During the decade before the economic crisis, spending on legal services in America grew twice as fast as inflation. The best lawyers made skyscrapers-full of money, tempting ever more students to pile into law schools. But most law graduates never get a big-firm job. Many of them instead become the kind of nuisance-lawsuit filer that makes the tort system a costly nightmare.
   There are many reasons for this. One is the excessive costs of a legal education. There is just one path for a lawyer in most American states: a four-year undergraduate degree at one of 200 law schools authorized by the American Bar Association and an expensive preparation for the bar exam. This leaves today’s average law-school graduate with \$100,000 of debt on top of undergraduate debts. Law-school debt means that they have to work fearsomely hard.
  Reforming the system would help both lawyers and their customers. Sensible ideas have been around for a long time, but the state-level bodies that govern the profession have been too conservative to implement them. One idea is to allow people to study law as an undergraduate degree. Another is to let students sit for the bar after only two years of law school. If the bar exam is truly a stern enough test for a would-be lawyer, those who can sit it earlier should be allowed to do so. Students who do not need the extra training could cut their debt mountain by a third.
The other reason why costs are so high is the restrictive guild-like ownership structure of the business. Except in the District of Columbia, non-lawyers may not own any share of a law firm. This keeps fees high and innovation slow. There is pressure for change from within the profession, but opponents of change among the regulators insist that keeping outsiders out of a law firm isolates lawyers from the pressure to make money rather than serve clients ethically.
In fact, allowing non-lawyers to own shares in law firms would reduce costs and improve services to customers, by encouraging law firms to use technology and to employ professional managers to focus on improving firms’ efficiency. After all, other countries, such as Australia and Britain, have started liberalizing their legal professions. America should follow.
\begin{questions} \sethlcolor{cyan}\question 26. A lot of students take up law as their profession due to \ltk{}.
\\ \fourch{ the growing demand from clients
}{ the increasing pressure of inflation
}{ the prospect of working in big firms
}{ the attraction of financial rewards
}\question 27. Which of the following adds to the costs of legal education in most American states?
\\ \fourch{ Higher tuition fees for undergraduate studies.
}{ Pursuing a bachelor’s degree in another major.
}{ Admissions approval from the bar association.
}{ Receiving training by professional associations.
}\question 28. Hindrance to the reform of the legal system originates from \ltk{}.
\\ \fourch{ non-professionals’ sharp criticism
}{ lawyers’ and clients’ strong resistance
}{ the rigid bodies governing the profession
}{ the stern exam for would-be lawyers
}\question 29. The guild-like ownership structure is considered “restrictive” partly because it \ltk{}.
\\ \fourch{ prevents lawyers from gaining due profits
}{ keeps lawyers from holding law-firm shares
}{ aggravates the ethical situation in the trade
}{ bans outsiders’ involvement in the profession
}\question 30. In this text, the author mainly discusses \ltk{}.
\\ \fourch{ flawed ownership of America’s law firms and its causes
}{ the factors that help make a successful lawyer in America
}{ a problem in America’s legal profession and solutions to it
}{ the role of undergraduate studies in America’s legal education
}\end{questions}      \subsection{Text 3}
The US\$3-million Fundamental physics prize is indeed an interesting experiment, as Alexander Polyakov said when he accepted this year’s award in March. And it is far from the only one of its type. As a News Feature article in Nature discusses, a string of lucrative awards for researchers have joined the Nobel Prizes in recent years. Many, like the Fundamental Physics Prize, are funded from the telephone-number-sized bank accounts of Internet entrepreneurs. These benefactors have succeeded in their chosen fields, they say, and they want to use their wealth to draw attention to those who have succeeded in science.
What’s not to like? Quite a lot, according to a handful of scientists quoted in the News Feature. You cannot buy class, as the old saying goes, and these upstart entrepreneurs cannot buy their prizes the prestige of the Nobels. The new awards are an exercise in self-promotion for those behind them, say scientists. They could distort the achievement-based system of peer-review-led research. They could cement the status quo of peer-reviewed research. They do not fund peer-reviewed research. They perpetuate the myth of the lone genius.
The goals of the prize-givers seem as scattered as the criticism. Some want to shock, others to draw people into science, or to better reward those who have made their careers in research.
As Nature has pointed out before, there are some legitimate concerns about how science prizes—both new and old—are distributed. The Breakthrough Prize in Life Sciences, launched this year, takes an unrepresentative view of what the life sciences include. But the Nobel Foundation’s limit of three recipients per prize, each of whom must still be living, has long been outgrown by the collaborative nature of modern research—as will be demonstrated by the inevitable row over who is ignored when it comes to acknowledging the discovery of the Higgs boson. The Nobels were, of course, themselves set up by a very rich individual who had decided what he wanted to do with his own money. Time, rather than intention, has given them legitimacy.
As much as some scientists may complain about the new awards, two things seem clear. First, most researchers would accept such a prize if they were offered one. Second, it is surely a good thing that the money and attention come to science rather than go elsewhere, It is fair to criticize and question the mechanism—that is the culture of research, after all—but it is the prize-givers’ money to do with as they please. It is wise to take such gifts with gratitude and grace.
\begin{questions} \sethlcolor{cyan}\question 31. The Fundamental Physics Prize is seen as \ltk{}.
\\ \fourch{ a symbol of the entrepreneurs’ wealth
}{ a possible replacement of the Nobel Prizes
}{ a handsome reward for researchers
}{ an example of bankers’ investments
}\question 32. The critics think that the new awards will most benefit \ltk{}.
\\ \fourch{ the profit-oriented scientists
}{ the founders of the awards
}{ the achievement-based system
}{ peer-review-led research
}\question   33. The discovery of the Higgs boson is a typical case which involves \ltk{}.
\\ \fourch{ the joint effort of modern researchers
}{ controversies over the recipients’ status
}{ the demonstration of research findings
}{ legitimate concerns over the new prizes
}\question   34. According to Paragraph 4, which of the following is true of the Nobels?
\\ \fourch{ History has never cast doubt on them.
}{ They are the most representative honor.
}{ Their legitimacy has long been in dispute.
}{ Their endurance has done justice to them.
}\question 35. The author believes that the new awards are \ltk{}.
\\ \fourch{ harmful to the culture of research
}{ acceptable despite the criticism
}{ subject to undesirable changes
}{ unworthy of public attention
 }\end{questions}      \subsection{Text 4}
“The Heart of the Matter,” the just-released report by the American Academy of Arts and Sciences (AAAS), deserves praise for affirming the importance of the humanities and social sciences to the prosperity and security of liberal democracy in America. Regrettably, however, the report's failure to address the true nature of the crisis facing liberal education may cause more harm than good.
  In 2010, leading congressional Democrats and Republicans sent letters to the AAAS asking that it identify actions that could be taken by "federal, state and local governments, universities, foundations, educators, individual benefactors and others" to "maintain national excellence in humanities and social scientific scholarship and education."In response, the American Academy formed the Commission on the Humanities and Social Sciences. Among the commission's 51 members are top-tier-university presidents, scholars, lawyers, judges, and business executives, as well as prominent figures from diplomacy, filmmaking, music and journalism.
The goals identified in the report are generally admirable. Because representative government presupposes an informed citizenry, the report supports full literacy; stresses the study of history and government, particularly American history and American government; and encourages the use of new digital technologies. To encourage innovation and competition, the report calls for increased investment in research, the crafting of coherent curricula that improve students' ability to solve problems and communicate effectively in the 21st century, increased funding for teachers and the encouragement of scholars to bring their learning to bear on the great challenges of the day. The report also advocates greater study of foreign languages, international affairs and the expansion of study abroad programs.
  Unfortunately, despite 2½ years in the making, "The Heart of the Matter" never gets to the heart of the matter: the illiberal nature of liberal education at our leading colleges and universities. The commission ignores that for several decades America's colleges and universities have produced graduates who don't know the content and character of liberal education and are thus deprived of its benefits. Sadly, the spirit of inquiry once at home on campus has been replaced by the use of the humanities and social sciences as vehicles for publicizing "progressive," or left-liberal propaganda.
  Today, professors routinely treat the progressive interpretation of history and progressive public policy as the proper subject of study while portraying conservative or classical liberal ideas—such as free markets or self-reliance —as falling outside the boundaries of routine, and sometimes legitimate, intellectual investigation.
  The AAAS displays great enthusiasm for liberal education. Yet its report may well set back reform by obscuring the depth and breadth of the challenge that Congress asked it to illuminate.
\begin{questions} \sethlcolor{cyan}\question 36.According to Paragraph 1, what is the author’s attitude toward the AAAS’s report?
\\ \fourch{ Critical
}{ Appreciative
}{ Contemptuous
}{ Tolerant
 }\question 37. Influential figures in the Congress required that the AAAS report on how to \ltk{}.
\\ \fourch{ safeguard individuals’ rights to education
}{ define the government’s role in education
}{ retain people’s interest in liberal education
}{ keep a leading position in liberal education
}\question 38. According to Paragraph 3, the report suggests \ltk{}.
\\ \fourch{ an exclusive study of American history
}{ a greater emphasis on theoretical subjects
}{ the application of emerging technologies
}{ funding for the study of foreign languages
}\question 39. The author implies in Paragraph 5 that professors are \ltk{}.
\\ \fourch{ supportive of free markets
}{ biased against classical liberal ideas
}{ cautious about intellectual investigation
}{ conservative about public policy
}\question 40. Which of the following would be the best title for the text?
\\ \fourch{ Illiberal Education and “The Heart of the Matter”
}{ The AAAS’s Contribution to Liberal Education
}{ Ways to Grasp “The Heart of the Matter”
}{ Progressive Policy vs. Liberal Education
}\end{questions}  