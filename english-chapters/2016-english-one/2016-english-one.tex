
\section{2016}
\subsection{Text 1}
  France, which prides itself as the global innovator of fashion, has decided its fashion industry has lost an absolute right to define physical beauty for woman. Its lawmakers gave preliminary approval last week to a law that would make it a crime to employ ultra-thin models on runways.The parliament also agreed to ban websites that”incite excessive thinness” by promoting extreme dieting.
  Such measures have a couple of uplifting motives. They suggest beauty should not be defined by looks that end up with impinging on health. That’s a start. And the ban on ultra-thin models seems to go beyond protecting models from starring themselves to health –as some have done. It tells the fashion industry that it move take responsibility for the signal it sends women, especially teenage girls, about the social tape –measure they must use to determine their individual worth.
  The bans, if fully enforced ,would suggest to woman (and many men )that they should not let others be orbiters of their beauty .And perhaps faintly, they hint that people should look to intangible qualities like character and intellect rather than dieting their way to sine zero or wasp-waist physiques .
     The French measures, however, rely too much on severe punishment to change a culture that still regards beauty as skin-deep-and bone-showing. Under the law, using a fashion model that does not meet a government-defined index of body mess could result in a \$85,000 fine and six months in prison.
     The fashion industry knows it has an inherent problem in focusing on material adornment and idealized body types. In Denmark, the United States, and a few other countries, it is trying to set voluntary standard for models and fashion images there rely more on pear pressure for enforcement.In contrast to France’s actions, Denmark’s fashion industry agreed last month on rules and sanctions regarding age, health, and other characteristics of models .The newly revised Danish Fashion Ethical charter clearly states, we are aware of and take responsibility for the impact the fashion industry has on body ideals, especially on young people. The charter’s main toll of enforcement is to deny access for designers and modeling agencies to Copenhagen. Fashion week, which is men by the Danish Fashion Institute .But in general it relies on a name-and –shame method of compliance.Relying on ethical persuasion rather than law to address the misuse of body ideals may be the best step. Even better would be to help elevate notions of beauty beyond the material standards of a particular industry.
  \begin{questions} \sethlcolor{cyan}\question 21. According to the first paragraph, what would happen in France?
  \\ \fourch{ Physical beauty would be redefined
  }{ New runways would be constructed
  }{ Websites about dieting would thrive
  }{ The fashion industry would decline
  }\question 22. The phrase “impinging on”(Line2 Para2) is closest in meaning to
  \\ \fourch{ heightening the value of
  }{ indicating the state of
  }{ losing faith in
  }{ doing harm to
  }\question 23.  Which of the following is true of the fashion industry
  \\ \fourch{ The French measures have already failed
  }{ New standards are being set in Denmark
  }{ Models are no longer under peer pressure
  }{ Its inherent problems are getting worse
  }\question 24. A designer is most likely to be rejected by CFW for
  \\ \fourch{ setting perfect physical conditions
  }{ caring too much about models’ character
  }{ showing little concern for health factors
  }{ pursuing a high age threshold for models
  }\question 25. Which of the following maybe the best title of the text?
  \\ \fourch{ A challenge to the Fashion Industry’s Body Ideals
  }{ A Dilemma for the starving models in France
}{ Just Another Round of struggle for beauty   
}{ The Great Threats to the Fashion Industry
 
  }\end{questions}      \subsection{Text 2}
  For the first time in the history more people live in towns than in the country. In Britain this has had a curious result. While polls show Britons rate “the countryside” alongside the royal family. Shakespeare and the National Health Service (NHS) as what make them proudest of their country, this has limited political support. A century ago Octavia Hill launched the National Trust not to rescue stylish houses but to save “the beauty of natural places for everyone forever”. It was specifically to provide city dwellers with spaces for leisure where they could experience “a refreshing air”. Hill’s pressure later led to the creation of national parks and green belts. They don’t make countryside any more, and every year concrete consumes more of it .It needs constant guardianship.
   At the next election none of the big parties seem likely to endorse this sentiment. The Conservatives’ planning reform explicitly gives rural development priority over conservation,
   even authorizing “off–plan” building where local people might object. The concept of sustainable development has been defined as profitable. Labour likewise wants to discontinue local planning where councils oppose development. The Liberal Democrats are silent only u sensing its chance, has sides with those pleading for a more considered approach to using green land. Its campaign to protect Rural England struck terror into many local conservative parties.  
   The sensible place to build new houses factories and offices is where people are in cities and towns where infrastructure is in place. The London agents Stirling Ackroyed recently identified enough sites for half of million houses in the Landon area alone with no intrusion on green belts. What is true of London is even truer of the provinces. The idea that “housing crisis” equals “concreted meadows” is pure lobby talk. The issue is not the need for more houses but, as always, where to put them under lobby pressure, George Osborne favours rural new-build against urban renovation and renewal. He favours out-of-town shopping sites against high streets. This is not a free market but a biased one. Rural towns and villages have grown and will always grow. They do so best where building sticks to their edges and respects their character. We do not ruin urban conservation areas. Why ruin rural ones? 
  Development should be planned, not let trip, After the Netherlands, Britain is Europe’s most crowed country. Half a century of town and country planning has enable it to retain an enviable rural coherence, while still permitting low-density urban living. There is no doubt of the alternative-the corrupted landscapes of southern Portugal, Spain or Ireland. Avoiding this rather than promoting it should unite the left and right of the political spectrum.
\begin{questions} \sethlcolor{cyan}\question 26. Britain’s public sentiment about the countryside
  \\ \fourch{ is not well reflected in politics
  }{ is fully backed by the royal family
  }{ didn’t start fill the Shakespearean age
  }{ has brought much benefit to the NHS
}\question 27. According to paragraph 2,the achievements of the National Trust are now being
  \\ \fourch{ largely overshadowed
  }{ properly protected
  }{ effectively reinforced
  }{ gradually destroyed
}\question 28. Which of the following can be offered from paragraph 3
  \\ \fourch{ Labour is under attack for opposing development
  }{ The Conservatives may abandon “off-plan” building
  }{ Ukip may gain from its support for rural conservation
  }{ The Liberal Democrats are losing political influence
}\question 29. The author holds that George Osbornes’s preference
  \\ \fourch{ shows his disregard for the character of rural area
  }{ stresses the necessity of easing the housing crisis
  }{ highlights his firm stand against lobby pressure
  }{ reveals a strong prejudice against urban areas
}\question 30. In the last paragraph the author show his appreciation of
  \\ \fourch{ the size of population in Britain
  }{ the enviable urban lifestyle in Britain
  }{ the town-and-country planning in Britain
  }{ the political life in today’s Britain
 
  }\end{questions}      \subsection{Text 3}
  “There is one and only one social responsibility of business” wrote Milton Friedman, a Nobel Prize-winning economist “That is, to use its resources and engage in activities designed to increase its profits.” But even if you accept Friedman’s premise and regard corporate social responsibility(CSR) policies as a waste of shareholders’s money, things may not be absolutely clear-act. New research suggests that CSR may create monetary value for companies at least when they are prosecuted for corruption. 
  The largest firms in America and Britain together spend more than \$15 billion a year on CSR, according to an estimate by EPG, a consulting firm. This could add value to their businesses in three ways. First, consumers may take CSR spending as a “signal” that a company’s products are of high quality. Second, customers may be willing to buy a company’s products as an indirect may to donate to the good causes it helps. And third, through a more diffuse “halo effect” whereby its good deeds earn it greater consideration from consumers and others. 
  Previous studies on CSR have had trouble differentiating these effects because consumers can be affected by all three. A recent study attempts to separate them by looking at bribery prosecutions under American’s Foreign Corrupt Practices Act(FCPA).It argues that since prosecutors do not consume a company’s products as part of their investigations,they could be influenced only by the halo effect.The study found that,among prosecuted firms,those with the most comprehensive CSR programmes tended to get more lenient penalties. Their analysis ruled out the possibility that it was firm’s political influence, rather than their CSR stand, that accounted for the leniency: Companies that contributed more to political campaigns did not receive lower fines. 
  In all, the study concludes that whereas prosecutors should only evaluate a case based on its merits, they do seem to be influenced by a company’s record in CSR. “We estimate that either eliminating a substantial labour-rights concern, such as child labour, or increasing corporate giving by about20\% result in fines that generally are 40\% lower than the typical punishment for bribing foreign officials.” says one researcher.
  Researchers admit that their study does not answer the question at how much businesses ought to spend on CSR. Nor does it reveal how much companies are banking on the halo effect, rather than the other possible benefits, when they companies get into trouble with the law, evidence of good character can win them a less costly punishment.
 
 \begin{questions} \sethlcolor{cyan}\question 31. The author views Milton Friedman’s statement about CSR with
  \\ \fourch{uncertainty
  }{skepticism
  }{approval
  }{tolerance
 }\question 32. According to Paragraph 2, CSR helps a company by
  \\ \fourch{guarding it against malpractices
  }{protecting it from consumers
  }{winning trust from consumers.
  }{raising the quality of its products 
 }\question   33. The expression “more lenient”(line 2,Para.4)is closest in meaning to
  \\ \fourch{less controversial
  }{more lasting
  }{more effective
  }{less severe
 }\question 34. When prosecutors evaluate a case, a company’s CSR record
  \\ \fourch{comes across as reliable evidence
  }{has an impact on their decision
  }{increases the chance of being penalized
  }{constitutes part of the investigation
 }\question 35. Which of the following is true of CSR according to the last paragraph?
  \\ \fourch{ The necessary amount of companies spending on it is unknown
  }{ Companies’ financial capacity for it has been overestimated
  }{ Its negative effects on businesses are often overlooked
  }{It has brought much benefit to the banking industry
 
  }\end{questions}      \subsection{Text 4}
  There will eventually come a day when The New York Times ceases to publish stories on newsprint. Exactly when that day will be is a matter of debate. ”Sometime in the future,” the paper’s publisher said back in 2010.
 
  Nostalgia for ink on paper and the rustle of pages aside, there’s plenty of incentive to ditch print. The infrastructure required to make a physical newspaper – printing presses, delivery trucks – isn’t just expensive; it’s excessive at a time when online – only competitors don’t have the same set of financial constraints. Readers are migrating away from print anyway. And though print ad sales still dwarf their online and mobile counterparts, revenue from print is still declining.
   Overhead may be high and circulation lower, but rushing to eliminate its print edition would be a mistake, says BuzzFeed CEO Jonah Peretti.
   Peretti says the Times shouldn’t waste time getting out of the print business, but only if they go about doing it the right way. “Figuring out a way to accelerate that transition would make sense for them,” he said, “but if you discontinue it, you’re going have your most loyal customers really upset with you.”Sometimes that’s worth making a change anyway. Peretti gives the example of Netflix discontinuing its DVD-mailing service to focus on streaming. “It was seen as blunder,” he said. The move turned out to be foresighted. And if Peretti were in charge at the Times? ”I wouldn’t pick a year to end print,” he said “I would raise prices and make it into more of a legacy product.”The most loyal customers would still get the product they favor, the idea goes, and they’d feel like they were helping sustain the quality of something they believe in. “So if you’re overpaying for print, you could feel like you were helping,” Peretti said. “Then increase it at a higher rate each year and essentially try to generate additional revenue.”
   In other words, if you’re going to make a print product, make it for the people who are already obsessed with it. Which may be what the Times is doing already. Getting the print edition seven days a week costs nearly \$500 a year – more than twice as much as a digital – only subscription.“It’s a really hard thing to do and it’s a tremendous luxury that BuzzFeed doesn’t have a legacy business,” Peretti remarked. “But we’re going to have questions like that where we have things we’re doing that don’t make sense when the market changes and the world changes. In those situations, it’s better to be more aggressive that less aggressive.”
  \begin{questions} \sethlcolor{cyan}\question 36. The New York Times is considering ending it’s print edition partly due to
  \\ \fourch{ the increasing online and sales
  }{ the pressure from its investors
  }{ the complaints from its readers
  }{ the high cost of operation
  }\question 37. Peretti suggests that in face of the present situation, The Times should
  \\ \fourch{ make strategic adjustments
  }{ end the print sedition for good
  }{ seek new sources of leadership
  }{ aim for efficient management
 }\question 38. It can be inferred from paragraphs 5and 6 that a ” legacy product”
  \\ \fourch{ helps restore the glory of former times
  }{ is meant for the most loyal customers
  }{ will have the cost of printing reduced
  }{ expands the popularity of the paper
 }\question 39. Peretti believes that in a changing world
  \\ \fourch{ traditional luxuries can stay unaffected
  }{ cautiousness facilitates problem-solving
  }{ aggressiveness better meets challenges
  }{ legacy businesses are becoming out dated
 }\question 40. which of the following would be the best title of the text?
  \\ \fourch{ shift to online newspapers all at once
  }{ Cherish the Newspapers still in Your Hand
  }{ keep Your Newspapers Forever in Fashion
  }{ Make Your print Newspapers a luxury Good
}\end{questions}    