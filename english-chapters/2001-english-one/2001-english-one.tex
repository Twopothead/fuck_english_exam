\section{2001年全真试题}
\subsection{Text 1 科学发展的职业化与专业化}

\ding{195}\sethlcolor{yellow}\hl{Specialization can be seen as a response to the problem of an increasing accumulation of scientific knowledge. By splitting up the subject matter into smaller units, one man could continue to handle the information and use it as the basis for further research}. But specialization was only one of a series of related developments in science affecting the process of communication. Another was the growing professionalisation of scientific activity.

\ding{193}\yhl{No} 
\ann{clear-cut distinction:明确区分}\sethlcolor{yellow}\hl{clear-cut distinction can be drawn between professionals and amateurs in science}: exceptions can be found to any rule. Nevertheless, the word “amateur” does carry a 
\ann{connotation:含义}connotation that the person concerned is not fully integrated into the scientific community and, in particular, may not fully share its values. 
The growth of specialization \underline{in the nineteenth century}, with its 
\ann{consequent requirement:随之而来的要求}\ding{192}\sethlcolor{yellow}\hl{consequent requirement of a longer, more complex training}, 
\ann{imply:暗示,意味着}implied \ding{193}\sethlcolor{yellow}\hl{greater problems for amateur participation} in science. The trend was naturally \ding{192}\sethlcolor{yellow}\hl{most obvious in
those areas of science based especially on a mathematical or laboratory training}, and 
\ding{194}\sethlcolor{yellow}\hl{can be illustrated in terms of the development of geology} in the United Kingdom.

A comparison of British geological publications over the last century and a half 
\ann{reveal:展现,揭示,泄露}reveals not simply an increasing emphasis on the 
\ann{primacy:首要,至高无上}primacy of research, but also a changing definition of what 
\ann{constitute:组成,构成}constitutes an acceptable research paper. Thus, \underline{in the nineteenth century}, local geological studies represented worthwhile research in their own right; but, \underline{in the twentieth century}, local studies have increasingly become acceptable to professionals only if they incorporate, and reflect on, the wider geological picture. Amateurs, on the other hand, have continued to pursue local studies in the old way. The overall result has been to make entrance to professional geological journals harder for amateurs, a result that has been reinforced by the widespread introduction of 
\ann{referee:审阅,鉴定}refereeing, first by national journals in the nineteenth century and then by several local geological journals in the twentieth century. As a logical consequence of this development, separate journals have now appeared aimed mainly towards either professional or amateur readership. A rather similar process of differentiation has led to professional geologists coming together nationally within one or two specific societies, 
\ann{whereas:然而;鉴于;反之}\ghl{whereas} the 
\ding{193}\sethlcolor{yellow}\hl{amateurs have tended either to remain in local societies or to come together nationally} in a different way.

Although the process of professionalisation and specialization was already well under way in British geology during the nineteenth century, its full consequences were thus delayed until the twentieth century. In science generally, however, the nineteenth century must be 
\ann{reckon:看作,认为是。be reckoned as:被认为/看作是}\ghl{reckoned} as the crucial period for this change in the structure of science.
\begin{questions} \sethlcolor{cyan}    \question 52.	\hl{The growth of specialization in the 19th century} might \hl{be more clearly seen} in sciences such as \ltk{}.\\
\fourch{ sociology and chemistry
}{ physics and psychology
}{ sociology and psychology
}{ \textcolor{blue}{physics and chemistry}
}
\begin{solution}
    be more clearly seen=most obvious in
\end{solution} 
 \question 52.	We can infer from the passage that \ltk{}.\\
\fourch{ there is little distinction between specialization and professionalisation \mmark{概念偷换,there is little distinction的是professionals \& amateurs 而不是 specialisation \& professionalisation.}
}{ \textcolor{blue}{amateurs can compete with professionals in some areas of science} \mmark{comptete with:相匹敌}
}{ professionals tend to welcome amateurs into the scientific community
}{ amateurs have national academic societies but no local ones
}  \question 53.	The author writes of the \hl{development of geology to demonstrate} \ltk{}.\\
\fourch{ \textcolor{blue}{ the process of specialization and professionalisation}
}{ 
    the hardship of amateurs in scientific study 
}{ the change of policies in scientific publications
}{ the \underline{discrimination} of professionals against amateurs
}
\begin{solution}
    此题我做对了,毙考题app上答案有误。B与C是职业化和专业化过程中的具体表现。
\end{solution}  
\question 54.	The \hl{direct reason for specialization} is \ltk{}.\\
\fourch{ the development in communication
}{ the growth of professionalisation
}{ \textcolor{blue}{the expansion of scientific knowledge}
}{ the splitting up of academic societies
}\end{questions}    
\subsection{Text 2 Can we end world poverty?}

A great deal of attention is being paid today to the so-called \ding{192}\sethlcolor{yellow}\hl{digital divide -- the division of the world into the info (information) rich and the info poor}. And that divide does exist today. My wife and I lectured about this 
\ann{loom:(问题或困难)降临,逼近}\ding{192}\ghl{looming} \yhl{danger} twenty years ago. What was less visible then, however, were the new, positive forces that work against the digital divide. There are reasons to be optimistic.

There are technological reasons to hope the digital divide will narrow. As the Internet becomes more and more \ding{193}\yhl{commercialized}, it is in the interest of business to universalize access -- after all, the more people online, the more \ding{193}\sethlcolor{yellow}\hl{potential} customers there are. \ding{193}\sethlcolor{yellow}\hl{More and more governments, afraid their countries will be left behind, want to spread Internet access}. Within the next decade or two, one to two billion people on the planet will be netted together. As a result, I now believe the digital divide will narrow rather than widen in the years ahead. And that is very good news because the Internet may well be the most powerful tool for 
\ann{combat:战胜}\ghl{combating} world poverty that we’ve ever had.

Of course, the use of the Internet isn’t the only way to defeat poverty. And the Internet is not the only tool we have. But it has enormous potential.\annmark{承上启下过渡段,再次肯定互联网战胜贫困巨大潜力}

\ding{194}\sethlcolor{yellow}\hl{To take advantage of this tool, some} \ann{impoverished:赤贫的,贫瘠的}\ghl{impoverished} \hl{countries will have to get over their outdated anti-colonial prejudices with respect to} \ding{193}\hl{foreign investment}. Countries that still think foreign investment is an 
\ann{invasion:侵犯}invasion of their 
\ann{sovereignty:主权}\ghl{sovereignty} might well study the history of infrastructure (the basic structural foundations of a society) \ding{194}\hl{in the United States}. When the United States built its industrial infrastructure, it didn’t have the 
\ann{capital:资金}capital to do so. And that is why America’s Second Wave 
\ann{infrastructure:基础设施}infrastructure -- including roads, harbors, highways, ports and so on -- were built with foreign investment. The English, the Germans, the Dutch and the French were investing in Britain’s former colony. They financed them. Immigrant Americans built them. Guess who owns them now? The Americans. I believe the same thing would be true in places like Brazil or anywhere else for that matter. \sethlcolor{yellow}\hl{The more foreign capital you have helping you build your Third Wave infrastructure, which today is an electronic infrastructure, the} 
\ann{better off:境况良好}\hl{better off you}’\hl{re going to be}. That doesn’t mean 
\ann{lie down and become fooled:卑躬屈膝,任人愚弄}lying down and becoming fooled, or letting foreign corporations run uncontrolled. But it does mean recognizing how important they can be in building the energy and telecom infrastructures needed to take full advantage of the Internet.
\begin{questions} \sethlcolor{cyan} \question 55.	\hl{Digital divide} is something \ltk{}.\\
\fourch{ getting worse because of the Internet
}{ the rich countries are responsible for
}{ \textcolor{blue}{the world must guard against} \mmark{gurad against 'this looming danger'}
}{ considered positive today
} \question 56.	\hl{Governments} attach importance to the \hl{Internet} because it \ltk{}.\\
\fourch{ \textcolor{blue}{offers economic potentials}
}{ can bring foreign funds \mmark{颠倒因果。把结果表现当原因是此类题干扰方式}
}{ can soon wipe out world poverty
}{ connects people all over the world
} \question 57.	The writer mentioned \hl{the case of the United States to justify} the policy of \ltk{}.\\
\fourch{ providing financial support overseas
}{ preventing foreign capital’s control
}{ building industrial infrastructure
}{ \textcolor{blue}{accepting foreign investment}
} \question 58.	It seems that now \hl{a country}’\hl{s economy depends much on} \ltk{}.\\
\fourch{ \textcolor{blue}{how well developed it is electronically}
}{ whether it is prejudiced against immigrants
}{ whether it adopts America’s industrial pattern
}{ how much control it has over foreign corporations
}
\end{questions}    \subsection{Text 3 美国报业遭受不信任危机}
Why do so many Americans distrust what they read in their newspapers? The American Society of Newspaper Editors is trying to answer this painful question. The organization is deep into a long self-analysis known as the journalism credibility project.
Sad to say, this project has turned out to be mostly low-level findings about factual errors and spelling and grammar mistakes, combined with lots of head-scratching puzzlement about what in the world those readers really want.
But the sources of distrust go way deeper. Most journalists learn to see the world through a set of standard templates (patterns) into which they plug each day’s events. In other words, there is a conventional story line in the newsroom culture that provides a backbone and a ready-made narrative structure for otherwise confusing news.
There exists a social and cultural disconnect between journalists and their readers, which helps explain why the “standard templates” of the newsroom seem alien to many readers. In a recent survey, questionnaires were sent to reporters in five middle-size cities around the country, plus one large metropolitan area. Then residents in these communities were phoned at random and asked the same questions.
Replies show that compared with other Americans, journalists are more likely to live in upscale neighborhoods, have maids, own Mercedeses, and trade stocks, and they’re less likely to go to church, do volunteer work, or put down roots in a community.
Reporters tend to be part of a broadly defined social and cultural elite, so their work tends to reflect the conventional values of this elite. The astonishing distrust of the news media isn’t rooted in inaccuracy or poor reportorial skills but in the daily clash of world views between reporters and their readers.
This is an explosive situation for any industry, particularly a declining one. Here is a troubled business that keeps hiring employees whose attitudes vastly annoy the customers. Then it sponsors lots of symposiums and a credibility project dedicated to wondering why customers are annoyed and fleeing in large numbers. But it never seems to get around to noticing the cultural and class biases that so many former buyers are complaining about. If it did, it would open up its diversity program, now focused narrowly on race and gender, and look for reporters who differ broadly by outlook, values, education, and class.
\begin{questions} \sethlcolor{cyan}    \question 59.	What is the passage mainly about?
\fourch{ needs of the readers all over the world
}{ causes of the public disappointment about newspapers
}{ origins of the declining newspaper industry
}{ aims of a journalism credibility project
} \question 60.	The results of the journalism credibility project turned out to be \ltk{}.\\
\fourch{ quite trustworthy
}{ somewhat contradictory
}{ very illuminating
}{ rather superficial
} \question 61.	The basic problem of journalists as pointed out by the writer lies in their \ltk{}.\\
\fourch{ working attitude
}{ conventional lifestyle
}{ world outlook
}{ educational background
} \question 62.	Despite its efforts, the newspaper industry still cannot satisfy the readers owing to its \ltk{}.\\
\fourch{ failure to realize its real problem
}{ tendency to hire annoying reporters
}{ likeliness to do inaccurate reporting
}{ prejudice in matters of race and gender
}\end{questions}
\subsection{Text 4}
The world is going through the biggest wave of mergers and acquisitions ever witnessed. The process sweeps from hyperactive America to Europe and reaches the emerging countries with unsurpassed might. Many in these countries are looking at this process and worrying: “Won’t the wave of business concentration turn into an uncontrollable anti-competitive force?”
There’s no question that the big are getting bigger and more powerful. Multinational corporations accounted for less than 20\% of international trade in 1982. Today the figure is more than 25% and growing rapidly. International affiliates account for a fast-growing segment of production in economies that open up and welcome foreign investment. In Argentina, for instance, after the reforms of the early 1990s, multinationals went from 43% to almost 70% of the industrial production of the 200 largest firms. This phenomenon has created serious concerns over the role of smaller economic firms, of national businessmen and over the ultimate stability of the world economy.
I believe that the most important forces behind the massive M\&A wave are the same that underlie the globalization process: falling transportation and communication costs, lower trade and investment barriers and enlarged markets that require enlarged operations capable of meeting customer’s demands. All these are beneficial, not detrimental, to consumers. As productivity grows, the world’s wealth increases.
Examples of benefits or costs of the current concentration wave are scanty. Yet it is hard to imagine that the merger of a few oil firms today could recreate the same threats to competition that were feared nearly a century ago in the U.S., when the Standard Oil trust was broken up. The mergers of telecom companies, such as WorldCom, hardly seem to bring higher prices for consumers or a reduction in the pace of technical progress. On the contrary, the price of communications is coming down fast. In cars, too, concentration is increasing -- witness Daimler and Chrysler, Renault and Nissan -- but it does not appear that consumers are being hurt.
Yet the fact remains that the merger movement must be watched. A few weeks ago, Alan Greenspan warned against the megamergers in the banking industry. Who is going to supervise, regulate and operate as lender of last resort with the gigantic banks that are being created? Won’t multinationals shift production from one place to another when a nation gets too strict about infringements to fair competition? And should one country take upon itself the role of “defending competition” on issues that affect many other nations, as in the U.S. vs. Microsoft case?
\begin{questions} \sethlcolor{cyan} \question 63.	What is the typical trend of businesses today?\\
\fourch{ to take in more foreign funds
}{ to invest more abroad
}{ to combine and become bigger
}{ to trade with more countries
} \question 64.	According to the author, one of the driving forces behind M\&A wave is \ltk{}.\\
\fourch{ the greater customer demands
}{ a surplus supply for the market
}{ a growing productivity
}{ the increase of the world’s wealth
} \question 65.	From paragraph 4 we can infer that \ltk{}.\\
\fourch{ the increasing concentration is certain to hurt consumers
}{ WorldCom serves as a good example of both benefits and costs
}{ the costs of the globalization process are enormous
}{ the Standard Oil trust might have threatened competition
} \question 66.	Toward the new business wave, the writer’s attitude can be said to be \ltk{}.\\
\fourch{ optimistic
}{ objective
}{ pessimistic
}{ biased
}\end{questions}    \subsection{Text 5}
When I decided to quit my full time employment it never occurred to me that I might become a part of a new international trend. A lateral move that hurt my pride and blocked my professional progress prompted me to abandon my relatively high profile career although, in the manner of a disgraced government minister, I covered my exit by claiming “I wanted to spend more time with my family”.
Curiously, some two-and-a-half years and two novels later, my experiment in what the Americans term “downshifting” has turned my tired excuse into an absolute reality. I have been transformed from a passionate advocate of the philosophy of “having it all,” preached by Linda Kelsey for the past seven years in the page of She magazine, into a woman who is happy to settle for a bit of everything.
I have discovered, as perhaps Kelsey will after her much-publicized resignation from the editorship of She after a build-up of stress, that abandoning the doctrine of “juggling your life,” and making the alternative move into “downshifting” brings with it far greater rewards than financial success and social status. Nothing could persuade me to return to the kind of life Kelsey used to advocate and I once enjoyed: 12-hour working days, pressured deadlines, the fearful strain of office politics and the limitations of being a parent on “quality time”.
In America, the move away from juggling to a simpler, less materialistic lifestyle is a well-established trend. Downshifting -- also known in America as “voluntary simplicity” -- has, ironically, even bred a new area of what might be termed anti-consumerism. There are a number of best-selling downshifting self-help books for people who want to simplify their lives; there are newsletters, such as The Tightwad Gazette, that give hundreds of thousands of Americans useful tips on anything from recycling their cling-film to making their own soap; there are even support groups for those who want to achieve the mid-’90s equivalent of dropping out.
While in America the trend started as a reaction to the economic decline -- after the mass redundancies caused by downsizing in the late ’80s -- and is still linked to the politics of thrift, in Britain, at least among the middle-class downshifters of my acquaintance, we have different reasons for seeking to simplify our lives.
For the women of my generation who were urged to keep juggling through the ’80s, downshifting in the mid-’90s is not so much a search for the mythical good life -- growing your own organic vegetables, and risking turning into one -- as a personal recognition of your limitations.
\begin{questions} \sethlcolor{cyan} \question 67.	Which of the following is true according to paragraph 1?\\
\fourch{ Full-time employment is a new international trend.
}{ The writer was compelled by circumstances to leave her job.
}{ “A lateral move” means stepping out of full-time employment.
}{ The writer was only too eager to spend more time with her family.
}    \question 68.	The writer’s experiment shows that downshifting \ltk{}.\\
\fourch{ enables her to realize her dream
}{ helps her mold a new philosophy of life
}{ prompts her to abandon her high social status
}{ leads her to accept the doctrine of She magazine
}    \question 69.	“Juggling one’s life” probably means living a life characterized by \ltk{}.\\
\fourch{ non-materialistic lifestyle
}{ a bit of everything
}{ extreme stress
}{ anti-consumerism
}    \question 70.	According to the passage, downshifting emerged in the U.S. as a result of \ltk{}.\\
\fourch{ the quick pace of modern life
}{ man’s adventurous spirit
}{ man’s search for mythical experiences
}{ the economic situation
}\end{questions}