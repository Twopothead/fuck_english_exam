
\section{2013}
 \subsection{Text 1}
    
    In the 2006 film version of The Devil Wears Prada ,Miranda Priestly, played by Meryl Streep, scolds her unattractive assistant for imagining that high fashion doesn’t affect her, Priestly explains how the deep blue color of the assistant’s sweater descended over the years from fashion shows to departments stores and to the bargain bin in which the poor girl doubtless found her garment.
    
    This top-down conception of the fashion business couldn’t be more out of date or at odds with the feverish would described in Overdressed, Eliazabeth Cline’s three-year indictment of “fast fashion”. In the last decade or so ,advances in technology have allowed mass-market labels such as Zara ,H\&M, and Uniqlo to react to trends more quickly and anticipate demand more precisely. Quicker turnarounds mean less wasted inventory, more frequent release, and more profit. These labels encourage style-conscious consumers to see clothes as disposable-meant to last only a wash or two, although they don’t advertise that –and to renew their wardrobe every few weeks. By offering on-trend items at dirt-cheap prices, Cline argues, these brands have hijacked fashion cycles, shaking an industry long accustomed to a seasonal pace.
    
    The victims of this revolution , of course ,are not limited to designers. For H\&M to offer a \$5.95 knit miniskirt in all its 2,300-pius stores around the world, it must rely on low-wage overseas labor, order in volumes that strain natural resources, and use massive amounts of harmful chemicals.
    
    Overdressed is the fashion world’s answer to consumer-activist bestsellers like Michael Pollan’s The Omnivore’s Dilemma. “Mass-produced clothing ,like fast food, fills a hunger and need, yet is non-durable and wasteful,” Cline argues. Americans, she finds, buy roughly 20 billion garments a year – about 64 items per person – and no matter how much they give away, this excess leads to waste.
    
    Towards the end of Overdressed, Cline introduced her ideal, a Brooklyn woman named Sarah Kate Beaumont, who since 2008 has made all of her own clothes – and beautifully. But as Cline is the first to note, it took Beaumont decades to perfect her craft; her example can’t be knocked off.
    
    Though several fast-fashion companies have made efforts to curb their impact on labor and the environment – including H\&M, with its green Conscious Collection line –Cline believes lasting change can only be effected by the customer. She exhibits the idealism common to many advocates of sustainability, be it in food or in energy. Vanity is a constant; people will only start shopping more sustainably when they can’t afford not to.
    
    \begin{questions} \sethlcolor{cyan}\question 21. Priestly criticizes her assistant for her
    \\ \fourch{ poor bargaining skill.
    }{ insensitivity to fashion.
    }{ obsession with high fashion.
    }{ lack of imagination.
    
    }\question 22. According to Cline, mass-maket labels urge consumers to
    \\ \fourch{ combat unnecessary waste.
    }{ shut out the feverish fashion world.
    }{ resist the influence of advertisements.
    }{ shop for their garments more frequently.
    
    }\question 23.  The word “indictment” (Line 3, Para.2) is closest in meaning to
    \\ \fourch{ accusation.
    }{ enthusiasm.
    }{ indifference.
    }{ tolerance.
    
    }\question 24. Which of the following can be inferred from the lase paragraph?
    \\ \fourch{ Vanity has more often been found in idealists.
    }{ The fast-fashion industry ignores sustainability.
    }{ People are more interested in unaffordable garments.
    }{ Pricing is vital to environment-friendly purchasing.
    
    }\question 25. What is the subject of the text?
    \\ \fourch{ Satire on an extravagant lifestyle.
    }{ Challenge to a high-fashion myth.
    }{ Criticism of the fast-fashion industry.
    }{ Exposure of a mass-market secret.
    
 }\end{questions}      \subsection{Text 2}
    
    An old saying has it that half of all advertising budgets are wasted-the trouble is, no one knows which half . In the internet age, at least in theory ,this fraction can be much reduced . By watching what people search for, click on and say online, companies can aim “behavioural” ads at those most likely to buy.
    
    In the past couple of weeks a quarrel has illustrated the value to advertisers of such fine-grained information: Should advertisers assume that people are happy to be tracked and sent behavioural ads? Or should they have explicit permission?
    
    In December 2010 America's Federal Trade Cornmission (FTC) proposed adding a "do not track "(DNT) option to internet browsers ,so that users could tell adwertisers that they did not want to be followed .Microsoft's Internet Explorer and Apple's Safari both offer DNT ;Google's Chrome is due to do so this year. In February the FTC and Digltal Adwertising Alliance (DAA) agreed that the industry would get cracking on responging to DNT requests.
    
    On May 31st Microsoft Set off the row: It said that Internet Explorer 10,the version due to appear windows 8, would have DNT as a default.
    
    It is not yet clear how advertisers will respond. Geting a DNT signal does not oblige anyone to stop tracking, although some companies have promised to do so. Unable to tell whether someone really objects to behavioural ads or whether they are sticking with Microsoft’s default, some may ignore a DNT signal and press on anyway.
    
    Also unclear is why Microsoft has gone it alone. Atter all, it has an ad business too, which it says will comply with DNT requests, though it is still working out how. If it is trying to upset Google, which relies almost wholly on default will become the norm. DNT does not seem an obviously huge selling point for windows 8-though the firm has compared some of its other products favourably with Google's on that count before. Brendon Lynch, M
    
    Microsoft's chief privacy officer, bloggde:"we believe consumers should have more control." Could it really be that simple?
    
    \begin{questions} \sethlcolor{cyan}\question 26. It is suggested in paragraph 1 that “behavioural” ads help advertisers to:
    \\ \fourch{ ease competition among themselves
    }{ lower their operational costs
    }{ avoid complaints from consumers
    }{ provide better online services
    
    }\question 27. “The industry” (Line 6,Para.3) refers to:
    \\ \fourch{ online advertisers
    }{ e-commerce conductors
    }{ digital information analysis
    }{ internet browser developers
    
    }\question 28. Bob Liodice holds that setting DNT as a default
    \\ \fourch{ many cut the number of junk ads
    }{ fails to affect the ad industry
    }{ will not benefit consumers
    }{ goes against human nature
    
    }\question 29. which of the following is ture according to Paragraph.6?
    \\ \fourch{ DNT may not serve its intended purpose
    }{ Advertisers are willing to implement DNT
    }{ DNT is losing its popularity among consumers
    }{ Advertisers are obliged to offer behavioural ads
    
    }\question 30. The author's attitude towards what Brendon Lynch said in his blog is one of:
    \\ \fourch{ indulgence
    }{ understanding
    }{ appreciaction
    }{ skepticism
    
}\end{questions}      \subsection{Text 3}
    
    Up until a few decades ago, our visions of the future were largely - though by no means uniformly - glowingly positive. Science and technology would cure all the ills of humanity, leading to lives of fulfillment and opportunity for all.
    
    Now utopia has grown unfashionable, as we have gained a deeper appreciation of the range of threats facing us, from asteroid strike to epidemic flu and to climate change. You might even be tempted to assume that humanity has little future to look forward to.
    
    But such gloominess is misplaced. The fossil record shows that many species have endured for millions of years - so why shouldn't we? Take a broader look at our species' place in the universe, and it becomes clear that we have an excellent chance of surviving for tens, if not hundreds, of thousands of years . Look up Homo sapiens in the "Red List" of threatened species of the International Union for the Conversation of Nature (IUCN) ,and you will read: "Listed as Least Concern as the species is very widely distributed, adaptable, currently increasing, and there are no major threats resulting in an overall population decline."
    
    So what does our deep future hold? A growing number of researchers and organisations are now thinking seriously about that question. For example, the Long Now Foundation has its flagship project a medical clock that is designed to still be marking time thousands of years hence .
    
    Perhaps willfully , it may be easier to think about such lengthy timescales than about the more immediate future. The potential evolution of today's technology, and its social consequences, is dazzlingly complicated, and it's perhaps best left to science fiction writers and futurologists to explore the many possibilities we can envisage. That's one reason why we have launched Arc, a new publication dedicated to the near future.
    
    But take a longer view and there is a surprising amount that we can say with considerable assurance. As so often, the past holds the key to the future: we have now identified enough of the long-term patterns shaping the history of the planet, and our species, to make evidence-based forecasts about the situations in which our descendants will find themselves.
    
    This long perspective makes the pessimistic view of our prospects seem more likely to be a passing fad. To be sure, the future is not all rosy. But we are now knowledgeable enough to reduce many of the risks that threatened the existence of earlier humans, and to improve the lot of those to come.
    
    \begin{questions} \sethlcolor{cyan}\question 31. Our vision of the future used to be inspired by
    \\ \fourch{ our desire for lives of fulfillment
    }{ our faith in science and technology
    }{ our awareness of potential risks
    }{ our belief in equal opportunity
    
    }\question 32. The IUCN’s “Red List” suggest that human being are
    \\ \fourch{ a sustained species
    }{ a threaten to the environment
    }{ the world’s dominant power
    }{ a misplaced race
    
    }\question   33. Which of the following is true according to Paragraph 5?
    \\ \fourch{ Arc helps limit the scope of futurological studies.
    }{ Technology offers solutions to social problem.
    }{ The interest in science fiction is on the rise.
    }{ Our Immediate future is hard to conceive.
    
    }\question 34. To ensure the future of mankind, it is crucial to
    \\ \fourch{ explore our planet’s abundant resources
    }{ adopt an optimistic view of the world
    }{ draw on our experience from the past
    }{ curb our ambition to reshape history
    
    }\question 35. Which of the following would be the best title for the text?
    \\ \fourch{ Uncertainty about Our Future
    }{ Evolution of the Human Species
    }{ The Ever-bright Prospects of Mankind
    }{ Science, Technology and Humanity
    
}\end{questions}      \subsection{Text 4}
    
    On a five to three vote,the Supreme Court knocked out much of Arizona's immigration law Monday-a modest policy victory for the Obama Aministration.But on the more important matter of the Constitution,the decision was an 8-0 defeat for the federal government and the states.
    
    An arizona.United States,the majority overturned three of the four contested provisions of Arizona's controversial plan to have state and local police enfour federal immigrations law.The Constitutional principles that Washington alone has the power to "establish a uniform Rule of Anturalization" and that federal laws precede state laws are noncontroversial.Arizona had attempted to fashion state police that ran to the existing federal ones.
    
    Justice Anthony Kennedy,joined by Chief Justice John Roberts and the Court's liberals,ruled that the state flew too close to the federal sun .On the overturned provisions the majority held the congress had deliberately "occupied the field " and Arizona had thus intruded on the federal's privileged powers
    
    However,the Justices said that Arizona police would be allowed to verify the legal status of people who come in contact with law enforcement.That’s because Congress has always envisioned joint federal-state immigration enforcement and explicitly encourages state officers to share information and cooperate with federal colleagues.
    
    Two of the three objecting Justice-Samuel Alito and Clarence Thomas-agreed with this Constitutional logic but disagreed about which Arizona rules conflicted with the federal statute. The only major objection came from Justice Antonin Scalia,who offered an even more robust defense of state privileges going back to the alien and Sedition Acts.
    
    The 8-0 objection to President Obama tures on what Justice Samuel Alito describes in his objection as “a shocking assertion assertion of federal executive power”. The White House argued tha Arizona’s laws conflicted with its enforcement priorities, even if state laws complied with federal statutes to the letter. In effect, the White House claimed that it could invalidate any otherwise legitimate state law that it disagrees with.
    
    Some powers do belong exclusively to the federal government,and control of citizenship and the borders is among them. But if Congress wanted to prevent states from using their own resources to check immigration status,it could.It could.It never did so.The administration was in essence asserting that because it didn't want to carry out Congress's immigration wishes,no state should be allowed to do so either.Every Justice rightly rejected this remarkable claim.
    
    \begin{questions} \sethlcolor{cyan}\question 36. Three provisions of Arizona’s plan were overturned because they
    \\ \fourch{ deprived the federal police of Constitutional powers.
    }{ disturbed the power balance between different states.
    }{ overstepped the authority of federal immigration law.
    }{ contradicted both the federal and state policies.
    
     }\question 37. On which of the following did the Justices agree,according to Paragraph4?
    \\ \fourch{ Federal officers’ duty to withhold immigrants’information.
    }{ States’ independence from federal immigration law.
    }{ States’ legitimate role in immigration enforcement.
    }{ Congress’s intervention in immigration enforcement.
    
    }\question 38. It can be inferred from Paragraph 5 that the Alien and Sedition Acts
    \\ \fourch{ violated the Constitution.
    }{ undermined the states’ interests.
    }{ supported the federal statute.
    }{ stood in favor of the states.
    
    }\question 39. The White House claims that its power of enforcement
    \\ \fourch{ outweighs that held by the states.
    }{ is dependent on the states’ support.
    }{ is established by federal statutes.
    }{ rarely goes against state laws.
    
    }\question 40. What can be learned from the last paragraph?
    \\ \fourch{ Immigration issues are usually decided by Congress.
    }{ Justices intended to check the power of the Administrstion.
    }{ Justices wanted to strengthen its coordination with Congress.
    }{ The Administration is dominant over immigration issues.

}\end{questions} 