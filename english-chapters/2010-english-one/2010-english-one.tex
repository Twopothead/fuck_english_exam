
\section{2010}
\subsection{Text 1}
Of all the changes that have taken place in English-language newspapers during the past quarter-century, perhaps the most far-reaching has been the inexorable decline in the scope and seriousness of their arts coverage. 
It is difficult to the point of impossibility for the average reader under the age of forty to imagine a time when high-quality arts criticism could be found in most big-city newspapers. Yet a considerable number of the most significant collections of criticism published in the 20th century consisted in large part of newspaper reviews. To read such books today is to marvel at the fact that their learned contents were once deemed suitable for publication in general-circulation dailies.
We are even farther removed from the unfocused newspaper reviews published in England between the turn of the 20th century and the eve of World War II, at a time when newsprint was dirt-cheap and stylish arts criticism was considered an ornament to the publications in which it appeared. In those far-off days, it was taken for granted that the critics of major papers would write in detail and at length about the events they covered. Theirs was a serious business, and even those reviewers who wore their learning lightly, like George Bernard Shaw and Ernest Newman, could be trusted to know what they were about. These men believed in journalism as a calling, and were proud to be published in the daily press. “So few authors have brains enough or literary gift enough to keep their own end up in journalism,” Newman wrote, “that I am tempted to define ‘journalism’ as ‘a term of contempt applied by writers who are not read to writers who are.’”
Unfortunately, these critics are virtually forgotten. Neville Cardus, who wrote for the Manchester Guardian from 1917 until shortly before his death in 1975, is now known solely as a writer of essays on the game of cricket. During his lifetime, though, he was also one of England’s foremost classical-music critics, a stylist so widely admired that his Autobiography (1947) became a best-seller. He was knighted in 1967, the first music critic to be so honored. Yet only one of his books is now in print, and his vast body of writings on music is unknown save to specialists.
Is there any chance that Cardus’s criticism will enjoy a revival? The prospect seems remote. Journalistic tastes had changed long before his death, and postmodern readers have little use for the richly upholstered Vicwardian prose in which he specialized. Moreover, the amateur tradition in music criticism has been in headlong retreat.
\begin{questions} \sethlcolor{cyan}\question 21. It is indicated in Paragraphs 1 and 2 that .
\\ \fourch{ arts criticism has disappeared from big-city newspapers
}{ English-language newspapers used to carry more arts reviews
}{ high-quality newspapers retain a large body of readers
}{ young readers doubt the suitability of criticism on dailies
}\question 22. Newspaper reviews in England before World War II were characterized by .
\\ \fourch{ free themes
}{ casual style
}{ elaborate layout
}{ radical viewpoints
}\question 23.  Which of the following would Shaw and Newman most probably agree on?
\\ \fourch{ It is writers’ duty to fulfill journalistic goals.
}{ It is contemptible for writers to be journalists.
}{ Writers are likely to be tempted into journalism.
}{ Not all writers are capable of journalistic writing.
}\question 24. What can be learned about Cardus according to the last two paragraphs?
\\ \fourch{ His music criticism may not appeal to readers today.
}{ His reputation as a music critic has long been in dispute.
}{ His style caters largely to modern specialists.
}{ His writings fail to follow the amateur tradition.
}\question 25. What would be the best title for the text?
\\ \fourch{ Newspapers of the Good Old Days
}{ The Lost Horizon in Newspapers
}{ Mournful Decline of Journalism
}{ Prominent Critics in Memory
}\end{questions}      \subsection{Text 2}
Over the past decade, thousands of patents have been granted for what are called business methods. Amazon.com received one for its "one-click" online payment system. Merrill Lynch got legal protection for an asset allocation strategy. One inventor patented a technique for lifting a box.
Now the nation's top patent court appears completely ready to scale back on business-method patents, which have been controversial ever since they were first authorized 10 years ago. In a move that has intellectual-property lawyers abuzz the U.S. court of Appeals for the federal circuit said it would use a particular case to conduct a broad review of business-method patents. In re Bilski, as the case is known , is "a very big deal", says Dennis D. Crouch of the University of Missouri School of law. It "has the potential to eliminate an entire class of patents."
Curbs on business-method claims would be a dramatic about-face, because it was the federal circuit itself that introduced such patents with is 1998 decision in the so-called state Street Bank case, approving a patent on a way of pooling mutual-fund assets. That ruling produced an explosion in business-method patent filings, initially by emerging internet companies trying to stake out exclusive rights to specific types of online transactions. Later, move established companies raced to add such patents to their files, if only as a defensive move against rivals that might beat them to the punch. In 2005, IBM noted in a court filing that it had been issued more than 300 business-method patents despite the fact that it questioned the legal basis for granting them. Similarly, some Wall Street investment films armed themselves with patents for financial products, even as they took positions in court cases opposing the practice.
The Bilski case involves a claimed patent on a method for hedging risk in the energy market. The Federal circuit issued an unusual order stating that the case would be heard by all 12 of the court's judges, rather than a typical panel of three, and that one issue it wants to evaluate is whether it should "reconsider" its state street Bank ruling.
The Federal Circuit's action comes in the wake of a series of recent decisions by the supreme Court that has narrowed the scope of protections for patent holders. Last April, for example the justices signaled that too many patents were being upheld for "inventions" that are obvious. The judges on the Federal circuit are "reacting to the anti-patent trend at the Supreme Court", says Harold C. Wegner, a patent attorney and professor at George Washington University Law School.
\begin{questions} \sethlcolor{cyan}\question 26. Business-method patents have recently aroused concern because of .
\\ \fourch{ their limited value to business
}{ their connection with asset allocation
}{ the possible restriction on their granting
}{ the controversy over authorization
}\question 27. Which of the following is true of the Bilski case?
\\ \fourch{ Its ruling complies with the court decisions.
}{ It involves a very big business transaction.
}{ It has been dismissed by the Federal Circuit.
}{ It may change the legal practices in the U.S..
}\question 28. The word “about-face” (Line 1, Para 3) most probably means .
\\ \fourch{ loss of good will
}{ increase of hostility
}{ change of attitude
}{ enhancement of dignity
}\question 29. We learn from the last two paragraphs that business-method patents.
\\ \fourch{ are immune to legal challenges
}{ are often unnecessarily issued
}{ lower the esteem for patent holders
}{ increase the incidence of risks
}\question 30. Which of the following would be the subject of the text?
\\ \fourch{ A looming threat to business-method patents
}{ Protection for business-method patent holders
}{ A legal case regarding business-method patents
}{ A prevailing trend against business-method patents
}\end{questions}      \subsection{Text 3}
In his book The Tipping Point, Malcolm Gladwell argues that social epidemics are driven in large part by the acting of a tiny minority of special individuals, often called influentials, who are unusually informed, persuasive, or well-connected. The idea is intuitively compelling, but it doesn't explain how ideas actually spread.
The supposed importance of influentials derives from a plausible sounding but largely untested theory called the "two step flow of communication": Information flows from the media to the influentials and from them to everyone else. Marketers have embraced the two-step flow because it suggests that if they can just find and influence the influentials, those selected people will do most of the work for them. The theory also seems to explain the sudden and unexpected popularity of certain looks, brands, or neighborhoods. In many such cases, a cursory search for causes finds that some small group of people was wearing, promoting, or developing whatever it is before anyone else paid attention. Anecdotal evidence of this kind fits nicely with the idea that only certain special people can drive trends
In their recent work, however, some researchers have come up with the finding that influentials have far less impact on social epidemics than is generally supposed. In fact, they don't seem to be required of all.
The researchers' argument stems from a simple observing about social influence, with the exception of a few celebrities like Oprah Winfrey—whose outsize presence is primarily a function of media, not interpersonal, influence—even the most influential members of a population simply don't interact with that many others. Yet it is precisely these non-celebrity influentials who, according to the two-step-flow theory, are supposed to drive social epidemics by influencing their friends and colleagues directly. For a social epidemic to occur, however, each person so affected, must then influence his or her own acquaintances, who must in turn influence theirs, and so on; and just how many others pay attention to each of these people has little to do with the initial influential. If people in the network just two degrees removed from the initial influential prove resistant, for example from the initial influential prove resistant, for example the cascade of change won't propagate very far or affect many people.
Building on the basic truth about interpersonal influence, the researchers studied the dynamics of populations by conducting thousands of computer simulations of populations, manipulating a number of variables relating to people’s ability to influence others and their tendency to be influenced. They found that the principal requirement for what is called “global cascades”—the widespread propagation of influence through networks—is the presence not of a few influentials but, rather, of a critical mass of easily influenced people.
\begin{questions} \sethlcolor{cyan}\question 31. By citing the book The Tipping Point, the author intends to .
\\ \fourch{analyze the consequences of social epidemics
}{discuss influentials’ function in spreading ideas
}{exemplify people’s intuitive response to social epidemics
}{describe the essential characteristics of influentials
}\question 32. The author suggests that the “two-step-flow theory” .
\\ \fourch{serves as a solution to marketing problems
}{has helped explain certain prevalent trends
}{has won support from influentials
}{requires solid evidence for its validity
}\question  33.What the researchers have observed recently shows that .
\\ \fourch{ the power of influence goes with social interactions
}{ interpersonal links can be enhanced through the media
}{ influentials have more channels to reach the public
}{ most celebrities enjoy wide media attention
}\question 34. The underlined phrase “these people” in Paragraph 4 refers to the ones who .
\\ \fourch{ stay outside the network of social influence
}{ have little contact with the source of influence
}{ are influenced and then influence others
}{ are influenced by the initial influential
}\question 35. What is the essential element in the dynamics of social influence?
\\ \fourch{The eagerness to be accepted.
}{The impulse to influence others.
}{The readiness to be influenced.
}{The inclination to rely on others.
}\end{questions}      \subsection{Text 4}
Bankers have been blaming themselves for their troubles in public. Behind the scenes, they have been taking aim at someone else: the accounting standard-setters. Their rules, moan the banks, have forced them to report enormous losses, and it's just not fair. These rules say they must value some assets at the price a third party would pay, not the price managers and regulators would like them to fetch.
Unfortunately, banks' lobbying now seems to be working. The details may be unknowable, but the independence of standard-setters, essential to the proper functioning of capital markets, is being compromised. And, unless banks carry toxic assets at prices that attract buyers, reviving the banking system will be difficult.
After a bruising encounter with Congress, America's Financial Accounting Standards Board (FASB) rushed through rule changes. These gave banks more freedom to use models to value illiquid assets and more flexibility in recognizing losses on long-term assets in their income statement. Bob Herz, the FASB's chairman, cried out against those who "question our motives." Yet bank shares rose and the changes enhance what one lobby group politely calls "the use of judgment by management."
European ministers instantly demanded that the International Accounting Standards Board (IASB) do likewise. The IASB says it does not want to act without overall planning, but the pressure to fold when it completes it reconstruction of rules later this year is strong. Charlie McCreevy, a European commissioner, warned the IASB that it did "not live in a political vacuum" but "in the real word" and that Europe could yet develop different rules.
It was banks that were on the wrong planet, with accounts that vastly overvalued assets. Today they argue that market prices overstate losses, because they largely reflect the temporary illiquidity of markets, not the likely extent of bad debts. The truth will not be known for years. But bank's shares trade below their book value, suggesting that investors are skeptical. And dead markets partly reflect the paralysis of banks which will not sell assets for fear of booking losses, yet are reluctant to buy all those supposed bargains.
To get the system working again, losses must be recognized and dealt with. America's new plan to buy up toxic assets will not work unless banks mark assets to levels which buyers find attractive. Successful markets require independent and even combative standard-setters. The FASB and IASB have been exactly that, cleaning up rules on stock options and pensions, for example, against hostility from special interests. But by giving in to critics now they are inviting pressure to make more concessions.
\begin{questions} \sethlcolor{cyan}\question 36. Bankers complained that they were forced to .
\\ \fourch{ follow unfavorable asset evaluation rules
}{collect payments from third parties
}{cooperate with the price managers
}{reevaluate some of their assets
 }\question 37. According to the author, the rule changes of the FASB may result in .
\\ \fourch{the diminishing role of management
}{the revival of the banking system
}{the banks’ long-term asset losses
}{the weakening of its independence
}\question 38. According to Paragraph 4, McCreevy objects to the IASB’s attempt to .
\\ \fourch{keep away from political influences
}{evade the pressure from their peers
}{act on their own in rule-setting
}{take gradual measures in reform
}\question 39. The author thinks the banks were “on the wrong planet” in that they .
\\ \fourch{misinterpreted market price indicators
}{exaggerated the real value of their assets
}{neglected the likely existence of bad debts
}{denied booking losses in their sale of assets
}\question 40. The author’s attitude towards standard-setters is one of .
\\ \fourch{satisfaction
}{skepticism
}{objectiveness
}{sympathy
}\end{questions}   