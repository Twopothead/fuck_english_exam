
\section{2009}
\subsection{Text 1}
Habits are a funny thing. We reach for them mindlessly, setting our brains on auto-pilot and relaxing into the unconscious comfort of familiar routine. “Not choice, but habit rules the unreflecting herd,” William Wordsworth said in the 19th century. In the ever-changing 21st century, even the word “habit” carries a negative connotation.
So it seems paradoxical to talk about habits in the same contextas creativity and innovation. But brain researchers have discovered that when we consciously develop new habits, we create parallel paths, and even entirely new brain cells, that can jump our trains of thought onto new, innovative tracks.
Rather than dismissing ourselves as unchangeable creatures of habit, we can instead direct our own change by consciously developing new habits. In fact, the more new things we try — the more we step outside our comfort zone — the more inherently creative we become, both in the workplace and in our personal lives.
But don’t bother trying to kill off old habits; once those ruts of procedure are worn into the brain, they’re there to stay. Instead, the new habits we deliberately press into ourselves create parallel pathways that can bypass those old roads.
“The first thing needed for innovation is a fascination with wonder,” says Dawna Markova, author of The Open Mind. “But we are taught instead to ‘decide,’ just as our president calls himself ‘the Decider.’ ” She adds, however, that “to decide is to kill off all possibilities but one. A good innovational thinker is always exploring the many other possibilities.”
All of us work through problems in ways of which we’re unaware, she says. Researchers in the late 1960 discovered that humans are born with the capacity to approach challenges in four primary ways: analytically, procedurally, relationally (or collaboratively) and innovatively. At the end of adolescence, however, the brain shuts down half of that capacity, preserving only those modes of thought that have seemed most valuable during the first decade or so of life.
The current emphasis on standardized testing highlights analysis and procedure, meaning that few of us inherently use our innovative and collaborative modes of thought. “This breaks the major rule in the American belief system — that anyone can do anything,” explains M. J. Ryan, author of the 2006 book This Year I Will... and Ms. Markova’s business partner. “That’s a lie that we have perpetuated, and it fosters commonness. Knowing what you’re good at and doing even more of it creates excellence.” This is where developing new habits comes in.
\begin{questions} \sethlcolor{cyan}\question 21. In Wordsworth’s view, “habits” is characterized by being .
\\ \fourch{ casual
}{ familiar
}{ mechanical
}{ changeable.
}\question 22. Brain researchers have discovered that the formation of habit can be .
\\ \fourch{ predicted
}{ regulated
}{ traced
}{ guided
}\question 23.  “ruts”(Line 1, Paragraph 4) is closest in meaning to .
\\ \fourch{ tracks
}{ series
}{ characteristics
}{ connections
}\question 24. Dawna Markova would most probably agree that .
\\ \fourch{ ideas are born of a relaxing mind
}{ innovativeness could be taught
}{ decisiveness derives from fantastic ideas
}{ curiosity activates creative minds
}\question 25. Ryan’s comments suggest that the practice of standardized testing
\\ \fourch{ prevents new habits from being formed
}{ no longer emphasizes commonness
}{ maintains the inherent American thinking model
}{ complies with the American belief system
}\end{questions}      \subsection{Text 2}
It is a wise father that knows his own child, but today a man can boost his paternal (fatherly) wisdom — or at least confirm that he’s the kid’s dad. All he needs to do is shell out \$30 for paternity testing kit (PTK) at his local drugstore — and another \$120 to get the results.
More than 60,000 people have purchased the PTKs since they
first become available without prescriptions last years, according to Doug Fogg, chief operating officer of Identigene, which makes the over-the-counter kits. More than two dozen companies sell DNA tests directly to the public, ranging in price from a few hundred dollars to more than \$2500.
Among the most popular: paternity and kinship testing, which adopted children can use to find their biological relatives and families can use to track down kids put up for adoption. DNA testing is also the latest rage among passionate genealogists — and supports businesses that offer to search for a family’s geographic roots .
Most tests require collecting cells by swabbing saliva in the mouth and sending it to the company for testing. All tests require a potential candidate with whom to compare DNA.
But some observers are skeptical. “There is a kind of false precision being hawked by people claiming they are doing ancestry testing,” says Troy Duster, a New York University sociologist. He notes that each individual has many ancestors — numbering in the hundreds just a few centuries back. Yet most ancestry testing only considers a single lineage, either the Y chromosome inherited through men in a father’s line or mitochondrial DNA, which is passed down only from mothers. This DNA can reveal genetic information about only one or two ancestors, even though, for example, just three generations back people also have six other great-grandparents or, four generations back, 14 other great-great-grandparents.
Critics also argue that commercial genetic testing is only as good as the reference collections to which a sample is compared. Databases used by some companies don’t rely on data collected systematically but rather lump together information from different research projects. This means that a DNA database may have a lot of data from some regions and not others, so a person’s test results may differ depending on the company that processes the results. In addition, the computer programs a company uses to estimate relationships may be patented and not subject to peer review or outside evaluation.
\begin{questions} \sethlcolor{cyan}\question 26. In paragraphs 1 and 2, the text shows PTK’s \ltk{}.
\\ \fourch{ easy availability
}{ flexibility in pricing
}{ successful promotion
}{ popularity with households
}\question 27. PTK is used to \ltk{}.
\\ \fourch{ locate one’s birth place
}{ promote genetic research
}{ identify parent-child kinship
}{ choose children for adoption
}\question 28. Skeptical observers believe that ancestry testing fails to\ltk{}.
\\ \fourch{ trace distant ancestors
}{ rebuild reliable bloodlines
}{ fully use genetic information
}{ achieve the claimed accuracy
}\question 29. In the last paragraph, a problem commercial genetic testing faces is \ltk{}.
\\ \fourch{ disorganized data collection
}{ overlapping database building
}{ excessive sample comparison
}{ lack of patent evaluation
}\question 30. An appropriate title for the text is most likely to be\ltk{}.
\\ \fourch{ Fors and Againsts of DNA Testing
}{ DNA Testing and Its Problems
}{ DNA Testing Outside the Lab
}{ Lies Behind DNA Testing
}\end{questions}      \subsection{Text 3}
The relationship between formal education and economic growth in poor countries is widely misunderstood by economists and politicians alike. Progress in both areas is undoubtedly necessary for the social, political, and intellectual development of these and all other societies; however, the conventional view that education should be one of the very highest priorities for promoting rapid economic development in poor countries is wrong. We are fortunate that it is, because building new educational systems there and putting enough people through them to improve economic performance would require two or three generations. The findings of a research institution have consistently shown that workers in all countries can be trained on the job to achieve radically higher productivity and, as a result, radically higher standards of living.
Ironically, the first evidence for this idea appeared in the United States. Not long ago, with the country entering a recession and Japan at its pre-bubble peak, the U.S. workforce was derided as poorly educated and one of primary causes of the poor U.S. economic performance. Japan was, and remains, the global leader in automotive-assembly productivity. Yet the research revealed that the U.S. factories of Honda, Nissan, and Toyota achieved about 95 percent of the productivity of their Japanese counterparts — a result of the training that U.S. workers received on the job.
More recently, while examing housing construction, the researchers discovered that illiterate, non-English-speaking Mexican workers in Houston, Texas, consistently met best-practice labor productivity standards despite the complexity of the building industry’s work.
What is the real relationship between education and economic development? We have to suspect that continuing economic growth promotes the development of education even when governments don’t force it. After all, that’s how education got started. When our ancestors were hunters and gatherers 10,000 years ago, they didn’t have time to wonder much about anything besides finding food. Only when humanity began to get its food in a more productive way was there time for other things.
As education improved, humanity’s productivity potential increased as well. When the competitive environment pushed our ancestors to achieve that potential, they could in turn afford more education. This increasingly high level of education is probably a necessary, but not a sufficient, condition for the complex political systems required by advanced economic performance. Thus poor countries might not be able to escape their poverty traps without political changes that may be possible only with broader formal education. A lack of formal education, however, doesn’t constrain the ability of the developing world’s workforce to substantially improve productivity for the foreseeable future. On the contrary, constraints on improving productivity explain why education isn’t developing more quickly there than it is.
\begin{questions} \sethlcolor{cyan}\question 31. The author holds in paragraph 1 that the importance of education in poor countries \ltk{}.
\\ \fourch{ is subject to groundless doubts
}{ has fallen victim of bias
}{ is conventionally downgraded
}{ has been overestimated
}\question 32. It is stated in paragraph 1 that the construction of a new education system \ltk{}.
\\ \fourch{ challenges economists and politicians
}{ takes efforts of generations
}{ demands priority from the government
}{ requires sufficient labor force
}\question   33. A major difference between the Japanese and U.S workforces is that \ltk{}.
\\ \fourch{ the Japanese workforce is better disciplined
}{ the Japanese workforce is more productive
}{ the U.S workforce has a better education
}{ the U.S workforce is more organize
}\question 34. The author quotes the example of our ancestors to show that education emerged \ltk{}.
\\ \fourch{ when people had enough time
}{ prior to better ways of finding food
}{ when people on longer went hungry
}{ as a result of pressure on government
}\question 35. According to the last paragraph, development of education \ltk{}.
\\ \fourch{ results directly from competitive environments
}{ does not depend on economic performance
}{ follows improved productivity
}{ cannot afford political changes
}\end{questions}      \subsection{Text 4}
The most thoroughly studied intellectuals in the history of the new world are the ministers and political leaders of seventeenth-century New England. According to the standard history of American philosophy, nowhere else in colonial America was “so much importance attached to intellectual pursuits.” According to many books and articles, New England’s leaders established the basic themes and preoccupations of an unfolding, dominant Puritan tradition in American intellectual life.
To take this approach to the New Englanders normally means to start with the Puritans’ theological innovations and their distinctive ideas about the church-important subjects that we may not neglect. But in keeping with our examination of southern intellectual life, we may consider the original Puritans as carriers of European culture, adjusting to New World circumstances. The New England colonies were the scenes of important episodes in the pursuit of widely understood ideals of civility and virtuosity.
The early settlers of Massachusetts Bay included men of impressive education and influence in England. Besides the ninety or so learned ministers who came to Massachusetts churches in the decade after 1629, there were political leaders like John Winthrop, an educated gentleman, lawyer, and official of the Crown before he journeyed to Boston. These men wrote and published extensively, reaching both New World and Old World audiences, and giving New England an atmosphere of intellectual earnestness.
We should not forget, however, that most New Englanders were less well educated. While few crafts men or farmers, let alone dependents and servants, left literary compositions to be analyzed, their thinking often had a traditional superstitious quality. A tailor named John Dane, who emigrated in the late 1630s, left an account of his reasons for leaving England that is filled with signs. Sexual confusion, economic frustrations, and religious hope-all name together in a decisive moment when he opened the Bible, told his father that the first line he saw would settle his fate, and read the magical words: “Come out from among them, touch no unclean thing, and I will be your God and you shall be my people.” One wonders what Dane thought of the careful sermons explaining the Bible that he heard in Puritan churches.
Meanwhile, many settles had slighter religious commitments than Dane’s, as one clergyman learned in confronting folk along the coast who mocked that they had not come to the New World for religion. “Our main end was to catch fish.”
\begin{questions} \sethlcolor{cyan}\question 36. The author notes that in the seventeenth-century New England\ltk{}.
\\ \fourch{ Puritan tradition dominated political life
}{ intellectual interests were encouraged
}{ Politics benefited much from intellectual endeavors
}{ intellectual pursuits enjoyed a liberal environment
 }\question 37. It is suggested in paragraph 2 that New Englanders\ltk{}.
\\ \fourch{ experienced a comparatively peaceful early history
}{ brought with them the culture of the Old World
}{ paid little attention to southern intellectual life
}{ were obsessed with religious innovations
}\question 38. The early ministers and political leaders in Massachusetts Bay\ltk{}.
\\ \fourch{ were famous in the New World for their writings
}{ gained increasing importance in religious affairs
}{ abandoned high positions before coming to the New World
}{ created a new intellectual atmosphere in New England
}\question 39. The story of John Dane shows that less well-educated New
Englanders were often \ltk{}.
\\ \fourch{ influenced by superstitions
}{ troubled with religious beliefs
}{ puzzled by church sermons
}{ frustrated with family earnings
}\question 40. The text suggests that early settlers in New England\ltk{}.
\\ \fourch{ were mostly engaged in political activities
}{ were motivated by an illusory prospect
}{ came from different intellectual backgrounds
}{ left few formal records for later reference
}\end{questions}    