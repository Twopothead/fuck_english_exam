\documentclass[printbox]{BHCexam}
\biaoti{~$2019$~年全国硕士研究生招生考试}
\fubiaoti{考研英语一试卷}
\usepackage{ctex}
\usepackage{palatino}
\usepackage{siunitx}%输入度数符号需要的单位宏包
\usepackage{tikz}
\usepackage{color, soul} %用color, 和 soul 包
\setulcolor{blue} %设置下划线的颜色为蓝
\setstcolor{yellow} %设置overstriking颜色为黄
\sethlcolor{green} %设置高亮显示为绿
\usepackage{ulem}
\usepackage{titlesec}

\usepackage{cooltooltips}
\usepackage{graphicx}
\usepackage{color}
\usepackage{hyperref}

\usepackage{geometry}
\usepackage{color,soul}
\sethlcolor{green}

\def\cool{\texttt{cool}}
\usepackage{tikz}


\newcommand{\annmark}[1]{%
    \textcolor{red}{$\bm\langle$#1$\bm\rangle$}%
}%
\newcommand{\mmark}[1]{%
    \textcolor{magenta}{$\bm\langle$#1$\bm\rangle$}%
}%

\newcommand{\ann}[1]{%
    \begin{tikzpicture}[remember picture, baseline=-0.75ex]%
        \node[coordinate] (inText) {};%
    \end{tikzpicture}%
    \marginpar{%
        \renewcommand{\baselinestretch}{1.0}%
        \begin{tikzpicture}[remember picture]%
            \definecolor{orange}{rgb}{1,0.5,0}%
            \draw node[fill=red!20,rounded corners,text width=\marginparwidth] (inNote){\footnotesize#1};%
    \end{tikzpicture}%
    }%
    \begin{tikzpicture}[remember picture, overlay]%
        \draw[draw = orange, thick]
            ([yshift=-0.2cm] inText)
                -| ([xshift=-0.2cm] inNote.west)
                -| (inNote.west);%
    \end{tikzpicture}%
}%


\setlength{\marginparwidth}{2.0cm}

\newcommand{\ghl}[1]{%
%绿色高亮 标注单词
\sethlcolor{green}\hl{$#1$}%
}
\newcommand{\yhl}[1]{%
%黄色高亮 标注关键句子
\sethlcolor{yellow}
    \hl{$#1$}%
}

\newcommand{\rhl}[1]{%
%红色高亮
\sethlcolor{red}\hl{$#1$}%
}

\newcommand{\chl}[1]{%
%题干高亮
\sethlcolor{cyan}\hl{$#1$}%
}



\sethlcolor{yellow}
\setstcolor{green}
\setulcolor{red} 
% \so{1.letterspacing}

%     \ul{2.underlining}

%     \st{3.striking out}

%     \hl{4.highlighting}

\usetikzlibrary{shapes.geometric, arrows}
\tikzstyle{startstop} = [rectangle, rounded corners, minimum width = 1cm, minimum height=0.5cm,text centered, draw = black]
\tikzstyle{io} = [trapezium, trapezium left angle=10, trapezium right angle=110, minimum width=0.5cm, minimum height=0.5cm, text centered, draw=black]
\tikzstyle{process} = [rectangle, minimum width=2cm, minimum height=0.5cm, text centered, draw=black]
\tikzstyle{decision} = [diamond, aspect = 3, text centered, draw=black]
% 箭头形式
\tikzstyle{arrow} = [->,>=latex]

% 同样的,也可以修改子目录的编号方式,而且各目录编号方式可以不同

% \renewcommand\thesection{\Alph{section}} 
% \renewcommand\thesection{\arabic{subsection}}

\renewcommand\thesection{}
\renewcommand\thesubsection{}
\titleformat{\section}[block]{\color{blue}\Large\bfseries\filcenter}{}{1em}{}
\titleformat{\subsection}[hang]{\bfseries\filcenter}{}{1em}{}
% \renewcommand\thesection{\Roman{section}} 

\geometry{right=2.5cm}

\begin{document}
\maketitle %先注释掉
% %\mininotice
\notice %先注释掉
\printanswers % 我要打印答案

% \AddEnumerateCounter{\chinese}{\chinese}{}
\maketitle  
% 暂时不搞标题节省时间
\tableofcontents
\clearpage

%选择题
% \xuanze


% \subsection{001哲学基本问题及其内容(对哲学的划分)}
% \begin{questions}
% \input{assets/test/helloworld}

% \question 
% 恩格斯把费尔巴哈等旧唯物主义者称为半截子的唯物主义,并指出真正的唯物 主义者在理解现实世界(自然界和历史)时是“按照它本身在每一个不以先入为主的唯心主义 怪想来对待它的人面前所呈现的那样来理解……除此以外,唯物主义并没有别的意义。”这 里的“半截子”主要指的是(  )
% \twoch{恩格斯1}{恩格斯2}{\textcolor{red}{恩格斯3}}{恩格斯4}
% \input{get_problems_detail_tex/problems_detail/computer/003_线性表的存储结构/950}

% \begin{questions}

% \subsection{001哲学基本问题及其内容(对哲学的划分)}
% \input{001_哲学基本问题及其内容(对哲学的划分)/sum}

% \question 抗日战争时期的``三三制''政权

% \fourch{1}{2}{3}{4}
% \fourch
% {\textcolor{red}{是指抗日民主政府在工作人员分配上实行“三三制”原则,即共产党员、非党的左派进步分子和不左不右的中间派各占1/3}}
% {\textcolor{red}{是抗日民族统一战线性质的政权,有利于结成最广泛抗日民族统一战线}}
% {\textcolor{red}{是一切赞成抗日又赞成民主的人们的政权}}
% {\textcolor{red}{是敌后抗战的最好政权形式}}

% % \par\onech{\textcolor{red}{是指抗日民主政府在工作人员分配上实行“三三制”原则,即共产党员、非党的左派进步分子和不左不右的中间派各占1/3}}{\textcolor{red}{是抗日民族统-战线性质的政权,有利于结成最广泛_日民族统一战线}}{\textcolor{red}{是一切赞成抗日又赞成民主的人们的政权}}{\textcolor{red}{是敌后抗战的最好政权形式}}
% \begin{solution}基本背诵内容
% \end{solution}
% \question 决定将中国共产党在抗日战争时期实行的减租减息政策改变为实现``耕者有其田''政策的是(
% )
% \par\twoch{《中国土地法大纲》}{\textcolor{red}{《关于清算、减租及土地问题的指示》}}{《兴国土地法》}{《井冈山土地法》}
% \begin{solution}1946年5月4日,中共中央发出《关于清算、减租及土地问题的指示》(史称《五四指示》),决定将党在抗日战争时期实行的减租减息政策改变为实现``耕者有其田''的政策。《中国土地法大纲》,明确规定废除封建性及半封建性剥削的土地制度,实现耕者有其田的土地制度;《兴国土地法》、《井冈山土地法》制定于土地革命战争时期。故B正确。


% This tex file is generated by get_politics_chapters/index.sh automatically.


% \yuedu
\section{2002年全真试题}


% \question 抗日战争时期的``三三制''政权

% \fourch{1}{2}{3}{4}
\section{2002年全真试题}
\subsection{Text1 Use Humor Effectively}\sethlcolor{green} 
If you intend using humor in your talk to make people smile, you must know how to identify shared experiences and problems. Your humor must be relevant to the audience and should help to show them that you are one of them or that you understand their situation and are 
\ann{赞同(而非同情)}\underline{in sympathy with} 
their point of view. 
\ding{192}\sethlcolor{yellow}\hl{Depending on whom you are addressing, the problems will be different. }
If you are talking to a group of managers, you may refer to the disorganized methods of their secretaries; alternatively if you are addressing secretaries, you may want to comment on their disorganized bosses.

Here is an example, which I heard at a nurses’ convention, of a story which works well because the audience all shared the same view of doctors. A man arrives in heaven and is being shown around by St. Peter. He sees wonderful accommodations, beautiful gardens, sunny weather, and so on. Everyone is very peaceful, polite and friendly until, waiting in a line for lunch, the new arrival is suddenly pushed aside by a man in a white coat, who rushes to the head of the line, 
\ann{营造紧张气氛,乃笑话细节}{grabs his food} and 
\ann{stomp:重踏移动,行进;stomp over:用力跺脚 怒气冲冲独自噔噔走向餐桌}\sethlcolor{green}\hl{stomps over}
 to a table by himself. “Who is that?” the new arrival asked St. Peter. 
\ann{医生自视甚高,自以为是}“Oh, that’s God,” 
\ding{193}
\underline{came the reply, “but sometimes he thinks he’s a doctor.”}

If you are part of the group which you are addressing, you will be in a position to know the experiences and problems which are common to all of you and it’ll be appropriate for you to make a passing remark about the 
\ann{inedible:不能食用的,不能吃的}   \hl{inedible}
canteen food or the chairman’s 
\sethlcolor{green}\ann{notorious:声名狼藉} \hl{notorious}
bad taste in ties. With other audiences you mustn’t attempt to cut in with humor as they will 
\ann{resent:感到愤怒}\ghl{resent} 
an outsider making 
\ann{disparaging:蔑视的,轻蔑的,诽谤的}\sethlcolor{green}\hl{disparaging} 
remarks about their canteen or their chairman. You will be on safer ground if you stick to 
\ann{scapegoats:替罪羊}\ghl{scapegoats} 
\ding{194}\sethlcolor{yellow}\hl{like the Post Office or the telephone system}.
\annmark{// 选择恰当的幽默话题,使幽默奏效}

If you feel awkward being humorous, you must practice 
\ding{195}\sethlcolor{yellow}\hl{so that it becomes more natural}.
 Include a few casual and apparently 
\ann{off-the-cuff:未经准备的,当场的,即席的}\ghl{off-the-cuff}
 remarks which you can deliver in a relaxed and unforced manner. Often it’s the delivery which causes the audience to smile, so speak slowly and remember that a raised eyebrow or an unbelieving look may help to show that you are making a light-hearted remark.
 \annmark{// 讲述幽默的方式}

Look for the humor. It often comes from the unexpected. A 
\ann{twist:曲解}\ghl{twist}
 on a familiar quote “If at first you don’t succeed, give up” or a play on words or on a situation. Search for 
\ann{exaggeration and understatements:夸大其词与轻描淡写}\sethlcolor{green}\hl{exaggeration and understatements}
. Look at your talk and pick out a few words or sentences which you can 
\ann{turn about:转来转去,玩转。这里指挑出你能拿来做文章几个词几个字,注入幽默}turn about 
and 
\ann{inject with:插入,注入}\sethlcolor{green}\hl{inject with}
 humor.
 \annmark{//建议人们刻意寻找幽默,随后提出生成幽默的方法}


\begin{questions} \sethlcolor{cyan}

\question  41.	To \hl{make your humor work}, you should \ltk{}.\\
\fourch{ take advantage of different kinds of audience
}{ make fun of the disorganized people
}{ 
    \textcolor{blue}{address different problems to different people}
}{ show sympathy for your listeners}
\begin{solution}
    show sympathy for 同情    
\end{solution}

\question  42.	The \hl{joke about doctors} implies that, in the eyes of nurses, they are \ltk{}.\\
\fourch{ impolite to new arrivals
}{ 
    \textcolor{red}{very conscious of their godlike role}
}{ entitled to some privileges
}{ very busy even during lunch hours
}
\begin{solution}
    very conscious of 很在意,医生自视甚高,自以为是。讽刺意味 主旨题
\end{solution}

\question  43.	It can be inferred from the text that \hl{public services} \ltk{}. \\
\fourch{ have benefited many people
}{ are the focus of public attention
}{ are an inappropriate subject for humor
}{ \textcolor{blue}{have often been the laughing stock}
}
\begin{solution}
    scapegoats替罪羊;passing remark 顺带的评论;laughing stock 笑柄  
\end{solution}
\question  44.	To achieve the desired result,\hl{ humorous stories} should be \hl{ delivered} \ltk{}.\\
\fourch{in well-worded language
}{ as awkwardly as possible
}{ in exaggerated statements
}{ \textcolor{blue}{as \underline{casually} as possible}}
\begin{solution}
    exaggerated:夸张;well-worded:措辞得当
\end{solution}

\question  45.	The best title for the text may be \ltk{}.\\
\twoch{ 
    \textcolor{red}{Use Humor Effectively}
}{ Various Kinds of Humor
}{ Add Humor to Speech
}{ Different Humor Strategies
}
\end{questions}
\begin{solution}
    标题题原则:概括性;针对性;醒目性。本文深入介绍如何使用幽默。D只在末段提及些具体的幽默策略,违背概括性原则
\end{solution}


\subsection{Text2 Hope:Reunification of Mankind}
Since the 
\ann{dawn:黎明,开端}\ghl{dawn} 
of human 
\ann{ingenuity:创造力}\ghl{ingenuity}
, people have \ding{192}
\ann{devise:想出,设计}\ghl{devised} 
\sethlcolor{yellow}\hl{ever more }
\ann{cunning:巧妙的}\ghl{cunning} 
\sethlcolor{yellow}\hl{tools to cope with work that is }
\ding{195}\st{dangerous, boring, burdensome}, or just plain
\ann{nasty:极差的}\ghl{nasty}
. That compulsion has resulted in robotics -- the science of 
\ann{cofer:授予}\ghl{conferring} 
various human capabilities on machines. And if scientists have yet to create the mechanical version of science fiction, they have begun to come close.

As a result, the modern world is increasingly populated by intelligent 
\ann{gizmos:小玩意,小装置}\ghl{gizmos}
\ding{193}\sethlcolor{yellow}\hl{ whose presence we barely notice} but \ding{193}\sethlcolor{yellow}\hl{whose universal existence has removed much human labor}. \underline{Our factories} 
 \ann{hum:发嗡嗡声 句意:轰鸣着机器人组装臂的节奏声}\ghl{hum}
  to the rhythm of robot assembly arms. \underline{Our banking} is done at 
  \ann{automated teller terminals:自动柜员终端}automated teller terminals
   that thank us with mechanical politeness for the transaction. 
   \underline{Our subway trains} are controlled by tireless robot-drivers. And thanks to the continual 
  \ann{miniaturization:小型化}\ghl{miniaturization} 
  of electronics and micro-mechanics, there are already robot systems that can perform some kinds of brain and bone surgery with 
  \ann{submillimeter:亚毫米}\ghl{submillimeter} 
  accuracy -- far greater precision than highly skilled physicians can achieve with their hands alone.
  \annmark{// 论述机器人技术高度发展}

But if robots are to reach the next stage of laborsaving 
\ann{utility:实用,效用,有用性}utility
, they will have to operate with less human 
\ann{supervision:监控}supervision
 and be able to make at least a few decisions for themselves -- goals that pose a real challenge. “While we know how to \ding{195}\sethlcolor{yellow} \hl{tell a robot to handle a specific error},” says Dave Lavery, manager of a robotics program at NASA, 
 \sethlcolor{yellow} \ding{194} “\hl{we can}’\hl{t yet give a robot enough }‘\hl{common sense}’ \hl{to reliably interact with a dynamic world}.”

Indeed the 
\ann{quest:追求,探索}quest
 for true artificial intelligence has produced very mixed results. Despite a 
 \ann{spell:一段时间}spell 
 of initial optimism in the 1960s and 1970s when it appeared that 
 \ann{transistor:晶体管}transistor 
 circuits and microprocessors might be able to copy the action of the human brain by the year 2010, researchers lately have begun to extend that forecast by decades if not centuries.

What they found, in attempting to \textcolor{blue}{model} thought, is that the human brain’s roughly one hundred billion nerve cells are much more talented -- and human 
\ann{perception:感知能力}\ghl{perception} 
far more complicated -- than previously imagined. They have built  \sethlcolor{yellow} \ding{196} \hl{robots that can} recognize the error of a machine panel by a 
\ann{fraction:少量,一点儿}\ghl{fraction}
 of a millimeter in a controlled factory environment. 
 \sethlcolor{yellow} \ding{196} \hl{But the human mind can glimpse a rapidly changing scene and immediately disregard the 98 percent that is irrelevant}, instantaneously focusing on the monkey at the side of a winding forest road or the single suspicious face in a big crowd. The most advanced computer systems on Earth can’t approach that kind of ability, and neuroscientists still don’t know quite how we do it.

\begin{questions} \sethlcolor{cyan}
\question  46.	\hl{Human ingenuity} was \hl{initially} demonstrated in \ltk{}.\\
\fourch{ the use of machines to produce science fiction
}{ the wide use of machines in manufacturing industry \mmark{第2段的张冠李戴}
}{ 
    \textcolor{red}{the invention of tools for difficult and dangerous work}
    \mmark{首句的句义改写}
}{ the elite’s cunning tackling of dangerous and boring work \mmark{首句的句义杂糅}
}
\begin{solution}
initial=dawn,本题是首句的同义改写。要最初原始社会,B是现代社会。
\end{solution}

\question  47.	The word “\hl{gizmos}” (line 1, paragraph 2) most probably means \ltk{}.\\
\onech{ programs
}{ experts
}{ 
    \textcolor{blue}{devices}
}{ creatures}

\question  48.	According to the text, what is \hl{beyond man}’s ability now is to \hl{design a robot} that can \ltk{}.\\
\fourch{ fulfill delicate tasks like performing brain surgery
}{ interact with human beings verbally
}{ have a little common sense \mmark{错在将enough(common sense)改为a little.}
}{ 
    \textcolor{red}{respond independently to a changing world} \mmark{changing=dynamic,是原句同义改写}
}

\question  49.	Besides reducing human labor, \hl{robots can} also \ltk{}.\\
\fourch{ make a few decisions for themselves
}{ 
    \textcolor{red}{deal with some errors with human intervention}
}{ improve factory environments \mmark{“恶劣环境”偷换概念成“改善工厂环境”。}
}{ cultivate human creativity \mmark{cultivate:培养}
}

\question  50.	The author uses the example of a \hl{monkey to argue that robots are} \ltk{}.\\
\fourch{ expected to copy human brain in internal structure
}{ able to perceive abnormalities immediately
}{ 
    \textcolor{blue}{far less able than human brain in focusing on relevant information}
}{ best used in a controlled environment
}
\end{questions}

\subsection{Text3 Oil Pleasant Surprise}
Could the bad old days of economic decline be about to return? 
\ding{192}\sethlcolor{yellow}\hl{Since OPEC agreed to supply-cuts} in March, the price of 
\ann{crude:天然的,未提炼的}\ghl{crude} oil has jumped to almost \$26 a 
\ann{barrel:桶}\ghl{barrel}, up from less than \$10 last December. This near-tripling of oil prices calls up scary memories of the 1973 oil shock, when prices 
\ann{quadruple:成为四倍}quadrupled, and 1979-80, when they also almost tripled. Both previous shocks resulted in double-digit inflation and global economic decline. So where are the headlines warning of 
\ann{glooom:忧郁,愁闷,无望}\ghl{gloom} and \ann{doom:厄运,死亡}\ghl{doom} this time?

The oil price was given another push up this week when Iraq suspended oil exports. Strengthening economic growth, at the same time as winter 
\ann{grip:紧握}\ghl{grips} the northern 
\ann{hemisphere:半球}hemisphere, could push the price higher still in the short term.

Yet there are good reasons to expect the economic consequences now to be less severe than in the 1970s. 
\ding{193}\sethlcolor{yellow}\hl{In most countries the cost of crude oil now accounts for a smaller share of the price of petrol than it did in the 1970s. In Europe, taxes account for up to four-fifths of the} 
\ann{retail price:零售价}\sethlcolor{green}\hl{retail price}
, so even quite big changes in the price of crude have a more 
\ann{非公开或强烈表达的,暗中的}muted effect on 
\ann{pump:泵 文中用加油的泵指代汽油}pump prices than in the past.

Rich economies are also less dependent on oil than they were, and so less sensitive to \ann{swing:摆动,摇摆 文中比喻油价波动}swings in the oil price. Energy conservation, a shift to other fuels and a decline in the importance of heavy, energy-intensive industries have reduced oil consumption. Software, consultancy and mobile telephones use far less oil than steel or car production. For each dollar of GDP (in constant prices) rich economies now use nearly 50\% less oil than in 1973. 
\ding{194}\sethlcolor{yellow}\hl{The OECD estimates in its latest Economic Outlook that, if oil prices averaged \$22 a barrel for a full year, compared with \$13 in 1998, this would increase the oil import bill in rich economies by only 0.25-0.5\% of GDP. That is less than one-quarter of the income loss in 1974 or 1980}. 
\ann{对比论证:富裕国家今昔对比(纵向),富裕国家与进口石油新兴国家对比(横向)}On the other hand, oil-importing 
\ann{emerging:新兴的,发展初期的 emerging economy:新兴经济体}\ghl{emerging} economies -- to which heavy industry has shifted -- have become more energy-intensive, and so could be more seriously squeezed.

One more reason not to lose sleep over the rise in oil prices is that, unlike the rises in the 1970s, it has not occurred against the background of general commodity-price inflation and global excess demand. A sizable portion of the world is only just emerging from economic decline. The Economist’s commodity price index is broadly unchanging from a year ago. In 1973 commodity prices jumped by 70\%, and in 1979 by almost 30%.

\begin{questions} \sethlcolor{cyan}
\question  51.	The \hl{main reason for the latest rise of oil price} is \ltk{}.\\
\fourch{  global inflation
}{ \textcolor{blue}{reduction in supply} \mmark{发现原因,分清主次.“主要原因/直接原因”多个成因时抓mainly,directly,primarily等选主要原因,次要原因是典型干扰项}
}{ fast growth in economy
}{ Iraq’s suspension of exports}

\question  52.	It can be inferred from the text that the retail price of petrol will go up dramatically if \ltk{}.\\
\twoch{  price of crude rises
}{ commodity prices rise
}{ consumption rises
}{ \textcolor{blue}{oil taxes rise}
}

\question  53.	The estimates in Economic Outlook show that in rich countries \ltk{}.\\
\fourch{  heavy industry becomes more energy-intensive
}{ income loss mainly results from fluctuating crude oil prices
}{ manufacturing industry has been seriously squeezed
}{ \textcolor{blue}{oil price changes have no significant impact on GDP}
}

\question  54.	We can draw a conclusion from the text that \ltk{}.\\
\fourch{\textcolor{blue}{oil-price shocks are less shocking now}
}{ inflation seems irrelevant to oil-price shocks
}{ energy conservation can keep down the oil prices
}{ the price rise of crude leads to the shrinking of heavy industry
}

\question  55.	From the text we can see that the writer seems \ltk{}.\\
\onech{  \textcolor{blue}{optimistic}
}{ sensitive
}{ gloomy
}{ scared
}
\end{questions}

\subsection{Text4 医助自杀之争}
The Supreme Court’s decisions on physician-assisted suicide carry important 
\ann{implications:[常用复数]可能的影响,可能的后果}\ghl{implications} for how medicine seeks to relieve dying patients of pain and suffering.

\sethlcolor{yellow} \ding{192}\hl{Although it ruled that there is no} 
\ann{constitutional:宪法的}\ghl{constitutional} 
\sethlcolor{yellow}\hl{right to physician-assisted suicide}, the Court in effect 
\ding{193}\sethlcolor{yellow}\hl{supported the medical principle of} 
“double effect,” 
\hl{a centuries-old moral principle holding that an action having two effects} -- 
\hl{a good one that is intended and a harmful one that is foreseen}  -- \hl{is permissible} if the actor intends only the good effect.

Doctors have used that principle in recent years to justify using high doses of 
\ann{morphine:吗啡}morphine to control 
\ann{terminally ill patients:晚期病人}terminally ill patients’ pain, even though increasing 
\ann{dosage:剂量}dosages will eventually kill the patient.

Nancy Dubler, director of Montefiore Medical Center, 
\ann{contend:主张,争辩}\ghl{contends} that the principle will 
\ann{shield:保护,庇护}\ghl{shield} doctors who “until now have very, very strongly insisted that they could not give patients sufficient mediation to control their pain if that might 
\ann{hasten death:加速死亡}hasten death.”

George Annas, chair of the health law department at Boston University, 
\ann{maintain:断言(sth)属实,坚持说}maintains that, as long as a doctor 
\ann{prescribe:开(处方)}prescribes a drug for a 
\ann{legitimate:合理的,公正的}\ghl{legitimate} medical purpose, the doctor has done nothing illegal even if the patient uses the drug to hasten death. “It’s like surgery,” he says. “We don’t call those deaths 
\ann{homicide:杀人(者).hom-人,cide-切}homicides because the doctors didn’t intend to kill their patients, although they risked their death. If you’re a physician, you can risk your patient’s suicide as long as you don’t intend their suicide.”

On another level, many in the medical community acknowledge that the assisted-suicide debate has been fueled in part by the despair of patients for whom modern medicine has prolonged the physical 
\ann{agony:巨大痛苦}\ghl{agony} of dying.

Just three weeks before the Court’s ruling on physician-assisted suicide, the National Academy of Science (NAS) released a two-volume report, Approaching Death: Improving Care at the End of Life. It identifies the 
\ding{194}\sethlcolor{yellow}\hl{undertreatment of pain} and the \ding{195}
\ann{aggressive:[贬]攻击性的,大胆的,不顾后果的,冒失的;[褒]强有力的,坚持己见的}\ghl{aggressive} use of “ineffectual and forced medical procedures that 
may prolong and even \ding{195}
\ann{dishonor the period of dying:死的不体面}\sethlcolor{yellow}\hl{dishonor the period of dying}” as the twin problems of end-of-life care.

The profession is taking steps \ding{195}\sethlcolor{yellow}\hl{to require young doctors to train} in 
\ann{hospices:(晚期病人)护理所}\ghl{hospices}, to test knowledge of aggressive pain management therapies, to develop a Medicare billing 
\ann{code:道德准则,行为规范}code for hospital-based care, and to develop new standards for assessing and treating pain at the end of life.

Annas says lawyers can play a key role in insisting that these well-meaning medical initiatives translate into better care. “Large numbers of physicians seem unconcerned with 
\ding{196} \sethlcolor{yellow}\hl{the pain their patients are needlessly and predictably suffering},” to the extent that it constitutes “systematic patient abuse.” He says medical licensing boards “must make it clear… that 
\ding{196}\hl{painful deaths} are 
\ann{presumptively:据推测,据断定}\ghl{presumptively} ones that are 
\sethlcolor{yellow}\hl{incompetently managed} and should result in 
\ann{license suspension:吊销执照}license suspension.”

\begin{questions} \sethlcolor{cyan}
\question 56.	From the first three paragraphs, we learn that \ltk{}.\\
\fourch{
 doctors \sout{used to} increase drug dosages to control their patients’ pain
 \mmark{have used 而不是'过去的'(现在已停止)used to}
 }{
    \textcolor{blue}{it is still illegal for doctors to help the dying end their lives}
    }{
 the Supreme Court strongly opposes physician-assisted suicide}{
 patients have no constitutional right to commit suicide}

\question 57.	Which of the following statements is true according to the text?\\
\fourch{
 Doctors will be held guilty if they risk their patients’ death.}{
 Modern medicine has assisted terminally ill patients in painless recovery.}{
    \textcolor{blue}{The Court ruled that high-dosage pain-relieving medication can be prescribed.\mmark{"double effect"}}
    }{
 A doctor’s medication is no longer justified by his intentions.}

\question 58.	According to the NAS’s report, \hl{one of the problems in end-of-life care} is \ltk{}.\\
\fourch{
 prolonged medical procedures}{
    \textcolor{blue}{inadequate treatment of pain} \mmark{=undertreatment of pain}
    }{
 systematic drug abuse}{
 insufficient hospital care}
 \begin{solution}
     两大问题:一是病痛不及时处理;二是大胆使用无效而强制性的医疗过程以延长死亡期,死得没尊严
 \end{solution}

\question 59.	Which of the following best defines the word “\hl{aggressive}” (line 3, paragraph 7)?\\
\twoch{
    \textcolor{red}{Bold}
    \mmark{bold:大胆的;醒目的}
    }{
 Harmful}{
 Careless}{
 Desperate
 \mmark{不顾一切的}
 }

\question 60.	George Annas would probably agree that \hl{doctors should be punished if they} \ltk{}.\\
\fourch{
 manage their patients incompetently \mmark{治疗病人不力。偷换概念:painful deaths偷换成patients}}{
 give patients more medicine than needed}{
 reduce drug dosages for their patients}{
    \textcolor{red}{prolong the needless suffering of the patients \mmark{延长病人不必要的痛苦}}
    }

\end{questions}




% \ann{inedible:不能食用的,不能吃的}\hl{inedible} 
% \end{solution}

% \end{questions}

\end{document}
